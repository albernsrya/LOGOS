%
% This is an example LaTeX file which uses the SANDreport class file.
% It shows how a SAND report should be formatted, what sections and
% elements it should contain, and how to use the SANDreport class.
% It uses the LaTeX article class, but not the strict option.
% ItINLreport uses .eps logos and files to show how pdflatex can be used
%
% Get the latest version of the class file and more at
%    http://www.cs.sandia.gov/~rolf/SANDreport
%
% This file and the SANDreport.cls file are based on information
% contained in "Guide to Preparing {SAND} Reports", Sand98-0730, edited
% by Tamara K. Locke, and the newer "Guide to Preparing SAND Reports and
% Other Communication Products", SAND2002-2068P.
% Please send corrections and suggestions for improvements to
% Rolf Riesen, Org. 9223, MS 1110, rolf@cs.sandia.gov
%

\documentclass[pdf,12pt]{INLreport}
% pslatex is really old (1994).  It attempts to merge the times and mathptm packages.
% My opinion is that it produces a really bad looking math font.  So why are we using it?
% If you just want to change the text font, you should just \usepackage{times}.
% \usepackage{pslatex}
\usepackage{times}
\usepackage[FIGBOTCAP,normal,bf,tight]{subfigure}
\usepackage{amsmath}
\usepackage{amssymb}
\usepackage{soul}
\usepackage{pifont}
\usepackage{enumerate}
\usepackage{listings}
\usepackage{fullpage}
\usepackage{xcolor}          % Using xcolor for more robust color specification
\usepackage{ifthen}          % For simple checking in newcommand blocks
\usepackage{textcomp}
\usepackage{mathtools}
\usepackage{relsize}
\usepackage{lscape}
\usepackage[toc,page]{appendix}

\graphicspath{{./figures/}}

\newtheorem{mydef}{Definition}
\newcommand{\norm}[1]{\lVert#1\rVert}
%\usepackage[table,xcdraw]{xcolor}
%\usepackage{authblk}         % For making the author list look prettier
%\renewcommand\Authsep{,~\,}

% Custom colors
\definecolor{deepblue}{rgb}{0,0,0.5}
\definecolor{deepred}{rgb}{0.6,0,0}
\definecolor{deepgreen}{rgb}{0,0.5,0}
\definecolor{forestgreen}{RGB}{34,139,34}
\definecolor{orangered}{RGB}{239,134,64}
\definecolor{darkblue}{rgb}{0.0,0.0,0.6}
\definecolor{gray}{rgb}{0.4,0.4,0.4}

\lstset {
  basicstyle=\ttfamily,
  frame=single
}


\setcounter{secnumdepth}{5}
\lstdefinestyle{XML} {
    language=XML,
    extendedchars=true,
    breaklines=true,
    breakatwhitespace=true,
%    emph={name,dim,interactive,overwrite},
    emphstyle=\color{red},
    basicstyle=\ttfamily,
%    columns=fullflexible,
    commentstyle=\color{gray}\upshape,
    morestring=[b]",
    morecomment=[s]{<?}{?>},
    morecomment=[s][\color{forestgreen}]{<!--}{-->},
    keywordstyle=\color{cyan},
    stringstyle=\ttfamily\color{black},
    tagstyle=\color{darkblue}\bf\ttfamily,
    morekeywords={name,type},
%    morekeywords={name,attribute,source,variables,version,type,release,x,z,y,xlabel,ylabel,how,text,param1,param2,color,label},
}
\lstset{language=python,upquote=true}

\usepackage{titlesec}
\newcommand{\sectionbreak}{\clearpage}
\setcounter{secnumdepth}{4}

%\titleformat{\paragraph}
%{\normalfont\normalsize\bfseries}{\theparagraph}{1em}{}
%\titlespacing*{\paragraph}
%{0pt}{3.25ex plus 1ex minus .2ex}{1.5ex plus .2ex}

%%%%%%%% Begin comands definition to input python code into document
\usepackage[utf8]{inputenc}

% Default fixed font does not support bold face
\DeclareFixedFont{\ttb}{T1}{txtt}{bx}{n}{9} % for bold
\DeclareFixedFont{\ttm}{T1}{txtt}{m}{n}{9}  % for normal

\usepackage{listings}

% Python style for highlighting
\newcommand\pythonstyle{\lstset{
language=Python,
basicstyle=\ttm,
otherkeywords={self, none, return},             % Add keywords here
keywordstyle=\ttb\color{deepblue},
emph={MyClass,__init__},          % Custom highlighting
emphstyle=\ttb\color{deepred},    % Custom highlighting style
stringstyle=\color{deepgreen},
frame=tb,                         % Any extra options here
showstringspaces=false            %
}}


% Python environment
\lstnewenvironment{python}[1][]
{
\pythonstyle
\lstset{#1}
}
{}

% Python for external files
\newcommand\pythonexternal[2][]{{
\pythonstyle
\lstinputlisting[#1]{#2}}}

\lstnewenvironment{xml}
{}
{}

% Python for inline
\newcommand\pythoninline[1]{{\pythonstyle\lstinline!#1!}}


\def\DRAFT{} % Uncomment this if you want to see the notes people have been adding
% Comment command for developers (Should only be used under active development)
\ifdefined\DRAFT
  \newcommand{\nameLabeler}[3]{\textcolor{#2}{[[#1: #3]]}}
\else
  \newcommand{\nameLabeler}[3]{}
\fi
% Commands for making the LaTeX a bit more uniform and cleaner
\newcommand{\TODO}[1]    {\textcolor{red}{\textit{(#1)}}}
\newcommand{\xmlAttrRequired}[1] {\textcolor{red}{\textbf{\texttt{#1}}}}
\newcommand{\xmlAttr}[1] {\textcolor{cyan}{\textbf{\texttt{#1}}}}
\newcommand{\xmlNodeRequired}[1] {\textcolor{deepblue}{\textbf{\texttt{<#1>}}}}
\newcommand{\xmlNode}[1] {\textcolor{darkblue}{\textbf{\texttt{<#1>}}}}
\newcommand{\xmlString}[1] {\textcolor{black}{\textbf{\texttt{'#1'}}}}
\newcommand{\xmlDesc}[1] {\textbf{\textit{#1}}} % Maybe a misnomer, but I am
                                                % using this to detail the data
                                                % type and necessity of an XML
                                                % node or attribute,
                                                % xmlDesc = XML description
\newcommand{\default}[1]{~\\*\textit{Default: #1}}
\newcommand{\nb} {\textcolor{deepgreen}{\textbf{~Note:}}~}


%%%%%%%% End comands definition to input python code into document

%\usepackage[dvips,light,first,bottomafter]{draftcopy}
%\draftcopyName{Sample, contains no OUO}{70}
%\draftcopyName{Draft}{300}

% The bm package provides \bm for bold math fonts.  Apparently
% \boldsymbol, which I used to always use, is now considered
% obsolete.  Also, \boldsymbol doesn't even seem to work with
% the fonts used in this particular document...
\usepackage{bm}


% Define tensors to be in bold math font.
\newcommand{\tensor}[1]{{\bm{#1}}}

% Override the formatting used by \vec.  Instead of a little arrow
% over the letter, this creates a bold character.
\renewcommand{\vec}{\bm}

% Define unit vector notation.  If you don't override the
% behavior of \vec, you probably want to use the second one.
\newcommand{\unit}[1]{\hat{\bm{#1}}}
% \newcommand{\unit}[1]{\hat{#1}}

% Use this to refer to a single component of a unit vector.
\newcommand{\scalarunit}[1]{\hat{#1}}

% \toprule, \midrule, \bottomrule for tables
\usepackage{booktabs}

% \llbracket, \rrbracket
\usepackage{stmaryrd}

\usepackage{hyperref}
\hypersetup{
    colorlinks,
    citecolor=black,
    filecolor=black,
    linkcolor=black,
    urlcolor=black
}

% Compress lists of citations like [33,34,35,36,37] to [33-37]
\usepackage{cite}

% If you want to relax some of the SAND98-0730 requirements, use the "relax"
% option. It adds spaces and boldface in the table of contents, and does not
% force the page layout sizes.
% e.g. \documentclass[relax,12pt]{SANDreport}
%
% You can also use the "strict" option, which applies even more of the
% SAND98-0730 guidelines. It gets rid of section numbers which are often
% useful; e.g. \documentclass[strict]{SANDreport}

% The INLreport class uses \flushbottom formatting by default (since
% it's intended to be two-sided document).  \flushbottom causes
% additional space to be inserted both before and after paragraphs so
% that no matter how much text is actually available, it fills up the
% page from top to bottom.  My feeling is that \raggedbottom looks much
% better, primarily because most people will view the report
% electronically and not in a two-sided printed format where some argue
% \raggedbottom looks worse.  If we really want to have the original
% behavior, we can comment out this line...
\raggedbottom
\setcounter{secnumdepth}{5} % show 5 levels of subsection
\setcounter{tocdepth}{5} % include 5 levels of subsection in table of contents

% ---------------------------------------------------------------------------- %
%
% Set the title, author, and date
%
\title{LOGOS User Manual}
%\author{%
%\begin{tabular}{c} Author 1 \\ University1 \\ Mail1 \\ \\
%Author 3 \\ University3 \\ Mail3 \end{tabular} \and
%\begin{tabular}{c} Author 2 \\ University2 \\ Mail2 \\ \\
%Author 4 \\ University4 \\ Mail4\\
%\end{tabular} }


\author{
\\Congjian Wang
\\Diego Mandelli
}

% There is a "Printed" date on the title page of a SAND report, so
% the generic \date should [WorkingDir:]generally be empty.
\date{}


% ---------------------------------------------------------------------------- %
% Set some things we need for SAND reports. These are mandatory
%
\SANDnum{INL/EXT-XX-XXXXX}
\SANDprintDate{\today}
\SANDauthor{Congjian Wang and Diego Mandelli}
\SANDreleaseType{Revision 0}
\def\component#1{\texttt{#1}}

% ---------------------------------------------------------------------------- %
\newcommand{\systemtau}{\tensor{\tau}_{\!\text{SUPG}}}

\usepackage{placeins}
\usepackage{array}

\newcolumntype{L}[1]{>{\raggedright\let\newline\\\arraybackslash\hspace{0pt}}m{#1}}
\newcolumntype{C}[1]{>{\centering\let\newline\\\arraybackslash\hspace{0pt}}m{#1}}
\newcolumntype{R}[1]{>{\raggedleft\let\newline\\\arraybackslash\hspace{0pt}}m{#1}}

% ---------------------------------------------------------------------------- %
%
% Start the document
%
\begin{document}
    \maketitle

    % ------------------------------------------------------------------------ %
    % The table of contents and list of figures and tables
    % Comment out \listoffigures and \listoftables if there are no
    % figures or tables. Make sure this starts on an odd numbered page
    %
    \cleardoublepage		% TOC needs to start on an odd page
    \tableofcontents
    %\listoffigures
    %\listoftables
    % ---------------------------------------------------------------------- %
    \SANDmain

    % ---------------------------------------------------------------------- %
    % This is where the body of the report begins; usually with an Introduction
    %
    \section{Introduction}
\label{sec:Introduction}

Industry Equipment Reliability (ER) and Asset Management (AM) Programs are essential elements that
help ensure the safe and economical operation of Nuclear Power Plants (NPPs). The effectiveness of
these programs is addressed in several industry developed and regulatory programs. However, these
programs have proven to be labor intensive and expensive. There is an opportunity to significantly
enhance the collection, analysis, and use of this information to provide more cost-effective plant
operation. LOGOS is providing computational capabilities to optimize plant resources such as
maintenance optimization (ER application) and optimal component replacement schedule (AM application)
by using state-of-the-art discrete optimization methods.

LOGOS is a software package and a RAVEN~\cite{RAVEN,RAVENtheoryMan} plugin which
contains a set of discrete optimization models that can be
employed for capital budgeting optimization problems, and LOGOS integrates economic and reliability
risk in a single analysis framework. More specifically,  Provided Systems, Structures and Components
(SSCs) health (e.g., failure rate or failure probability), OM costs, replacement costs, cost
associated to component failure and budget constraints, LOGOS provides the optimal set of projects
(e.g., SSC replacement) that maximizes profit and satisfies the provided requirements. Input data
listed above can be either deterministic or stochastic in nature, i.e., they can be point values
or probability distribution functions. In the latter case, several scenarios are generated by
sampling the provided distributions.

The developed models are based on different versions of the knapsack optimization algorithms.
Two main classes of optimization models have been initially developed: deterministic and stochastic.
Stochastic optimization models evolve deterministic models by explicitly considering data
uncertainties (associated to constraints or item cost and reward).

These models can be employed as stand-alone models or interfaced with the INL developed RAVEN code
to propagate data uncertainties and analyze the generated data (i.e., sensitivity analysis).

\subsection{Acquiring and Installing LOGOS}
LOGOS is supported on three separate computing platforms: Linux, OSX (Apple Macintosh), and Microsoft
Windows. Currently, LOGOS is downloadable from LOGOS GitLab repository:
\url{https://hpcgitlab.hpc.inl.gov/RAVEN_PLUGINS/LOGOS.git}. New users should contact LOGOS developers to
get start with LOGOS. This typically involves the following steps:

\begin{itemize}
  \item \textit{Download LOGOS}
    \\ You can download the source code of RAVEN from \url{https://hpcgitlab.hpc.inl.gov/RAVEN_PLUGINS/LOGOS.git}.
  \item \textit{Install LOGOS dependencies}
	\begin{lstlisting}[language=bash]
	path/to/LOGOS/build.sh
	\end{lstlisting}
  \item \textit{Run LOGOS}
	\begin{lstlisting}[language=bash]
	./run_tests
	\end{lstlisting}
  	Alternatively, the \texttt{logos} script
    contained in the folder ``\texttt{LOGOS}'' can be directly used:
\begin{lstlisting}[language=bash]
path/to/LOGOS/logos -i <inputFile.xml> -o <outputFile.csv>
\end{lstlisting}
	\item \textit{Used as RAVEN Plugin}, RAVEN need to be installed
		\\ Instructions are available from \url{https://github.com/idaholab/raven/wiki}.
\end{itemize}

\subsection{User Manual Formats}
In order to highlight some parts of the user manual having a particular meaning
(input structure, examples, terminal commands, etc.), specific formats have been used.
This section provides the formats with a specific meaning:
\begin{itemize}
\item \textbf{\textit{Python Coding:}}
\begin{lstlisting}[language=python]
class AClass():
  def aMethodImplementation(self):
    pass
\end{lstlisting}
\item \textbf{\textit{LOGOS XML input example:}}
\begin{lstlisting}[style=XML,morekeywords={anAttribute}]
<MainXMLBlock>
  ...
  <aXMLnode name='anObjectName' anAttribute='aValue'>
     <aSubNode>body</aSubNode>
  </aXMLnode>
  <!-- This is  commented block -->
  ...
</MainXMLBlock>
\end{lstlisting}
\item \textbf{\textit{Bash Commands:}}
\begin{lstlisting}[language=bash]
cd path/to/LOGOS/
./build.sh
cd ../../
\end{lstlisting}
\end{itemize}

\subsection{Components of LOGOS}
In LOGOS, eXtensible Markup Language (XML) format is used to create the input file. For more
information about XML, please click on the link:
\href{https://www.w3schools.com/xml/default.asp}{\textbf{XML tutorial}}.
%
\\The main input blocks are as follows:
\begin{itemize}
  \item \xmlNode{Logos}: The root node containing the
  entire input, all of
  the following blocks fit inside the \emph{Logos} block.
  %
  \item \xmlNode{Settings}: Specifies the calculation settings
  %
  \item \xmlNode{Sets}: Specifies a collection of data, possibly including
	numeric data (e.g. real or integer values) as well as symbolic data (e.g. strings)
	that is typically used to specify the valid indices for an indexed components.
	\nb numeric data provided in the \xmlNode{Sets} would be treated as strings.
  %
	\item \xmlNode{Parameters}: Specifies a collection of parameters, a parameter
	is a numerical value that is used to formulate constraints and objectives in a
	optimization model. It can denote a single value, an array of values or a multi-dimensional
	array of values.
	%
	\item \xmlNode{Uncertainties}:
	%
	%
	\item \xmlNode{ExternalConstraints}:
	%
\end{itemize}

Each of these components are explained in dedicated sections of the user manual.

\subsection{Capabilities of LOGOS}
This document provides a detailed description of the LOGOS, and the features included in LOGOS are:
\begin{itemize}
	\item Deterministic Capital Budgeting (See Section~\ref{sec:DeterministicCapitalBudgeting})
	\item Risk-informed stochastic Capital Budgeting (See Section~\ref{sec:StochasticCapitalBudgeting})
	\item Multiple Knapsack problem optimization
	\item Multi-dimensional Knapsack problem optimization
	\item Multi-choice Knapsack problem optimization
	\item Multi-choice multi-dimensional Knapsack problem optimization
	\item SSC cashflow and NPV models
	\item Plugin for the RAVEN code.
\end{itemize}

    \section{Deterministic Capital Budgeting}
\label{sec:DeterministicCapitalBudgeting}

We consider a capital budgeting problem for a nuclear generation station, with possible extension to
a larger fleet of plants. Due to limited resources, we can only select a subset from a list of
several candidate capital projects. Our goal is to maximize overall NPV associated with the
selected subset. In doing so, we must respect resource limits and capture key structural and
stochastic dependencies of the system, although in this section we start with the simpler
deterministic case, ignoring randomness.  Example projects include upgrading a steam turbine,
refurbishing or replacing a set of reactor coolant pumps, and replacing a set of feed-water heaters.

\[
\begin{array}{ll}
%%%%%%%%%%%%%% INDICES AND SET %%%%%%%%%%%%%%%%
\multicolumn{2}{l}{\mbox{\em Indexes and sets:} } \\
t \in T  & \mbox{time periods (years)} \\
i \in I  & \mbox{investment candidate projects} \\
j \in J_{i}	& \mbox{options for selecting project $i$} \\%, e.g., initiate project $i$ in year $t$ or $t+2$ and in a standard (three year) or in an expedited (two year) manner} \\
% i^{'},j^{'} \in IJ_{ij} & \mbox{piggybacking situations} \\%, i.e., option $j^{'}$ for project $i^{'}$ can be selected only if option $j$ is selected for project $i$} \\
k \in K	& \mbox{types of resources} \\%, e.g., capital funds, O\&M funds, labor-hours, time during outage} \\
\\
%%%%%%%%%%%%%% DATA %%%%%%%%%%%%%%%%
\multicolumn{2}{l}{\mbox{\em Data:}} \\
a_{ij} & \mbox{reward (revenue less financial cost) of selecting project $i$ via option $j$}  \\
b_{kt} & \mbox{available budget for a resource of type $k$ in year $t$}\\
c_{ijkt}  & \mbox{consumption of resource of type $k$ in year $t$ if project $i$ is performed via option $j$} \\
\\
%%%%%%%%%%%%%% DECISION VARS %%%%%%%%%%%%%%%%
\multicolumn{2}{l}{\mbox{\em Decision variables:}}  \\
x_{ij} & \mbox{1 if project $i$ is selected via option $j$; 0 otherwise} \hspace*{4.0in}\\
\end{array}
\]

\vst \noi {\em Formulation:}
\begin{subequations}\label{model-deter}
\begin{eqnarray}
&\dst \max_{x} &  \dst \sum_{i \in I, j \in J_{i}} a_{ij} x_{ij} \label{obj_deter} \\
& s. t.  & \sum_{j \in J_{i}} x_{ij} = 1,   i \in I \\
& & \sum_{i \in I, j \in J_{i}} c_{ijkt} x_{ij} \leq b_{kt}, k \in K, t \in T \\
& & x_{ij} \in \{0,1\}, j \in J_{i}, i \in I.
\end{eqnarray}
\end{subequations}

The decision variables, $x_{ij}$, indicate whether we choose to do project i by means j. Restated,
if $x_{ij}=1$, then we recommend doing project $i$ via option $j$, and taken together these decision
variables produce both a portfolio of selected projects and a schedule for performing those projects
over time.  The set of available options, $j \in J_i$, can explicitly include the “do-nothing” option,
and the first constraint ensures that we choose exactly one option from the available set for each
project, including the possibility of selecting the do-nothing option. Even if we select the
do-nothing option for a project, it induces an NPV, $a_{ij}$, which may be negative, representing
growing O\&M costs, losses in plant efficiency, etc. The second structural constraint ensures that
the budget of each resource $k$ is respected in each year $t$. The objective function
includes the NPV for each project-option pair, $a_{ij}$, and the correct NPV is selected by
the $0-1$ decision variable, $x_{ij}$.

% The third structural constraint
% captures piggybacking situations in which option $j^{'}$ for project $i^{'}$ (which may have cheaper
% costs) may be selected only if project-option pair $(i,j)$ is also selected.

We note that sometimes there are projects that must be done, e.g., for safety and or regulatory
reasons. This can be handled within the mathematical formulation just given, without introducing
additional constructs. The set $J_i$ typically includes a do-nothing option for each project,
but when project $i$ must be done, we simply do not include the do-nothing option. Mathematically,
an alternative is to not include an explicit do-nothing option, to replace the first structural
constraint with an inequality, and to add an additional set of must-do projects with an equality
constraint. Both options are mathematically equivalent and simply represent a choice to be made by
the analyst. In LOGOS, we use \xmlNode{regulatoryMandated} and \xmlNode{nonSelection} to
handle these conditions. \xmlNode{regulatoryMandated} is used to specify the must-do projects,
while \xmlNode{nonSelection} is used to activate the do-nothing option.

\nb If a project is listed under \xmlNode{regulatoryMandated}, do-nothing option is not allowed
for this project. In addition, Handling the do-nothing option implicitly leads to NPV be
calculated relative to that of the do-nothing option.

The objective of capital budgeting is to find the combination of the binary decisions for
every investment such that the overall profit is as large as possible. The output is a
collection of projects to be carried out, and we refer this selected collection of projects
as a project portfolio. However, as it is frequently the case for capital budgeting with
NPP applications, in practice several optional constraints, such as resources/liabilities,
dependencies/synergies, options, time windows for every investment etc., have to be
fulfilled. This leads to a various variations of the knapsack problem (described above).
In the following sub-section, we will present several different variants of about
capital budgeting problem, i.e. different variants of knapsack problem.

\subsection{Single Knapsack Problem Optimization}
\label{subsec:skp}

\subsubsection{Simple Knapsack Problem}
The simple Knapsack Problem (KP) for capital budgeting can be defined as follows:
We are given an instance of the capital budgeting problem with investment set $I$,
consisting of $I$ investments $i$ with profit $a_i$, e.g. NPV, and cost $c_i$,
and the available budget $b$. Then the objective is to select a subset of $I$ such
that the total profit of the selected investments is maximized and total cost does
not exceed $b$. Alternatively, KP can be formulated as a solution of the following
linear integer programming formulation:

\vst \noi {\em Formulation:}
\begin{subequations}\label{simpleKP}
\begin{eqnarray}
&\dst \max_{x} &  \dst \sum_{i \in I} a_{i} x_{i} \\
& s. t. & \sum_{i \in I} c_{i} x_{i} \leq b\\
& & x_{i} \in \{0,1\}, i \in I.
\end{eqnarray}
\end{subequations}

Example LOGOS input XML:
\begin{lstlisting}[style=XML]
<Logos>
  <Sets>
    <investments>
      1,2,3,4,5,6,7,8,9,10
    </investments>
  </Sets>

  <Parameters>
    <net_present_values index="investments">
      18,20,17,19,25,21,27,23,25,24
    </net_present_values>
    <costs index="investments">
      1,3,7,4,8,9,6,10,2,5
    </costs>
    <available_capitals>
      15
    </available_capitals>
  </Parameters>

  <Settings>
    <solver>cbc</solver>
    <sense>maximize</sense>
  </Settings>
</Logos>
\end{lstlisting}

When run this case, LOGOS would generate a CSV (comma separated values) file that
contains solutions for the optimization problem, i.e. values of decision variables
and maximum profit (MaxNPV is used to describe the maximum profit). The header of
this CSV file contains the indices listed under \xmlNode{investments} which are
used as indices for decision variables, and the objective variable \textbf{MaxNPV}.
The data provides the values for both decision variables and objective variable.

Example LOGOS output CSV:
\begin{lstlisting}[language=python]
1,2,3,4,5,6,7,8,9,10,MaxNPV
1.0,1.0,0.0,1.0,0.0,0.0,0.0,0.0,1.0,1.0,106.0
\end{lstlisting}

In this case, the projects \textbf{1, 2, 4, 9, 10} are selected with maximum
profit 106.0.

\subsubsection{Bounded Knapsack Problem}
In the capital budgeting problem described above it may be the case that not all
investments/projects are different from each other. In particular, in practice
there may be given a number $n_i$ of identical pumps/valves to be replaced. In this
case the number of decision variables is equal to the number of different
investments instead of the total number of investments. The constraint for
decision variable becomes:
\begin{equation}
0\leq x_i \leq n_i, i\in N
\end{equation}
The resulting problem is called the bounded knapsack problem (BKP) formally defined by

\vst \noi {\em Formulation:}
\begin{subequations}\label{boundedKP}
\begin{eqnarray}
&\dst \max_{x} &  \dst \sum_{i \in I} a_{i} x_{i} \\
& s. t. & \sum_{i \in I} c_{i} x_{i} \leq b\\
& & x_{i} \in \{0,n_i\}, i \in I.
\end{eqnarray}
\end{subequations}

Example LOGOS input XML:
\begin{lstlisting}[style=XML]
<Logos>
  <Sets>
    <investments>
      1, 2, 3, 4, 5, 6, 7, 8, 9, 10, 11, 12, 13, 14, 15, 16, 17, 18, 19, 20, 21, 22
    </investments>
  </Sets>

  <Parameters>
    <net_present_values index="investments">
      150,35,200,60,60,45,60,40,30,10,70,30,15,10,40,70,75,80,20,12,50,10
    </net_present_values>
    <costs index="investments">
      9,13,153,50,15,68,27,39,23,52,11,32,24,48,73,42,43,22,7,18,4,30
    </costs>
    <available_capitals>
      400
    </available_capitals>
  </Parameters>

  <Settings>
    <lowerBounds>
      0, 0, 0, 0, 0, 0, 0, 0, 0, 0, 0, 0, 0, 0, 0, 0, 0, 0, 0, 0, 0, 0
    </lowerBounds>
    <upperBounds>
      1,1,2,2,2,3,3,3,1,3,1,1,2,2,1,1,1,1,1,2,1,2
    </upperBounds>
    <solver>glpk</solver>
    <sense>maximize</sense>
  </Settings>
</Logos>
\end{lstlisting}

Example LOGOS output CSV:
\begin{lstlisting}[language=python]
1,2,...,21,22,MaxNPV
1.0,1.0,...,1.0,0.0,1010.0
\end{lstlisting}

\subsubsection{Multi-Dimensional Knapsack Problem: DKP}
Moving in a different direction, if we now take into account not only the cost constraint but also
the limited commitment of critical resources, including: (i) capital cost, (ii) operation
and maintenance costs, (iii) time and labor-hours during a planned outage, (iv) personnel,
installation and maintenance equipment, space, and more. Denoting the cost of every
investment by $c_{ik}$ for each resource $k$ and introduce the corresponding limited resource
$b_k$ we can formulate the capital budgeting problem as multi-dimensional knapsack problem
or D-dimensional knapsack problem formally defined by:

\vst \noi {\em Formulation:}
\begin{subequations}\label{boundedKP}
\begin{eqnarray}
&\dst \max_{x} &  \dst \sum_{i \in I} a_{i} x_{i} \\
& s. t. & \sum_{i \in I} c_{ik} x_{i} \leq b_k, k\in K\\
& & x_{i} \in \{0,1\}, i \in I.
\end{eqnarray}
\end{subequations}

Where the limited resources set is denoted by $K$, consisting of $k$ “colors” of money
within capital costs, within operation and maintenance costs, within personnel availability, etc.
Another example is that the plant has multi-year investments. Consider a DKP problem in
which the costs of each investment and the available capitals vary according to time
period $t$. By defining $c_{it}$ as the cost of investment $i$ at time period $i$,
and $b_t$ as the available capital at time period $t$, we get:

\vst \noi {\em Formulation:}
\begin{subequations}\label{boundedKP}
\begin{eqnarray}
&\dst \max_{x} &  \dst \sum_{i \in I} a_{i} x_{i} \\
& s. t. & \sum_{i \in I} c_{it} x_{i} \leq b_t, t\in T\\
& & x_{i} \in \{0,1\}, i \in I.
\end{eqnarray}
\end{subequations}

Example LOGOS input XML:
\begin{lstlisting}[style=XML]
<Logos>
  <Sets>
    <investments>
      1,2,3,4,5,6,7,8,9
    </investments>
    <time_periods>
      1,2,3,4,5
    </time_periods>
  </Sets>

  <Parameters>
    <net_present_values index="investments">
      2.315,0.824,22.459,60.589,0.667,5.173,4.003,0.582,0.122
    </net_present_values>
    <costs index="investments, time_periods">
      0.219,0.257,0.085,0.0,0.0,
      0.0,0.0,0.122,0.103,0.013,
      5.044,1.839,0.0,0.0,0.0,
      6.74,6.134,10.442,0.0,0.0,
      0.425,0.0,0.0,0.0,0.0,
      2.125,2.122,0.0,0.0,0.0,
      2.387,0.19,0.012,2.383,0.192,
      0.0,0.95,0.0,0.0,0.0,
      0.03,0.03,0.688,0.0,0.0
    </costs>
    <available_capitals index="time_periods">
      0.665,4.712,9.642,3.458,1.683
    </available_capitals>
  </Parameters>

  <Settings>
    <solver>glpk</solver>
    <sense>maximize</sense>
  </Settings>
</Logos>
\end{lstlisting}

Example LOGOS output CSV:
\begin{lstlisting}[language=python]
1,2,3,4,5,6,7,8,9,MaxNPV
1.0,1.0,0.0,0.0,1.0,0.0,0.0,1.0,0.0,4.388
\end{lstlisting}


\subsection{Multiple Knapsack Problem Optimization}
\label{subsec:mkp}


\subsection{Multiple-Choice Multi-Dimensional Knapsack Problem Optimization}
\label{subsec:mckp}


    \section*{Document Version Information}
    This document has been compiled using the following version of the plug-in git repository:
    \newline
    c5d7b2e5bfa7b6d5c9e94acbad8d0082d5a64ce6 Congjian Wang Wed, 8 May 2019 09:23:10 -0600


    % ---------------------------------------------------------------------- %
    % References
    %
    \clearpage
    % If hyperref is included, then \phantomsection is already defined.
    % If not, we need to define it.
    \providecommand*{\phantomsection}{}
    \phantomsection
    \addcontentsline{toc}{section}{References}
    \bibliographystyle{ieeetr}
    \bibliography{user_manual}


    % ---------------------------------------------------------------------- %

\end{document}
