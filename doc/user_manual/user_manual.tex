%
% This is an example LaTeX file which uses the SANDreport class file.
% It shows how a SAND report should be formatted, what sections and
% elements it should contain, and how to use the SANDreport class.
% It uses the LaTeX article class, but not the strict option.
% ItINLreport uses .eps logos and files to show how pdflatex can be used
%
% Get the latest version of the class file and more at
%    http://www.cs.sandia.gov/~rolf/SANDreport
%
% This file and the SANDreport.cls file are based on information
% contained in "Guide to Preparing {SAND} Reports", Sand98-0730, edited
% by Tamara K. Locke, and the newer "Guide to Preparing SAND Reports and
% Other Communication Products", SAND2002-2068P.
% Please send corrections and suggestions for improvements to
% Rolf Riesen, Org. 9223, MS 1110, rolf@cs.sandia.gov
%

\documentclass[pdf,12pt]{INLreport}

\newcommand{\vst}{\vspace*{0.2in}}
\newcommand{\dst}{\displaystyle}
\newcommand{\noi}{\noindent}

% pslatex is really old (1994).  It attempts to merge the times and mathptm packages.
% My opinion is that it produces a really bad looking math font.  So why are we using it?
% If you just want to change the text font, you should just \usepackage{times}.
% \usepackage{pslatex}
\usepackage{times}
\usepackage[FIGBOTCAP,normal,bf,tight]{subfigure}
\usepackage{amsmath}
\usepackage{amssymb}
\usepackage{soul}
\usepackage{pifont}
\usepackage{enumerate}
\usepackage{listings}
\usepackage{fullpage}
\usepackage{xcolor}          % Using xcolor for more robust color specification
\usepackage{ifthen}          % For simple checking in newcommand blocks
\usepackage{textcomp}
\usepackage{mathtools}
\usepackage{relsize}
\usepackage{lscape}
\usepackage[toc,page]{appendix}

\graphicspath{{./figures/}}

\newtheorem{mydef}{Definition}
\newcommand{\norm}[1]{\lVert#1\rVert}
%\usepackage[table,xcdraw]{xcolor}
%\usepackage{authblk}         % For making the author list look prettier
%\renewcommand\Authsep{,~\,}

% Custom colors
\definecolor{deepblue}{rgb}{0,0,0.5}
\definecolor{deepred}{rgb}{0.6,0,0}
\definecolor{deepgreen}{rgb}{0,0.5,0}
\definecolor{forestgreen}{RGB}{34,139,34}
\definecolor{orangered}{RGB}{239,134,64}
\definecolor{darkblue}{rgb}{0.0,0.0,0.6}
\definecolor{gray}{rgb}{0.4,0.4,0.4}

\lstset {
  basicstyle=\ttfamily,
  frame=single
}


\setcounter{secnumdepth}{5}
\lstdefinestyle{XML} {
    language=XML,
    extendedchars=true,
    breaklines=true,
    breakatwhitespace=true,
%    emph={name,dim,interactive,overwrite},
    emphstyle=\color{red},
    basicstyle=\ttfamily,
%    columns=fullflexible,
    commentstyle=\color{gray}\upshape,
    morestring=[b]",
    morecomment=[s]{<?}{?>},
    morecomment=[s][\color{forestgreen}]{<!--}{-->},
    keywordstyle=\color{cyan},
    stringstyle=\ttfamily\color{black},
    tagstyle=\color{darkblue}\bf\ttfamily,
    morekeywords={name,type},
%    morekeywords={name,attribute,source,variables,version,type,release,x,z,y,xlabel,ylabel,how,text,param1,param2,color,label},
}
\lstset{language=python,upquote=true}

\usepackage{titlesec}
\newcommand{\sectionbreak}{\clearpage}
\setcounter{secnumdepth}{4}

%\titleformat{\paragraph}
%{\normalfont\normalsize\bfseries}{\theparagraph}{1em}{}
%\titlespacing*{\paragraph}
%{0pt}{3.25ex plus 1ex minus .2ex}{1.5ex plus .2ex}

%%%%%%%% Begin comands definition to input python code into document
\usepackage[utf8]{inputenc}

% Default fixed font does not support bold face
\DeclareFixedFont{\ttb}{T1}{txtt}{bx}{n}{9} % for bold
\DeclareFixedFont{\ttm}{T1}{txtt}{m}{n}{9}  % for normal

\usepackage{listings}

% Python style for highlighting
\newcommand\pythonstyle{\lstset{
language=Python,
basicstyle=\ttm,
otherkeywords={self, none, return},             % Add keywords here
keywordstyle=\ttb\color{deepblue},
emph={MyClass,__init__},          % Custom highlighting
emphstyle=\ttb\color{deepred},    % Custom highlighting style
stringstyle=\color{deepgreen},
frame=tb,                         % Any extra options here
showstringspaces=false            %
}}


% Python environment
\lstnewenvironment{python}[1][]
{
\pythonstyle
\lstset{#1}
}
{}

% Python for external files
\newcommand\pythonexternal[2][]{{
\pythonstyle
\lstinputlisting[#1]{#2}}}

\lstnewenvironment{xml}
{}
{}

% Python for inline
\newcommand\pythoninline[1]{{\pythonstyle\lstinline!#1!}}


\def\DRAFT{} % Uncomment this if you want to see the notes people have been adding
% Comment command for developers (Should only be used under active development)
\ifdefined\DRAFT
  \newcommand{\nameLabeler}[3]{\textcolor{#2}{[[#1: #3]]}}
\else
  \newcommand{\nameLabeler}[3]{}
\fi
% Commands for making the LaTeX a bit more uniform and cleaner
\newcommand{\TODO}[1]    {\textcolor{red}{\textit{(#1)}}}
\newcommand{\xmlAttrRequired}[1] {\textcolor{red}{\textbf{\texttt{#1}}}}
\newcommand{\xmlAttr}[1] {\textcolor{cyan}{\textbf{\texttt{#1}}}}
\newcommand{\xmlNodeRequired}[1] {\textcolor{deepblue}{\textbf{\texttt{<#1>}}}}
\newcommand{\xmlNode}[1] {\textcolor{darkblue}{\textbf{\texttt{<#1>}}}}
\newcommand{\xmlString}[1] {\textcolor{black}{\textbf{\texttt{'#1'}}}}
\newcommand{\xmlDesc}[1] {\textbf{\textit{#1}}} % Maybe a misnomer, but I am
                                                % using this to detail the data
                                                % type and necessity of an XML
                                                % node or attribute,
                                                % xmlDesc = XML description
\newcommand{\default}[1]{~\\*\textit{Default: #1}}
\newcommand{\nb} {\textcolor{deepgreen}{\textbf{~Note:}}~}


%%%%%%%% End comands definition to input python code into document

%\usepackage[dvips,light,first,bottomafter]{draftcopy}
%\draftcopyName{Sample, contains no OUO}{70}
%\draftcopyName{Draft}{300}

% The bm package provides \bm for bold math fonts.  Apparently
% \boldsymbol, which I used to always use, is now considered
% obsolete.  Also, \boldsymbol doesn't even seem to work with
% the fonts used in this particular document...
\usepackage{bm}


% Define tensors to be in bold math font.
\newcommand{\tensor}[1]{{\bm{#1}}}

% Override the formatting used by \vec.  Instead of a little arrow
% over the letter, this creates a bold character.
\renewcommand{\vec}{\bm}

% Define unit vector notation.  If you don't override the
% behavior of \vec, you probably want to use the second one.
\newcommand{\unit}[1]{\hat{\bm{#1}}}
% \newcommand{\unit}[1]{\hat{#1}}

% Use this to refer to a single component of a unit vector.
\newcommand{\scalarunit}[1]{\hat{#1}}

% \toprule, \midrule, \bottomrule for tables
\usepackage{booktabs}

% \llbracket, \rrbracket
\usepackage{stmaryrd}

\usepackage{hyperref}
\hypersetup{
    colorlinks,
    citecolor=black,
    filecolor=black,
    linkcolor=black,
    urlcolor=black
}

% Compress lists of citations like [33,34,35,36,37] to [33-37]
\usepackage{cite}

% If you want to relax some of the SAND98-0730 requirements, use the "relax"
% option. It adds spaces and boldface in the table of contents, and does not
% force the page layout sizes.
% e.g. \documentclass[relax,12pt]{SANDreport}
%
% You can also use the "strict" option, which applies even more of the
% SAND98-0730 guidelines. It gets rid of section numbers which are often
% useful; e.g. \documentclass[strict]{SANDreport}

% The INLreport class uses \flushbottom formatting by default (since
% it's intended to be two-sided document).  \flushbottom causes
% additional space to be inserted both before and after paragraphs so
% that no matter how much text is actually available, it fills up the
% page from top to bottom.  My feeling is that \raggedbottom looks much
% better, primarily because most people will view the report
% electronically and not in a two-sided printed format where some argue
% \raggedbottom looks worse.  If we really want to have the original
% behavior, we can comment out this line...
\raggedbottom
\setcounter{secnumdepth}{5} % show 5 levels of subsection
\setcounter{tocdepth}{5} % include 5 levels of subsection in table of contents

% ---------------------------------------------------------------------------- %
%
% Set the title, author, and date
%
\title{LOGOS User Manual}
%\author{%
%\begin{tabular}{c} Author 1 \\ University1 \\ Mail1 \\ \\
%Author 3 \\ University3 \\ Mail3 \end{tabular} \and
%\begin{tabular}{c} Author 2 \\ University2 \\ Mail2 \\ \\
%Author 4 \\ University4 \\ Mail4\\
%\end{tabular} }


\author{
\\Congjian Wang
\\Diego Mandelli
}

% There is a "Printed" date on the title page of a SAND report, so
% the generic \date should [WorkingDir:]generally be empty.
\date{}


% ---------------------------------------------------------------------------- %
% Set some things we need for SAND reports. These are mandatory
%
\SANDnum{INL/EXT-XX-XXXXX}
\SANDprintDate{\today}
\SANDauthor{Congjian Wang and Diego Mandelli}
\SANDreleaseType{Revision 0}
\def\component#1{\texttt{#1}}

% ---------------------------------------------------------------------------- %
\newcommand{\systemtau}{\tensor{\tau}_{\!\text{SUPG}}}

\usepackage{placeins}
\usepackage{array}

\newcolumntype{L}[1]{>{\raggedright\let\newline\\\arraybackslash\hspace{0pt}}m{#1}}
\newcolumntype{C}[1]{>{\centering\let\newline\\\arraybackslash\hspace{0pt}}m{#1}}
\newcolumntype{R}[1]{>{\raggedleft\let\newline\\\arraybackslash\hspace{0pt}}m{#1}}

% ---------------------------------------------------------------------------- %
%
% Start the document
%
\begin{document}
    \maketitle

    % ------------------------------------------------------------------------ %
    % The table of contents and list of figures and tables
    % Comment out \listoffigures and \listoftables if there are no
    % figures or tables. Make sure this starts on an odd numbered page
    %
    \cleardoublepage		% TOC needs to start on an odd page
    \tableofcontents
    %\listoffigures
    %\listoftables
    % ---------------------------------------------------------------------- %
    \SANDmain

    % ---------------------------------------------------------------------- %
    % This is where the body of the report begins; usually with an Introduction
    %
    \section{Introduction}
\label{sec:Introduction}

Industry Equipment Reliability (ER) and Asset Management (AM) Programs are essential elements that
help ensure the safe and economical operation of Nuclear Power Plants (NPPs). The effectiveness of
these programs is addressed in several industry developed and regulatory programs. However, these
programs have proven to be labor intensive and expensive. There is an opportunity to significantly
enhance the collection, analysis, and use of this information to provide more cost-effective plant
operation. LOGOS is providing computational capabilities to optimize plant resources such as
maintenance optimization (ER application) and optimal component replacement schedule (AM application)
by using state-of-the-art discrete optimization methods.

LOGOS is a software package and a RAVEN~\cite{RAVEN,RAVENtheoryMan} plugin which
contains a set of discrete optimization models that can be
employed for capital budgeting optimization problems, and LOGOS integrates economic and reliability
risk in a single analysis framework. More specifically,  Provided Systems, Structures and Components
(SSCs) health (e.g., failure rate or failure probability), OM costs, replacement costs, cost
associated to component failure and budget constraints, LOGOS provides the optimal set of projects
(e.g., SSC replacement) that maximizes profit and satisfies the provided requirements. Input data
listed above can be either deterministic or stochastic in nature, i.e., they can be point values
or probability distribution functions. In the latter case, several scenarios are generated by
sampling the provided distributions.

The developed models are based on different versions of the knapsack optimization algorithms.
Two main classes of optimization models have been initially developed: deterministic and stochastic.
Stochastic optimization models evolve deterministic models by explicitly considering data
uncertainties (associated to constraints or item cost and reward).

These models can be employed as stand-alone models or interfaced with the INL developed RAVEN code
to propagate data uncertainties and analyze the generated data (i.e., sensitivity analysis).

\subsection{Acquiring and Installing LOGOS}
LOGOS is supported on three separate computing platforms: Linux, OSX (Apple Macintosh), and Microsoft
Windows. Currently, LOGOS is downloadable from LOGOS GitLab repository:
\url{https://hpcgitlab.hpc.inl.gov/RAVEN_PLUGINS/LOGOS.git}. New users should contact LOGOS developers to
get start with LOGOS. This typically involves the following steps:

\begin{itemize}
  \item \textit{Download LOGOS}
    \\ You can download the source code of RAVEN from \url{https://hpcgitlab.hpc.inl.gov/RAVEN_PLUGINS/LOGOS.git}.
  \item \textit{Install LOGOS dependencies}
	\begin{lstlisting}[language=bash]
	path/to/LOGOS/build.sh
	\end{lstlisting}
  \item \textit{Run LOGOS}
	\begin{lstlisting}[language=bash]
	./run_tests
	\end{lstlisting}
  	Alternatively, the \texttt{logos} script
    contained in the folder ``\texttt{LOGOS}'' can be directly used:
\begin{lstlisting}[language=bash]
path/to/LOGOS/logos -i <inputFile.xml> -o <outputFile.csv>
\end{lstlisting}
	\item \textit{Used as RAVEN Plugin}, RAVEN need to be installed
		\\ Instructions are available from \url{https://github.com/idaholab/raven/wiki}.
\end{itemize}

\subsection{User Manual Formats}
In order to highlight some parts of the user manual having a particular meaning
(input structure, examples, terminal commands, etc.), specific formats have been used.
This section provides the formats with a specific meaning:
\begin{itemize}
\item \textbf{\textit{Python Coding:}}
\begin{lstlisting}[language=python]
class AClass():
  def aMethodImplementation(self):
    pass
\end{lstlisting}
\item \textbf{\textit{LOGOS XML input example:}}
\begin{lstlisting}[style=XML,morekeywords={anAttribute}]
<MainXMLBlock>
  ...
  <aXMLnode name='anObjectName' anAttribute='aValue'>
     <aSubNode>body</aSubNode>
  </aXMLnode>
  <!-- This is  commented block -->
  ...
</MainXMLBlock>
\end{lstlisting}
\item \textbf{\textit{Bash Commands:}}
\begin{lstlisting}[language=bash]
cd path/to/LOGOS/
./build.sh
cd ../../
\end{lstlisting}
\end{itemize}

\subsection{Components of LOGOS}
In LOGOS, eXtensible Markup Language (XML) format is used to create the input file. For more
information about XML, please click on the link:
\href{https://www.w3schools.com/xml/default.asp}{\textbf{XML tutorial}}.
%
\\The main input blocks are as follows:
\begin{itemize}
  \item \xmlNode{Logos}: The root node containing the
  entire input, all of
  the following blocks fit inside the \emph{Logos} block.
  %
  \item \xmlNode{Settings}: Specifies the calculation settings
  %
  \item \xmlNode{Sets}: Specifies a collection of data, possibly including
	numeric data (e.g. real or integer values) as well as symbolic data (e.g. strings)
	that is typically used to specify the valid indices for an indexed components.
	\nb numeric data provided in the \xmlNode{Sets} would be treated as strings.
  %
	\item \xmlNode{Parameters}: Specifies a collection of parameters, a parameter
	is a numerical value that is used to formulate constraints and objectives in a
	optimization model. It can denote a single value, an array of values or a multi-dimensional
	array of values.
	%
	\item \xmlNode{Uncertainties}:
	%
	%
	\item \xmlNode{ExternalConstraints}:
	%
\end{itemize}

Each of these components are explained in dedicated sections of the user manual.

\subsection{Capabilities of LOGOS}
This document provides a detailed description of the LOGOS, and the features included in LOGOS are:
\begin{itemize}
	\item Deterministic Capital Budgeting (See Section~\ref{sec:DeterministicCapitalBudgeting})
	\item Risk-informed stochastic Capital Budgeting (See Section~\ref{sec:StochasticCapitalBudgeting})
	\item Multiple Knapsack problem optimization
	\item Multi-dimensional Knapsack problem optimization
	\item Multi-choice Knapsack problem optimization
	\item Multi-choice multi-dimensional Knapsack problem optimization
	\item SSC cashflow and NPV models
	\item Plugin for the RAVEN code.
\end{itemize}

    \section{Overview of Modeling Components}
\label{sec:ModelingComponents}

We consider a capital budgeting problem for a nuclear generation station, with possible
extension to a larger fleet of plants. Due to limited resources, we can only select a
subset from a number of candidate investment projects. Our goal is to maximize overall net
present value (NPV), or a variant of this objective when we incorporate uncertainty in
project cost–and project revenue–streams. In doing so, we must respect resource limits
and capture key structural and stochastic dependencies of the system. Example projects
include upgrading a steam turbine, refurbishing or replacing a set of reactor coolant pumps,
and replacing a set of feed-water heaters. Selecting an individual project has multiple
facets and implications.

\begin{itemize}
  \item \textbf{Rewards or Net Present Values}: Selecting a project can improve revenue, e.g.,
  upgrading a steam turbine may lead to an uprate in plant capacity resulting in larger
  revenue from selling power. Replacing a key system component can improve reliability,
  increasing revenue due to a reduction in forced outages and reducing operations and
  maintenance (O\&M) costs. Choosing to perform minimum maintenance versus refurbishing
  a component versus replacing and improving a system can produce “reward” streams over
  years which can be negative or positive depending on the selection. Parameter
  \xmlNode{net\_present\_values} is used to specify the rewards (see Section~\ref{subsec:Parameters}).

  \item \textbf{Resources and Liabilities}: Critical resources, including: (i) capital costs,
  (ii) O\&M costs, (iii) time and labor-hours during a planned outage, (iv) personnel,
  installation \& maintenance equipment, space, and more. Within these categories, resources
  can be placed in further subcategories, each with its own budget, due to a plant’s organizational
  structure so that there are multiple “colors” of money within capital costs, within O\&M costs,
  within personnel availability, etc. Set \xmlNode{resources} and parameter
  \xmlNode{available\_capitals} are used to specify the resources and
  liabilities (see Section~\ref{subsec:Sets} and Section~\ref{subsec:Parameters}).

  \item \textbf{Costs}: Selecting a project in year $t$ induces multiple
  cost streams in year $t$ and in subsequent years, where we interpret “cost” broadly to
  include commitment of critical resources. Parameter \xmlNode{costs} is used to specify
  the costs (see Section~\ref{subsec:Parameters}).

  \item \textbf{Time Periods}: Multiple capital projects can compete for time that would
  limit project selection. Set \xmlNode{time\_periods} is used to provide indices for
  \textbf{costs} and \textbf{available capitals} (see Section~\ref{subsec:Sets}).

  \item \textbf{Options}: The goal of selecting a project is typically to improve or maintain
  a particular function that the plant performs, and there may be multiple ways to carry out
  the task. A project may be performed over a three-year period, say, years `$t$, $t+1$, $t+2$', or the
  start of the project could instead be two years hence with project implementation over
  years `$t+2$, $t+3$, $t+4$'. Alternatively, at increased cost and increased benefit, it may be
  possible to complete the project in two years, `$t$, $t+1$' or `$t+2$, $t+3$'. When selecting a project
  to uprate plant capacity, we may have two options that increase capacity by 3\% or 6\%.
  In all these cases, we can perform the project in at most one way, from a collection of
  multiple options. We represent this by cloning a “project” into multiple project-option pairs,
  and adding a constraint saying that we can select at most one from this set of options.
  Set \xmlNode{options} is used to provided indices for these multiple project-option pairs
  (see Section~\ref{subsec:Sets}).

  \item \textbf{Capitals}: If we consider a maintenance for multi-units NPP in parallel,
  i.e. it has to be decided whether to accept a particular replacement and in the positive
  case in which unit to conduct the corresponding replacement. In this case, set \xmlNode{capitals}
  is used to provided indices for these units (see Section~\ref{subsec:Sets}).

  \item \textbf{Available Capitals}: available budgets for resources/units. Parameter
  \xmlNode{available\_capitals} is used to specify the available capitals for different
  resources/captials at different $t$ (see Section~\ref{subsec:Parameters}).

  \item \textbf{Non-Selection}: Not selecting a project also has implications, inducing growth
  in O\&M costs in future years, a decrease in plant production, an increase in forced outages,
  and even risking a premature end to plant life. Thus, not selecting a project can be seen as
  one more “option” as to how a larger project is executed, expanding the list just discussed.
  Selection is of the “do nothing” option is reflected in both liability streams and reward
  streams. This can be activated via setting \xmlNode{nonSelection} to \xmlString{True}
  (see Section~\ref{subsec:Settings}).

  \item \textbf{Uncertainty}: One limitation of traditional optimization models for capital
  budgeting is that they do not account for uncertainty in reward and cost streams associated
  with individual projects, they do not account for uncertainty in resource availability in
  future years. Projects can incur cost over-runs, especially when projects are large, performed
  infrequently, and when there is uncertainty regarding technical viability, external contractors,
  and/or suppliers of requisite parts and materials. Occasionally, projects are performed ahead
  of schedule and with cost savings. Planned budgets for capital improvements can be cut and key
  personnel may be lost. Or, there may be surprise windfalls in budgets for maintenance activities
  due to decreased costs for “unplanned” maintenance. XML node \xmlNode{Uncertainties} is used
  to specify the uncertainties (see Section~\ref{subsec:Uncertainties}).

  %\item \textbf{Synergies}: Selecting a project may require replacing a structure, system, or
  %component (SSC) during a planned outage of the plant. Depending on the physical location of
  %an SSC in the plant and its relationship to other components, selecting one project may
  %reduce the cost of selecting another project (e.g., time or know-how required to implement
  %the project) if they are selected at the same time or close in proximity. For example, if
  %a plant has two units, selecting a project for one unit in a spring outage (e.g., replacement
  %of a condensate cooler and a set of feed-water heaters) may be followed by the same activity
  %in the fall outage in the second unit, at reduced cost.

  %\item \textbf{Planned Outage}: Nuclear power plants have planned outages at regular
  %intervals (e.g., every 18 months) often in the fall and spring to be well-prepared for winter
  %and summer peaks in load. While refueling only takes a fraction of a two-month (say) period
  %without power production, maintenance projects may be deferred until an outage. Moreover,
  %an outage can provide the only possible time period in which to carry out certain types of
  %projects. Because of lost revenue, an operator seeks to limit downtime. As a result, this
  %provides a special type of resource constraint limiting project selection due to multiple
  %projects competing for time, space, personnel, and equipment during an outage.

\end{itemize}

LOGOS consists of a collection of modeling entities/components that define different
aspects of the model, including \xmlNode{Sets}, \xmlNode{Parameters},
\xmlNode{Uncertainties} and \xmlNode{ExternalConstraints}. In addition, \xmlNode{Setting}
block specifies how the overall computation should run.

%
\subsection{Sets}
\label{subsec:Sets}

This sub-section contains the information regarding the XML nodes used to define the
\xmlNode{Sets} of the optimization model that is being performed through LOGOS.
\xmlNode{Sets} specifies a collection of data, possibly including
numeric data (e.g. real or integer values) as well as symbolic data (e.g. strings)
that is typically used to specify the valid indices for an indexed components.
\nb numeric data provided in the \xmlNode{Sets} would be treated as strings.
\xmlNode{Sets} accepts the following additional sub-nodes:
\begin{itemize}
  \item \xmlNode{investments}, \xmlDesc{comma/space-separated string, required}, specifies
  the valid indices for investment projects.
  \item \xmlNode{capitals}, \xmlDesc{comma/space-separated string, optional},
  specifies the indices for NPP units.
  \item \xmlNode{time\_periods}, \xmlDesc{comma/space-separated string, optional},
  specifies the indices for time.
  \item \xmlNode{resources}, \xmlDesc{comma/space-separated string, optional},
  specifies indices for the resources and liabilities.
  \item \xmlNode{options}, \xmlDesc{semi-colon separated list of strings, optional},
  specifies the indices for multiple project-option pairs.
  This sub-node accepts the following attribute:
  \begin{itemize}
    \item \xmlAttr{index}, \xmlDesc{string, required}, specifies the index dependence.
    Valid index is \xmlString{investments}.
  \end{itemize}
\end{itemize}

Example XML:
\begin{lstlisting}[style=XML]
<Sets>
  <investments>
      HPFeedwaterHeaterUpgrade
      PresurizerReplacement
      ...
      ReplaceMoistureSeparatorReheater
  </investments>
  <time_periods>year1 year2 year3 year4 year5</time_periods>
  <resources>CapitalFunds OandMFunds</resources>
  <options index='investments'>
    PlanA PlanB DoNothing;
    PlanA PlanB PlanC;
    ...
    PlanA PlanB PlanC DoNothing
  </options>
</Sets>
\end{lstlisting}


%
\subsection{Parameters}
\label{subsec:Parameters}
This sub-section contains the information regarding the XML nodes used to define the
\xmlNode{Parameters} of the optimization model that is being performed through LOGOS:
\begin{itemize}
  \item \xmlNode{net\_present\_values}, \xmlDesc{comma/space-separated string, required},
  specifies the NPVs for capital projects or project-option pairs. This node accepts the
  following optional attribute:
  \begin{itemize}
    \item \xmlAttr{index}, \xmlDesc{comma-separated string, optional},
    specifies the indices of this parameter, keywords should be predefined in \xmlNode{Sets}.
    Valid keywords are \xmlString{investments} and \xmlString{options}.
    \default{investments}
  \end{itemize}
  \item \xmlNode{costs}, \xmlDesc{comma/space-separated string, required},
  specifies the costs for capital projects or project-option pairs. This node accepts the
  following optional attribute:
  \begin{itemize}
    \item \xmlAttr{index}, \xmlDesc{comma-separated string, optional},
    specifies the indices of this parameter, keywords should be predefined in \xmlNode{Sets}.
    Valid keywords are \xmlString{investments}, \xmlString{investments, time\_periods},
    \xmlString{options}, \xmlString{options, resources}, \xmlString{options, time\_periods},
    and \xmlString{options, resources, time\_periods}.
    \default{'investments'}
  \end{itemize}
  \item \xmlNode{available\_capitals}, \xmlDesc{comma/space-separated string, required},
  specifies the available capitals for capital projects or project-option pairs.
  This node accepts the following optional attribute:
  \begin{itemize}
    \item \xmlAttr{index}, \xmlDesc{comma-separated string, optional},
    specifies the indices of this parameter, keywords should be predefined in \xmlNode{Sets}.
    Valid keywords are \xmlString{resources}, \xmlString{time\_periods}, \xmlString{capitals},
    \xmlString{resources, time\_periods}, and \xmlString{capitals, time\_periods}
    \default{None}
  \end{itemize}
\end{itemize}

Example XML:
\begin{lstlisting}[style=XML]
<Parameters>
  <net_present_values index='options'>
    27.98 27.17 0.
    -10.07 -9.78 -9.22
    ...
    8.26 7.56 7.34 0.
  </net_present_values>
  <costs index='options,resources,time_periods'>
    12.99 1.3 0 0 0
    ...
    0.01 0 0 0 0
  </costs>
  <available_capitals index="resources,time_periods">
    22.6 36.7 20.6 23.6 22.7
    0.08 0.17 0.05 0.15 0.14
  </available_capitals>
</Parameters>
\end{lstlisting}


%
\subsection{Uncertainties}
\label{subsec:Uncertainties}
This sub-section contains the information regarding the XML nodes used to define the
\xmlNode{Uncertainties} of the optimization model that is being performed through LOGOS:
\begin{itemize}
  \item \xmlNode{available\_capitals}, \xmlDesc{optional}, specifies the scenarios
  associated with available capitals. This node accepts the attribute \xmlAttr{index} which
  should be consistent with \xmlNode{available\_capitals} defined in \xmlNode{Parameters}.
  This node accepts the following sub-nodes:
  \begin{itemize}
    \item \xmlNode{totalScenarios}, \xmlDesc{integer, required}, specifies the total
    number of scenarios for this parameter.
    \item \xmlNode{probabilities}, \xmlDesc{comma/space-separated float, required},
    specifies the probability for each scenario. The length should be equal total number of
    scenarios.
    \item \xmlNode{scenarios}, \xmlDesc{comma/space-separated float, required},
    specifies all scenarios for this parameter. The length should be equal total number
    of scenarios multiply the length of this parameter defined in \xmlNode{Parameters}.
  \end{itemize}

  \item \xmlNode{net\_present\_values}, \xmlDesc{optional}, specifies the scenarios
  associated with net\_present\_values. This node accepts the attribute \xmlAttr{index} which
  should be consistent with \xmlNode{net\_present\_values} defined in \xmlNode{Parameters}.
  \begin{itemize}
    \item \xmlNode{totalScenarios}, \xmlDesc{integer, required}, specifies the total
    number of scenarios for this parameter.
    \item \xmlNode{probabilities}, \xmlDesc{comma/space-separated float, required},
    specifies the probability for each scenario. The length should be equal total number of
    scenarios.
    \item \xmlNode{scenarios}, \xmlDesc{comma/space-separated float, required},
    specifies all scenarios for this parameter. The length should be equal total number
    of scenarios multiply the length of this parameter defined in \xmlNode{Parameters}.
  \end{itemize}
\end{itemize}

The overall number of scenarios is the total number of scenarios of \xmlNode{available\_capitals}
multiply the total number of scenarios of \xmlNode{net\_present\_values}.

Example XML:
\begin{lstlisting}[style=XML]
<Uncertainties>
  <available_capitals index="resources,time_periods">
    <totalScenarios>10</totalScenarios>
    <probabilities>
      0.5, 0.5
    </probabilities>
    <scenarios>
      20.0 34.0 17.0 20.0 18.0 0.08 0.17 0.05 0.15 0.14
      23.0 38.0 22.0 25.0 24.0 0.08 0.17 0.05 0.15 0.14
    </scenarios>
  </available_capitals>
  <net_present_values index='options'>
    <totalScenarios>9</totalScenarios>
    <probabilities>
      0.3 0.8
    </probabilities>
    <scenarios>
      13.3129 12.0228 0.0 -10.07
      ...
    </scenarios>
  </net_present_values>
</Uncertainties>
\end{lstlisting}


%
\subsection{External Constraints}
\label{subsec:ExternalConstraints}

This sub-section contains the information regarding the XML nodes used to define the
\xmlNode{ExternalConstraints} of the optimization model that is being performed through LOGOS.
This node accepts the following sub-node(s):
\begin{itemize}
  \item \xmlNode{constraint}, \xmlDesc{string, required}, specifies the external python
  module file name (file name with its absolute or relative path). This external python
  module contains the user defined additional constraint.
  \nb If a relative path is specified, the code first checks relative to the working directory,
  then it checks with respect to the location of input file. The working directory can be
  specified in \xmlNode{Settings} (see Section~\ref{subsec:Settings}). In addition, the extension
  `.py' is optional for the module file name that inputted in this node.
  on the optimization problem. This sub-node also requires the following attribute:
  \begin{itemize}
    \item \xmlAttr{name}, \xmlDesc{string, required}, specifies the name of constraint that will
    be added to optimization problem.
  \end{itemize}
\end{itemize}

Example XML:
\begin{lstlisting}[style=XML]
<ExternalConstraints>
  <constraint name="con_I">externalConst</constraint>
  <constraint name="con_II">externalConstII.py</constraint>
</ExternalConstraints>
\end{lstlisting}

These constraints are Python modules, with a format that is automatically interpretable by
LOGOS. For example, users can define their own constraint and the code will be embedded
and use the constraint as though it were an active part of the code itself.
In the following, an example of a user-defined external constraint is reported.

Example Python Function:
\begin{lstlisting}[language=python]
# External constraint function
import numpy as np
import pyomo.environ as pyomo

def initialize():
  """
    Optional Method
    Optimization model parameters values can be updated/modified
    without directly accessing the optimization model.
    Value(s) will be updated in-place.
    @ In, None
    @ Out, updateDict, dict, {paramName:paramInfoDict},  where
      paramInfoDict contains {Indices:Values}
      Indices are parameter indices (either strings or tuples of
      strings, depending on whether there is one or
      more than one dimension). Values are the new values being
      assigned to the parameter at the given indices.
  """
  updateDict = {'available_capitals':{'None':16},
                'costs':{'1':1,'2':3,'3':7,'4':4,'5':8,
                         '6':9,'7':6,'8':10,'9':2,'10':5}
               }
  return updateDict

def constraint(var, sets, params):
  """
    Required Method
    External constraint provided by users that will be added to
    optimization problem
    @ In, sets, dict, all "Sets" provided in the Logos input
      file will be stored and available in this dictionary,
      i.e. {setName: setObject}
    @ In, params, dict, all "Parameters" provided in the
      Logos input file will be stored and
      available in this dictionary, i.e. {paramName: paramObject}
    @ In, var, object, the internally used decision variable,
      the dimensions/indices of this variable depend the type of
      optimization problems (i.e. "<problem_type>" from Logos
      input file). Currently, we will accept the following
      problem types:

      1. "singleknapsack": in this case, "var" will be var[:],
         where the index will be the element from
         xml node of "investment" in Logos input file.

      2. "multipleknapsack": in this case, "var" will be var[:,:],
         where the indices are the combinations element from set
         "investment" and element from set "capitals" in Logos
         input file

      3. "mckp": in this case, "var" will be var[:,:], where the
         indices are the combinations element from set
         "investment" and element from set "options" in Logos
         input file

      (Note that all element that is used as index will be
      converted to string even if a number is provided in
      the Logos input file).

    @ Out, constraint, tuple, either (constraintRule,)
      or (constraintRule, indices)

    (Note that any modifications in provided sets and params
    will only have impact on this local module,
    i.e. the external constraint. In other words, the Sets
    and Params used in the internal constraints and
    objective will be kept unchanged!)
  """
  # All sets and parameters can be retrieved from dictionary
  # "sets" and "params" investments = sets['investments']

  def constraintRule(self, i):
    """
      Expression for user provided external constraint
      @ In, self, object, required to present, but not used
      @ In, i, str, element for the index set
      @ Out, constraintRule, function expression, expression
        to define user provided constraint

      Note that: Constraints can be indexed by lists or sets.
      When the return of function "constraint" contains
      lists or sets except the "constraintRule", the elements
      are iteratively passed to the rule function. If there
      is more than one, then the cross product is sent.
      For example, this constraint could be interpreted as
      placing limit on "ith" decision variable "var".
      A list of constraints for all "ith" decision variable
      "var" will be added to the optimization model
    """
    return var[i] <= 1

  # A tuple is required for the return, the first element
  # should be always the "constraintRule",
  # while the rest of elements are the lists or sets
  # if the user wants to construct the constraints
  # iteratively (See the docstring in "constraintRule"),
  # otherwise, keep it empty
  return (constraintRule, investments)
\end{lstlisting}

%
\subsection{Settings: Options for Optimization}
\label{subsec:Settings}

This sub-section contains the information regarding the XML nodes used to define the
\xmlNode{Settings} of the optimization model that is being performed through LOGOS:
\begin{itemize}
  \item \xmlNode{problem\_type}, \xmlDesc{string, required parameter}, specifies the type of
  optimization problem. Available types including \xmlString{SingleKnapsack},
  \xmlString{MultipleKnapsack} and \xmlString{MCKP}.
  \item \xmlNode{solver}, \xmlDesc{string, optional parameter}, available solvers including:
  \xmlNode{cbc} from \url{https://github.com/coin-or/Cbc.git} and \xmlNode{glpk} from
  \url{https://www.gnu.org/software/glpk/}
  \item \xmlNode{sense}, \xmlDesc{string, optional parameter}, specifies the \xmlString{minimize}
  or \xmlString{maximize} for minimization or maximization, respectively.
  \default{minimize}
  \item \xmlNode{regulatoryMandated}, \xmlDesc{comma/space-separated string, optional parameter},
  specifies the regulatory mandated or must do projects.
  \item \xmlNode{nonSelection}, \xmlDesc{boolean, optional parameter}, indicates whether the
  options of investments including \textit{DoNothing} option or not.
  \default{False}
  \item \xmlNode{lowerBounds}, \xmlDesc{comma/space-separated integers, optional parameter}, specifies the lower bounds
  for decision variables.
  \item \xmlNode {upperBounds}, \xmlDesc{comma/space-separated integers, optional parameter}, specifies the upper bounds
  for decision variables.
  \item \xmlNode{consistentConstraintI}, \xmlDesc{string, optional parameter}, indicates whether
  this constraint is enabled or not.
  \default{True}
  \item \xmlNode{consistentConstraintII}, \xmlDesc{string, optional parameter}, indicates whether
  this constraint is enabled or not.
  \default{False}
  \item \xmlNode{solverOptions}, \xmlDesc{optional parameter}. This node will accept
  different options for given solver provided in \xmlNode{solver}. A simple XML nodes only contains
  node tag and node text can be used to provide the options for the solver. For example:
  \begin{lstlisting}[style=XML]
    <solverOptions>
      <threads>1</threads>
      <StochSolver>EF</StochSolver>
    </solverOptions>
  \end{lstlisting}
\end{itemize}

Example XML:
\begin{lstlisting}[style=XML]
<Settings>
  <regulatoryMandated>
    PresurizerReplacement
    ...
    ReplaceInstrumentationAndControlCables
  </regulatoryMandated>
  <nonSelection>True</nonSelection>
  <consistentConstraintI>True</consistentConstraintI>
  <consistentConstraintII>True</consistentConstraintII>
  <solver>cbc</solver>
  <solverOptions>
    <threads>1</threads>
    <StochSolver>EF</StochSolver>
  </solverOptions>
  <sense>maximize</sense>
  <problem_type>mckp</problem_type>
</Settings>
\end{lstlisting}

    \section{Deterministic Capital Budgeting}
\label{sec:DeterministicCapitalBudgeting}

We consider a capital budgeting problem for a nuclear generation station, with possible extension to
a larger fleet of plants. Due to limited resources, we can only select a subset from a list of
several candidate capital projects. Our goal is to maximize overall NPV associated with the
selected subset. In doing so, we must respect resource limits and capture key structural and
stochastic dependencies of the system, although in this section we start with the simpler
deterministic case, ignoring randomness.  Example projects include upgrading a steam turbine,
refurbishing or replacing a set of reactor coolant pumps, and replacing a set of feed-water heaters.

\[
\begin{array}{ll}
%%%%%%%%%%%%%% INDICES AND SET %%%%%%%%%%%%%%%%
\multicolumn{2}{l}{\mbox{\em Indexes and sets:} } \\
t \in T  & \mbox{time periods (years)} \\
i \in I  & \mbox{investment candidate projects} \\
j \in J_{i}	& \mbox{options for selecting project $i$} \\%, e.g., initiate project $i$ in year $t$ or $t+2$ and in a standard (three year) or in an expedited (two year) manner} \\
% i^{'},j^{'} \in IJ_{ij} & \mbox{piggybacking situations} \\%, i.e., option $j^{'}$ for project $i^{'}$ can be selected only if option $j$ is selected for project $i$} \\
k \in K	& \mbox{types of resources} \\%, e.g., capital funds, O\&M funds, labor-hours, time during outage} \\
\\
%%%%%%%%%%%%%% DATA %%%%%%%%%%%%%%%%
\multicolumn{2}{l}{\mbox{\em Data:}} \\
a_{ij} & \mbox{reward (revenue less financial cost) of selecting project $i$ via option $j$}  \\
b_{kt} & \mbox{available budget for a resource of type $k$ in year $t$}\\
c_{ijkt}  & \mbox{consumption of resource of type $k$ in year $t$ if project $i$ is performed via option $j$} \\
\\
%%%%%%%%%%%%%% DECISION VARS %%%%%%%%%%%%%%%%
\multicolumn{2}{l}{\mbox{\em Decision variables:}}  \\
x_{ij} & \mbox{1 if project $i$ is selected via option $j$; 0 otherwise} \hspace*{4.0in}\\
\end{array}
\]

\vst \noi {\em Formulation:}
\begin{subequations}\label{model-deter}
\begin{eqnarray}
&\dst \max_{x} &  \dst \sum_{i \in I, j \in J_{i}} a_{ij} x_{ij} \label{obj_deter} \\
& s. t.  & \sum_{j \in J_{i}} x_{ij} = 1,   i \in I \\
& & \sum_{i \in I, j \in J_{i}} c_{ijkt} x_{ij} \leq b_{kt}, k \in K, t \in T \\
& & x_{ij} \in \{0,1\}, j \in J_{i}, i \in I.
\end{eqnarray}
\end{subequations}

The decision variables, $x_{ij}$, indicate whether we choose to do project i by means j. Restated,
if $x_{ij}=1$, then we recommend doing project $i$ via option $j$, and taken together these decision
variables produce both a portfolio of selected projects and a schedule for performing those projects
over time.  The set of available options, $j \in J_i$, can explicitly include the “do-nothing” option,
and the first constraint ensures that we choose exactly one option from the available set for each
project, including the possibility of selecting the do-nothing option. Even if we select the
do-nothing option for a project, it induces an NPV, $a_{ij}$, which may be negative, representing
growing O\&M costs, losses in plant efficiency, etc. The second structural constraint ensures that
the budget of each resource $k$ is respected in each year $t$. The objective function
includes the NPV for each project-option pair, $a_{ij}$, and the correct NPV is selected by
the $0-1$ decision variable, $x_{ij}$.

% The third structural constraint
% captures piggybacking situations in which option $j^{'}$ for project $i^{'}$ (which may have cheaper
% costs) may be selected only if project-option pair $(i,j)$ is also selected.

We note that sometimes there are projects that must be done, e.g., for safety and or regulatory
reasons. This can be handled within the mathematical formulation just given, without introducing
additional constructs. The set $J_i$ typically includes a do-nothing option for each project,
but when project $i$ must be done, we simply do not include the do-nothing option. Mathematically,
an alternative is to not include an explicit do-nothing option, to replace the first structural
constraint with an inequality, and to add an additional set of must-do projects with an equality
constraint. Both options are mathematically equivalent and simply represent a choice to be made by
the analyst. In LOGOS, we use \xmlNode{regulatoryMandated} and \xmlNode{nonSelection} to
handle these conditions. \xmlNode{regulatoryMandated} is used to specify the must-do projects,
while \xmlNode{nonSelection} is used to activate the do-nothing option.

\nb If a project is listed under \xmlNode{regulatoryMandated}, do-nothing option is not allowed
for this project. In addition, Handling the do-nothing option implicitly leads to NPV be
calculated relative to that of the do-nothing option.

The objective of capital budgeting is to find the combination of the binary decisions for
every investment such that the overall profit is as large as possible. The output is a
collection of projects to be carried out, and we refer this selected collection of projects
as a project portfolio. However, as it is frequently the case for capital budgeting with
NPP applications, in practice several optional constraints, such as resources/liabilities,
dependencies/synergies, options, time windows for every investment etc., have to be
fulfilled. This leads to a various variations of the knapsack problem (described above).
In the following sub-section, we will present several different variants of about
capital budgeting problem, i.e. different variants of knapsack problem.

\subsection{Single Knapsack Problem Optimization}
\label{subsec:skp}

\subsubsection{Simple Knapsack Problem}
The simple Knapsack Problem (KP) for capital budgeting can be defined as follows:
We are given an instance of the capital budgeting problem with investment set $I$,
consisting of $I$ investments $i$ with profit $a_i$, e.g. NPV, and cost $c_i$,
and the available budget $b$. Then the objective is to select a subset of $I$ such
that the total profit of the selected investments is maximized and total cost does
not exceed $b$. Alternatively, KP can be formulated as a solution of the following
linear integer programming formulation:

\vst \noi {\em Formulation:}
\begin{subequations}\label{simpleKP}
\begin{eqnarray}
&\dst \max_{x} &  \dst \sum_{i \in I} a_{i} x_{i} \\
& s. t. & \sum_{i \in I} c_{i} x_{i} \leq b\\
& & x_{i} \in \{0,1\}, i \in I.
\end{eqnarray}
\end{subequations}

Example LOGOS input XML:
\begin{lstlisting}[style=XML]
<Logos>
  <Sets>
    <investments>
      1,2,3,4,5,6,7,8,9,10
    </investments>
  </Sets>

  <Parameters>
    <net_present_values index="investments">
      18,20,17,19,25,21,27,23,25,24
    </net_present_values>
    <costs index="investments">
      1,3,7,4,8,9,6,10,2,5
    </costs>
    <available_capitals>
      15
    </available_capitals>
  </Parameters>

  <Settings>
    <solver>cbc</solver>
    <sense>maximize</sense>
  </Settings>
</Logos>
\end{lstlisting}

When run this case, LOGOS would generate a CSV (comma separated values) file that
contains solutions for the optimization problem, i.e. values of decision variables
and maximum profit (MaxNPV is used to describe the maximum profit). The header of
this CSV file contains the indices listed under \xmlNode{investments} which are
used as indices for decision variables, and the objective variable \textbf{MaxNPV}.
The data provides the values for both decision variables and objective variable.

Example LOGOS output CSV:
\begin{lstlisting}[language=python]
1,2,3,4,5,6,7,8,9,10,MaxNPV
1.0,1.0,0.0,1.0,0.0,0.0,0.0,0.0,1.0,1.0,106.0
\end{lstlisting}

In this case, the projects \textbf{1, 2, 4, 9, 10} are selected with maximum
profit 106.0.

\subsubsection{Bounded Knapsack Problem}
In the capital budgeting problem described above it may be the case that not all
investments/projects are different from each other. In particular, in practice
there may be given a number $n_i$ of identical pumps/valves to be replaced. In this
case the number of decision variables is equal to the number of different
investments instead of the total number of investments. The constraint for
decision variable becomes:
\begin{equation}
0\leq x_i \leq n_i, i\in N
\end{equation}
The resulting problem is called the bounded knapsack problem (BKP) formally defined by

\vst \noi {\em Formulation:}
\begin{subequations}\label{boundedKP}
\begin{eqnarray}
&\dst \max_{x} &  \dst \sum_{i \in I} a_{i} x_{i} \\
& s. t. & \sum_{i \in I} c_{i} x_{i} \leq b\\
& & x_{i} \in \{0,n_i\}, i \in I.
\end{eqnarray}
\end{subequations}

Example LOGOS input XML:
\begin{lstlisting}[style=XML]
<Logos>
  <Sets>
    <investments>
      1, 2, 3, 4, 5, 6, 7, 8, 9, 10, 11, 12, 13, 14, 15, 16, 17, 18, 19, 20, 21, 22
    </investments>
  </Sets>

  <Parameters>
    <net_present_values index="investments">
      150,35,200,60,60,45,60,40,30,10,70,30,15,10,40,70,75,80,20,12,50,10
    </net_present_values>
    <costs index="investments">
      9,13,153,50,15,68,27,39,23,52,11,32,24,48,73,42,43,22,7,18,4,30
    </costs>
    <available_capitals>
      400
    </available_capitals>
  </Parameters>

  <Settings>
    <lowerBounds>
      0, 0, 0, 0, 0, 0, 0, 0, 0, 0, 0, 0, 0, 0, 0, 0, 0, 0, 0, 0, 0, 0
    </lowerBounds>
    <upperBounds>
      1,1,2,2,2,3,3,3,1,3,1,1,2,2,1,1,1,1,1,2,1,2
    </upperBounds>
    <solver>glpk</solver>
    <sense>maximize</sense>
  </Settings>
</Logos>
\end{lstlisting}

Example LOGOS output CSV:
\begin{lstlisting}[language=python]
1,2,...,21,22,MaxNPV
1.0,1.0,...,1.0,0.0,1010.0
\end{lstlisting}

\subsubsection{Multi-Dimensional Knapsack Problem: DKP}
Moving in a different direction, if we now take into account not only the cost constraint but also
the limited commitment of critical resources, including: (i) capital cost, (ii) operation
and maintenance costs, (iii) time and labor-hours during a planned outage, (iv) personnel,
installation and maintenance equipment, space, and more. Denoting the cost of every
investment by $c_{ik}$ for each resource $k$ and introduce the corresponding limited resource
$b_k$ we can formulate the capital budgeting problem as multi-dimensional knapsack problem
or D-dimensional knapsack problem formally defined by:

\vst \noi {\em Formulation:}
\begin{subequations}\label{boundedKP}
\begin{eqnarray}
&\dst \max_{x} &  \dst \sum_{i \in I} a_{i} x_{i} \\
& s. t. & \sum_{i \in I} c_{ik} x_{i} \leq b_k, k\in K\\
& & x_{i} \in \{0,1\}, i \in I.
\end{eqnarray}
\end{subequations}

Where the limited resources set is denoted by $K$, consisting of $k$ “colors” of money
within capital costs, within operation and maintenance costs, within personnel availability, etc.
Another example is that the plant has multi-year investments. Consider a DKP problem in
which the costs of each investment and the available capitals vary according to time
period $t$. By defining $c_{it}$ as the cost of investment $i$ at time period $i$,
and $b_t$ as the available capital at time period $t$, we get:

\vst \noi {\em Formulation:}
\begin{subequations}\label{boundedKP}
\begin{eqnarray}
&\dst \max_{x} &  \dst \sum_{i \in I} a_{i} x_{i} \\
& s. t. & \sum_{i \in I} c_{it} x_{i} \leq b_t, t\in T\\
& & x_{i} \in \{0,1\}, i \in I.
\end{eqnarray}
\end{subequations}

Example LOGOS input XML:
\begin{lstlisting}[style=XML]
<Logos>
  <Sets>
    <investments>
      1,2,3,4,5,6,7,8,9
    </investments>
    <time_periods>
      1,2,3,4,5
    </time_periods>
  </Sets>

  <Parameters>
    <net_present_values index="investments">
      2.315,0.824,22.459,60.589,0.667,5.173,4.003,0.582,0.122
    </net_present_values>
    <costs index="investments, time_periods">
      0.219,0.257,0.085,0.0,0.0,
      0.0,0.0,0.122,0.103,0.013,
      5.044,1.839,0.0,0.0,0.0,
      6.74,6.134,10.442,0.0,0.0,
      0.425,0.0,0.0,0.0,0.0,
      2.125,2.122,0.0,0.0,0.0,
      2.387,0.19,0.012,2.383,0.192,
      0.0,0.95,0.0,0.0,0.0,
      0.03,0.03,0.688,0.0,0.0
    </costs>
    <available_capitals index="time_periods">
      0.665,4.712,9.642,3.458,1.683
    </available_capitals>
  </Parameters>

  <Settings>
    <solver>glpk</solver>
    <sense>maximize</sense>
  </Settings>
</Logos>
\end{lstlisting}

Example LOGOS output CSV:
\begin{lstlisting}[language=python]
1,2,3,4,5,6,7,8,9,MaxNPV
1.0,1.0,0.0,0.0,1.0,0.0,0.0,1.0,0.0,4.388
\end{lstlisting}


\subsection{Multiple Knapsack Problem Optimization}
\label{subsec:mkp}


\subsection{Multiple-Choice Multi-Dimensional Knapsack Problem Optimization}
\label{subsec:mckp}

    \section{Prioritizing Project Selection to Hedge Against Uncertainty}
\label{sec:StochasticCapitalBudgeting}

One limitation of traditional optimization models for capital budgeting is that
they do not account for risk/uncertainty in profit and cost streams associated
with individual projects, nor do they account for risk in resource availability
in future years. Projects can incur cost overruns, especially when projects are
large, performed infrequently, or when there is risk regarding technical viability,
external contractors, and/or suppliers of requisite parts and materials.
Occasionally, projects are performed ahead of schedule and with savings in cost.
Planned budgets for capital improvements can be cut, and key personnel may be
lost. Or, there may be surprise budgetary windfalls for maintenance activities
due to decreased costs for “unplanned” maintenance. In such cases, how should
we resolve capital budgeting when we have risk forecasts for costs, profits, and
budgets? One approach is to re-solve the models described in the previous
section once refined forecasts for these parameters become available. However,
it is not always practical to fully revise a project portfolio whenever better
forecasts become available.

In order to prioritize the project selection using a risk forecast for these
parameters, the two-stage stochastic optimization model~\cite{PrioritizingProjectSelection} is employed to provide
priority lists to decision-makers to support better risk-informed decisions.
Its inputs include those described in above sections for different variants of
the capital budgeting problem, except that a probabilistic description of the
uncertain parameters is integrated into the optimization process. The two-stage
stochastic optimization model forms a priority list as its first-stage decision,
then forms a corresponding project portfolio for each scenario as its
second-stage decision. When forming the optimal second-stage project portfolio
under a specific scenario, the stochastic optimization model ensures that the
portfolio is consistent with the first-stage prioritization (i.e., a project can
be selected only if all high-priority projects are also selected.) Thus, the
portfolios of projects corresponding to different scenarios are nested.

The notation and formulation of the risk-informed models are as follows:

\[
\begin{array}{ll}
%%%%%%%%%%%%%% INDICES AND SET %%%%%%%%%%%%%%%%
\multicolumn{2}{l}{\mbox{\em Indices and sets:} } \\
t \in T  & \mbox{time periods (years)} \\
i,i^{'} \in I  & \mbox{candidate projects} \\
j \in J_{i}	& \mbox{options for selecting project $i$} \\
%i^{'},j^{'} \in IJ_{ij} & \mbox{piggybacking situations} \\
k \in K	& \mbox{types of resources} \\
m \in M & \mbox{units of NPP} \\
\omega \in \Omega & \mbox{scenarios}\\
\\
%%%%%%%%%%%%%% DATA %%%%%%%%%%%%%%%%
\multicolumn{2}{l}{\mbox{\em Data:}} \\
a_{i}^{\omega} & \mbox{reward for selecting project $i$ under scenario $\omega$}  \\
a_{ij}^{\omega} & \mbox{reward for selecting project $i$ via option $j$ under scenario $\omega$}  \\
b^{\omega} & \mbox{available budget under scenario $\omega$}\\
b_{k}^{\omega} & \mbox{available budget for a resource of type $k$ under scenario $\omega$}\\
b_{t}^{\omega} & \mbox{available budget in year $t$ under scenario $\omega$}\\
b_{m}^{\omega} & \mbox{available budget for unit $m$ under scenario $\omega$}\\
b_{kt}^{\omega} & \mbox{available budget for a resource of type $k$ in year $t$ under scenario $\omega$}\\
c_{i}^{\omega} & \mbox{cost of investment $i$ under scenario $\omega$} \\
c_{ik}^{\omega} & \mbox{consumption of resource of type $k$} \\
& \mbox{if project $i$ is selected under scenario $\omega$}\\
c_{ijt}^{\omega} & \mbox{consumption of resource in year $t$ } \\
& \mbox{if project $i$ is performed via option $j$ under scenario $\omega$}\\
c_{ijkt}^{\omega} & \mbox{consumption of resource of type $k$ in year $t$ } \\
& \mbox{if project $i$ is performed via option $j$ under scenario $\omega$}\\
q^{\omega} & \mbox{probability of scenario $\omega$}\\
\\
%%%%%%%%%%%%%% DECISION VARS %%%%%%%%%%%%%%%%
\multicolumn{2}{l}{\mbox{\em Decision variables:}}  \\
x_{i}^{\omega} & \mbox{1 if project $i$ is selected under scenario $\omega$; 0 otherwise} \hspace*{4.0in}\\
x_{im}^{\omega} & \mbox{1 if project $i$ is selected for unit $m$ under scenario $\omega$; 0 otherwise} \hspace*{4.0in}\\
x_{ij}^{\omega} & \mbox{1 if project $i$ is selected via option $j$ under scenario $\omega$; 0 otherwise} \hspace*{4.0in}\\
y_{ii^{'}} & \mbox{1 if project $i$ has no lower priority than project $i^{'}$; 0 otherwise} \hspace*{4.0in}\\
\end{array}
\]

\subsection{Risk-Informed Single Knapsack Problem Optimization}
\label{subsec:RIskp}

\subsubsection{Risk-Informed Simple Knapsack Problem}
\vst \noi {\em Formulation:}
\begin{subequations}\label{RISimpleKP}
\begin{eqnarray}
&\dst \max_{x} &  \dst \sum_{\omega\in\Omega} q^\omega \dst \sum_{i \in I} a_{i}^\omega x_{i}^\omega \\
& s. t. & \sum_{i \in I} c_{i}^\omega x_{i}^\omega \leq b^\omega \label{stoc_cona}\\
& & y_{ii'} + y_{i'i} \geq 1, i<i' \label{stoc_conb} \\
& & x_{i}^\omega \geq x_{i'}^\omega + y_{ii'}-1, i\neq i' \label{stoc_conc}
\end{eqnarray}
\end{subequations}
For simplicity in what follows, the variable $y_{ii'}=1$ means that project $i$
is of higher priority than $i'$, even though the variable definition allows for ties
(i.e., projects of the same priority).
Constraint~(\ref{stoc_cona}) requires us to be within budget under each scenario.
Constraint~(\ref{stoc_conb}) indicates that either project $i$ is of higher priority
than project $i'$, or vice versa, or that both are of equal priority (i.e., a tie).
Constraint~(\ref{stoc_conc}) indicates that if project $i$ is of higher priority than
project $i'$ ($y_{ii'}=1$), and we select the lower priority project, then we
must also select the higher priority project; if $y_{ii'}=0$, or if $x_{i'}^\omega=0$,
then the constraint is vacuous.
In order to handle the risk in the capital budgeting problems, the entity
\xmlNode{Uncertainties} (see section~\ref{subsec:Uncertainties}) is used to specify
different scenarios of input parameters.

Example LOGOS input XML:
\begin{lstlisting}[style=XML]
<Logos>
  <Sets>
    <investments>
      1,2,3,4,5,6,7,8,9,10
    </investments>
  </Sets>

  <Parameters>
    <net_present_values index="investments">
      18,20,17,19,25,21,27,23,25,24
    </net_present_values>
    <costs index="investments">
      1,3,7,4,8,9,6,10,2,5
    </costs>
    <available_capitals>
      15
    </available_capitals>
  </Parameters>

  <Uncertainties>
    <available_capitals>
      <totalScenarios>10</totalScenarios>
      <probabilities>
        0.012, 0.019, 0.032, 0.052, 0.086, 0.142, 0.235, 0.188, 0.141, 0.093
      </probabilities>
      <scenarios>
        11, 12, 13, 14, 15, 16, 17, 18, 19, 20
      </scenarios>
    </available_capitals>
    <net_present_values>
      <totalScenarios>2</totalScenarios>
      <probabilities>
        0.3, 0.7
      </probabilities>
      <scenarios>
        18,20,17,19,25,21,27,23,25,24,
        18,20,17,19,25,21,27,23,25,24
      </scenarios>
    </net_present_values>
  </Uncertainties>
  ...
</Logos>
\end{lstlisting}

When running this case, LOGOS would generate a CSV (comma separated values) file
containing solutions for the optimization problem (i.e. values of decision variables
and maximum profit [MaxNPV is used to describe the maximum profit]). The header of
this CSV file contains the indices listed under \xmlNode{investments}
used as indices for decision variable, the objective variable \textbf{MaxNPV},
the scenario name and the associated probability weight.
The data provides the values for both decision variables and the objective variable.

Example LOGOS output CSV:
\begin{lstlisting}[basicstyle=\small,language=python]
1,10,2,3,4,5,6,7,8,9,ScenarioName,ProbabilityWeight,MaxNPV
1.0,0.0,0.0,0.0,0.0,0.0,0.0,1.0,0.0,1.0,scenario_1,0.0036,70.0
1.0,0.0,0.0,0.0,0.0,0.0,0.0,1.0,0.0,1.0,scenario_6,0.0224,70.0
1.0,0.0,0.0,0.0,0.0,0.0,0.0,1.0,0.0,1.0,scenario_5,0.0096,70.0
1.0,0.0,0.0,0.0,0.0,0.0,0.0,1.0,0.0,1.0,scenario_4,0.0133,70.0
1.0,0.0,0.0,0.0,0.0,0.0,0.0,1.0,0.0,1.0,scenario_3,0.0057,70.0
1.0,0.0,0.0,0.0,0.0,0.0,0.0,1.0,0.0,1.0,scenario_2,0.0084,70.0
1.0,1.0,0.0,0.0,0.0,0.0,0.0,1.0,0.0,1.0,scenario_7,0.0156,94.0
1.0,1.0,0.0,0.0,0.0,0.0,0.0,1.0,0.0,1.0,scenario_8,0.0364,94.0
1.0,1.0,0.0,0.0,0.0,0.0,0.0,1.0,0.0,1.0,scenario_9,0.0258,94.0
1.0,1.0,0.0,0.0,0.0,0.0,0.0,1.0,0.0,1.0,scenario_12,0.0994,94.0
1.0,1.0,0.0,0.0,0.0,0.0,0.0,1.0,0.0,1.0,scenario_11,0.0426,94.0
1.0,1.0,0.0,0.0,0.0,0.0,0.0,1.0,0.0,1.0,scenario_10,0.0602,94.0
1.0,1.0,1.0,0.0,0.0,0.0,0.0,1.0,0.0,1.0,scenario_16,0.1316,114.0
1.0,1.0,1.0,0.0,0.0,0.0,0.0,1.0,0.0,1.0,scenario_19,0.0279,114.0
1.0,1.0,1.0,0.0,0.0,0.0,0.0,1.0,0.0,1.0,scenario_15,0.0564,114.0
1.0,1.0,1.0,0.0,0.0,0.0,0.0,1.0,0.0,1.0,scenario_20,0.0651,114.0
1.0,1.0,1.0,0.0,0.0,0.0,0.0,1.0,0.0,1.0,scenario_14,0.1645,114.0
1.0,1.0,1.0,0.0,0.0,0.0,0.0,1.0,0.0,1.0,scenario_13,0.0705,114.0
1.0,1.0,1.0,0.0,0.0,0.0,0.0,1.0,0.0,1.0,scenario_17,0.0423,114.0
1.0,1.0,1.0,0.0,0.0,0.0,0.0,1.0,0.0,1.0,scenario_18,0.0987,114.0
\end{lstlisting}

\subsubsection{Risk-Informed Multi-Dimensional Knapsack Problem}

\vst \noi {\em Formulation:}
\begin{subequations}\label{RISimpleKP}
\begin{eqnarray}
&\dst \max_{x} &  \dst \sum_{\omega\in\Omega} q^\omega \dst \sum_{i \in I} a_{i}^\omega x_{i}^\omega \\
& s. t. & \sum_{i \in I} c_{it}^\omega x_{i}^\omega \leq b_{t}^\omega, t\in T \\
& & y_{ii'} + y_{i'i} \geq 1, i<i'\\
& & x_{i}^\omega \geq x_{i'}^\omega + y_{ii'}-1, i\neq i'
\end{eqnarray}
\end{subequations}

Example LOGOS input XML:
\begin{lstlisting}[style=XML]
<Logos>
  <Sets>
    <investments>
      1,2,3,4,5,6,7,8,9,10,11,12,13,14,15,16
    </investments>
    <time_periods>
      1,2,3,4,5
    </time_periods>
  </Sets>

  <Parameters>
    <net_present_values index="investments">
      2.315,0.824,22.459,60.589,0.667,5.173,4.003,0.582,0.122,
      -2.870,-0.102,-0.278,-0.322,-3.996,-0.246,-20.155
    </net_present_values>
    <costs index="investments, time_periods">
      0.219,0.257,0.085,0.0,0.0,
      0.0,0.0,0.122,0.103,0.013,
      5.044,1.839,0.0,0.0,0.0,
      6.74,6.134,10.442,0.0,0.0,
      0.425,0.0,0.0,0.0,0.0,
      2.125,2.122,0.0,0.0,0.0,
      2.387,0.19,0.012,2.383,0.192,
      0.0,0.95,0.0,0.0,0.0,
      0.03,0.03,0.688,0.0,0.0,
      0,0.2,0.763,0.739,2.539,
      0.081,0.032,0,0,0,
      0.3,0,0,0,0,
      0.347,0,0,0,0,
      4.025,0.297,0,0,0,
      0.095,0.095,0.095,0,0,
      5.487,5.664,0.5,6.803,6.778
    </costs>
    <available_capitals index="time_periods">
      18,18,18,18,18
    </available_capitals>
  </Parameters>

  <Uncertainties>
    <available_capitals>
      <totalScenarios>10</totalScenarios>
      <probabilities>
        0.012, 0.019, 0.032, 0.052, 0.086, 0.142, 0.235, 0.188, 0.141, 0.093
      </probabilities>
      <!--
        scenarios is ordered by numberScenarios * parametersIndex, the number of
        scenarios is determined by the number of elements in <probabilities>,
        for this case total element in scenarios:
        numberScenarios * time_periods = 10 * 5
      -->
      <scenarios>
        11, 11, 11, 11, 11,
        12, 12, 12, 12, 12,
        13, 13, 13, 13, 13,
        14, 14, 14, 14, 14,
        15, 15, 15, 15, 15,
        16, 16, 16, 16, 16,
        17, 17, 17, 17, 17,
        18, 18, 18, 18, 18,
        19, 19, 19, 19, 19,
        20, 20, 20, 20, 20
      </scenarios>
    </available_capitals>
  </Uncertainties>
  <Settings>
    <regulatoryMandated>10,11,12,13,14,15,16</regulatoryMandated>
    <solver>cbc</solver>
    <sense>maximize</sense>
  </Settings>
</Logos>
\end{lstlisting}

Example LOGOS output CSV:
\begin{lstlisting}[basicstyle=\tiny,language=python]
1,10,11,12,13,14,15,16,2,3,4,5,6,7,8,9,ScenarioName,ProbabilityWeight,MaxNPV
1.0,1.0,1.0,1.0,1.0,1.0,1.0,1.0,1.0,0.0,0.0,1.0,0.0,0.0,1.0,0.0,scenario_1,0.012,-23.581
1.0,1.0,1.0,1.0,1.0,1.0,1.0,1.0,1.0,0.0,0.0,1.0,0.0,0.0,1.0,1.0,scenario_2,0.019,-23.459
1.0,1.0,1.0,1.0,1.0,1.0,1.0,1.0,1.0,0.0,0.0,1.0,0.0,0.0,1.0,1.0,scenario_3,0.032,-23.459
1.0,1.0,1.0,1.0,1.0,1.0,1.0,1.0,1.0,0.0,0.0,1.0,0.0,0.0,1.0,1.0,scenario_4,0.052,-23.459
1.0,1.0,1.0,1.0,1.0,1.0,1.0,1.0,1.0,0.0,0.0,1.0,0.0,0.0,1.0,1.0,scenario_5,0.086,-23.459
1.0,1.0,1.0,1.0,1.0,1.0,1.0,1.0,1.0,0.0,0.0,1.0,0.0,0.0,1.0,1.0,scenario_6,0.142,-23.459
1.0,1.0,1.0,1.0,1.0,1.0,1.0,1.0,1.0,0.0,0.0,1.0,0.0,0.0,1.0,1.0,scenario_7,0.235,-23.459
1.0,1.0,1.0,1.0,1.0,1.0,1.0,1.0,1.0,0.0,1.0,1.0,0.0,0.0,1.0,1.0,scenario_8,0.188,37.130
1.0,1.0,1.0,1.0,1.0,1.0,1.0,1.0,1.0,0.0,1.0,1.0,0.0,0.0,1.0,1.0,scenario_9,0.141,37.1230
1.0,1.0,1.0,1.0,1.0,1.0,1.0,1.0,1.0,0.0,1.0,1.0,1.0,0.0,1.0,1.0,scenario_10,0.093,42.303
\end{lstlisting}

\subsection{Risk-Informed Multiple Knapsack Problem Optimization}
\label{subsec:RImkp}

Model formulation:\\

\begin{equation}\label{rimkp_obja}
\mathop{\max}_{x,y} \sum _{ \omega  \in  \Omega }q^{ \omega } \sum _{i \in I} \sum _{m \in M}a_{i}^{ \omega }x_{im}^{ \omega }
\end{equation}

\begin{equation}\label{rimkp_objb}
~~~~~~~~~~~~s.t.~~~~~y_{ii^{'}}+y_{i^{'}i} \geq 1,~ i<i^{'}\text{, i, }i^{'} \in I
\end{equation}

\begin{equation}\label{rimkp_objc}
~~~~~~~~\sum_{m=1}^{M} x_{im}^\omega \geq \sum_{m=1}^{M} x_{i'm}^\omega + y_{ii'} -1,~ i \neq i^{'}\text{, i, }i^{'} \in I,  \omega  \in  \Omega
\end{equation}

Constraint~(\ref{rimkp_objb}) indicates that either project $i$  is of higher priority
than project  $i^{'}$, or vice versa, or that both are of equal priority (i.e., a tie).
Constraint~(\ref{rimkp_objc}) indicates that if project  $i$  is higher priority than
project  $i^{'}$ ($y_{ii^{'}}=1$), and we select the lower priority project
\textit{for some unit}, then we must also select the higher priority project;
if  $y_{ii^{'}}=0$, or if $\sum_{m=1}^{M} x_{i'm}^\omega=0$  then the constraint is vacuous.\par

\begin{equation}\label{rimkp_objd}
 \sum _{i \in I}^{} c_{i}^{ \omega }x_{im}^{ \omega }~  \leq  b_{m}^{ \omega },~ m \in M,  \omega  \in  \Omega
\end{equation}

Constraint~(\ref{rimkp_objd}) requires that we be within budget
for each unit under each scenario.

\begin{equation}\label{rimkp_obje}
\sum_{m\in M} x_{im}^{ \omega } \leq 1,~ i \in I, \omega  \in  \Omega
\end{equation}

Constraint~(\ref{rimkp_obje}) simultaneously ensures that we select project $i$
only for one unit.

Example LOGOS input XML:
\begin{lstlisting}[style=XML]
  <Sets>
    <investments>
      1,2,3,4,5,6,7,8,9,10
    </investments>
    <capitals>
      unit_1, unit_2
    </capitals>
  </Sets>

  <Parameters>
    <net_present_values index="investments">
      78, 35, 89, 36, 94, 75, 74, 79, 80, 16
    </net_present_values>
    <costs index="investments">
      18, 9, 23, 20, 59, 61, 70, 75, 76, 30
    </costs>
    <available_capitals index="capitals">
      103, 156
    </available_capitals>
  </Parameters>

  <Uncertainties>
    <available_capitals>
      <totalScenarios>2</totalScenarios>
      <probabilities>
        0.3 0.7
      </probabilities>
      <scenarios>
        103, 156,
        103, 156
      </scenarios>
    </available_capitals>
    <net_present_values>
      <totalScenarios>2</totalScenarios>
      <probabilities>
        0.3, 0.7
      </probabilities>
      <scenarios>
        78, 35, 89, 36, 94, 75, 74, 79, 80, 16,
        78, 35, 89, 36, 94, 75, 74, 79, 80, 16
      </scenarios>
    </net_present_values>
  </Uncertainties>

  <Settings>
    <solver>glpk</solver>
    <sense>maximize</sense>
    <problem_type>MultipleKnapsack</problem_type>
  </Settings>
</Logos>
\end{lstlisting}

Example LOGOS output CSV:
\begin{lstlisting}[basicstyle=\tiny,language=python]
"('1', 'unit_1')","('1', 'unit_2')","('10', 'unit_1')","('10', 'unit_2')","('2', 'unit_1')","('2', 'unit_2')","('3', 'unit_1')",
"('3', 'unit_2')","('4', 'unit_1')","('4', 'unit_2')","('5', 'unit_1')","('5', 'unit_2')","('6', 'unit_1')","('6', 'unit_2')",
"('7', 'unit_1')","('7', 'unit_2')","('8', 'unit_1')","('8', 'unit_2')","('9', 'unit_1')","('9', 'unit_2')",
ScenarioName,ProbabilityWeight,MaxNPV
1.0,0.0,0.0,0.0,0.0,0.0,1.0,0.0,0.0,1.0,0.0,1.0,1.0,0.0,0.0,0.0,0.0,0.0,0.0,1.0,scenario_1,0.09,452.0
1.0,0.0,0.0,0.0,0.0,0.0,1.0,0.0,0.0,1.0,0.0,1.0,1.0,0.0,0.0,0.0,0.0,0.0,0.0,1.0,scenario_2,0.21,452.0
1.0,0.0,0.0,0.0,0.0,0.0,1.0,0.0,0.0,1.0,0.0,1.0,1.0,0.0,0.0,0.0,0.0,0.0,0.0,1.0,scenario_3,0.21,452.0
0.0,1.0,0.0,0.0,0.0,0.0,1.0,0.0,1.0,0.0,1.0,0.0,0.0,1.0,0.0,0.0,0.0,0.0,0.0,1.0,scenario_4,0.49,452.0
\end{lstlisting}

\subsection{Risk-Informed Multiple-Choice Multi-Dimensional Knapsack Problem Optimization}
\label{subsec:RImckp}

Model formulation:\\

\begin{equation}\label{stoc_obja}
\mathop{\max}_{x,y} \sum _{ \omega  \in  \Omega }^{}q^{ \omega } \sum _{i \in I}^{} \sum _{j \in J_{i}}^{}a_{ij}^{ \omega }x_{ij}^{ \omega }
\end{equation}

\begin{equation}\label{stoc_objb}
~~~~~~~~~~~~s.t.~~~~~y_{ii^{'}}+y_{i^{'}i} \geq 1,~ i<i^{'}\text{, i, }i^{'} \in I
\end{equation}

\begin{equation}\label{stoc_objc}
~~~~~~~~\sum_{j=1}^{J_i} x_{ij}^\omega \geq \sum_{j=1}^{J_i} x_{i'j}^\omega + y_{ii'} -1,~ i \neq i^{'}\text{, i, }i^{'} \in I,  \omega  \in  \Omega
\end{equation}

Constraint~(\ref{stoc_objb}) indicates that either project $i$ is of higher priority
than project  $i^{'}$, or vice versa, or that both are of equal priority (i.e., a tie).
Constraint~(\ref{stoc_objc}) indicates that if project  $i$  is higher priority than
project  $i^{'}$  $y_{ii^{'}}=1$, and we select the lower priority project
\textit{under some option}, then we must also select the higher priority project;
if  $y_{ii^{'}}=0$,  or if  $\sum_{j=1}^{J_i} x_{i'j}^\omega=0$,  then the constraint is vacuous.\par

\begin{equation}\label{stoc_objd}
 \sum _{i \in I}^{} \sum _{j \in J_{i}}^{}\text{~ c}_{ijkt}^{ \omega }x_{ij}^{ \omega }~  \leq  b_{kt}^{ \omega },~ k \in K, t \in T,  \omega  \in  \Omega
\end{equation}

Constraint~(\ref{stoc_objd}) requires that we be within budget in each time period,
for each resource type, and under each scenario.

\begin{equation}\label{stoc_obje}
\sum_{j\in J_i} x_{ij}^{ \omega } \leq 1,~ i \in I, \omega  \in  \Omega
\end{equation}

Constraint~(\ref{stoc_obje}) simultaneously ensures that we select project  $i$
via, at most, one option. Note that this illustrates a situation in which we
must include the {\it DoNothing}  option among the alternatives to optional projects.\par

Example LOGOS input XML:
\begin{lstlisting}[style=XML]
<Logos>
  <Sets>
    <investments>
      1, 2, 3, 4, 5, 6, 7, 8, 9, 10, 11, 12, 13, 14, 15, 16, 17
    </investments>
    <options index='investments'>
      1;
      1;
      1;
      1,2,3;
      1,2,3,4;
      1,2,3,4,5,6,7;
      1;
      1;
      1;
      1;
      1;
      1;
      1;
      1;
      1;
      1;
      1
    </options>
  </Sets>

  <Parameters>
    <net_present_values index='options'>
      2.046
      2.679
      2.489
      2.61
      2.313
      1.02
      3.013
      2.55
      3.351
      3.423
      3.781
      2.525
      2.169
      2.267
      2.747
      4.309
      6.452
      2.849
      7.945
      2.538
      1.761
      3.002
      3.449
      2.865
      3.999
      2.283
      0.9
      8.608
    </net_present_values>
    <costs index='options'>
      36538462
      83849038
      4615385
      2788461538
      2692307692
      5480769231
      1634615385
      2981730768
      7211538462
      9038461538
      649038462
      650000000
      216346154
      212500000
      3076923077
      3942307692
      1144230769
      675721154
      1442307692
      99711538
      4807692
      123076923
      138461538
      86538462
      108653846
      75092404
      6413462
      147932692
    </costs>
    <available_capitals>
      15E9
    </available_capitals>
  </Parameters>

  <Uncertainties>
    <available_capitals>
      <totalScenarios>3</totalScenarios>
      <probabilities>
        0.2,0.6,0.2
      </probabilities>
      <scenarios>
        5E9,10E9,15E9
      </scenarios>
    </available_capitals>
  </Uncertainties>

  <Settings>
    <solver>cbc</solver>
    <solverOptions>
      <threads>1</threads>
      <StochSolver>EF</StochSolver>
    </solverOptions>
    <sense>maximize</sense>
    <problem_type>mckp</problem_type>
  </Settings>
</Logos>
\end{lstlisting}

Example LOGOS output CSV:
\begin{lstlisting}[basicstyle=\tiny,language=python]
"('1', '1')","('10', '1')","('11', '1')","('12', '1')","('13', '1')","('14', '1')","('15', '1')","('16', '1')",
"('17', '1')","('2', '1')","('3', '1')","('4', '1')","('4', '2')","('4', '3')","('5', '1')","('5', '2')","('5', '3')",
"('5', '4')","('6', '1')","('6', '2')","('6', '3')","('6', '4')","('6', '5')","('6', '6')","('6', '7')","('7', '1')",
"('8', '1')","('9', '1')",ScenarioName,ProbabilityWeight,MaxNPV
1.0,1.0,1.0,1.0,1.0,1.0,1.0,1.0,1.0,1.0,1.0,0.0,0.0,0.0,0.0,0.0,0.0,0.0,0.0,0.0,0.0,0.0,0.0,0.0,1.0,1.0,1.0,1.0,scenario_1,0.2,53.865
1.0,1.0,1.0,1.0,1.0,1.0,1.0,1.0,1.0,1.0,1.0,1.0,0.0,0.0,1.0,0.0,0.0,0.0,0.0,0.0,0.0,0.0,0.0,0.0,1.0,1.0,1.0,1.0,scenario_2,0.6,59.488
1.0,1.0,1.0,1.0,1.0,1.0,1.0,1.0,1.0,1.0,1.0,1.0,0.0,0.0,0.0,0.0,1.0,0.0,0.0,0.0,0.0,0.0,0.0,0.0,1.0,1.0,1.0,1.0,scenario_3,0.2,59.826
\end{lstlisting}


    \section*{Document Version Information}
    This document has been compiled using the following version of the plug-in git repository:
    \newline
    c5d7b2e5bfa7b6d5c9e94acbad8d0082d5a64ce6 Congjian Wang Wed, 8 May 2019 09:23:10 -0600


    % ---------------------------------------------------------------------- %
    % References
    %
    \clearpage
    % If hyperref is included, then \phantomsection is already defined.
    % If not, we need to define it.
    \providecommand*{\phantomsection}{}
    \phantomsection
    \addcontentsline{toc}{section}{References}
    \bibliographystyle{ieeetr}
    \bibliography{user_manual}


    % ---------------------------------------------------------------------- %

\end{document}
