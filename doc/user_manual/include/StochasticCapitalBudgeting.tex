\section{Prioritizing Project Selection to Hedge Against Uncertainty}
\label{sec:StochasticCapitalBudgeting}

One limitation of traditional optimization models for capital budgeting is that
they do not account for risk/uncertainty in profit and cost streams associated
with individual projects, they do not account for risk in resource availability
in future years. Projects can incur cost over-runs, especially when projects are
large, performed infrequently, and when there is risk regarding technical viability,
external contractors, and/or suppliers of requisite parts and materials.
Occasionally, projects are performed ahead of schedule and with cost savings.
Planned budgets for capital improvements can be cut and key personnel may be
lost. Or, there may be surprise windfalls in budgets for maintenance activities
due to decreased costs for “unplanned” maintenance. In these cases, how should
we resolve capital budgeting when we have risk forecasts for costs, profits and
budgets? One approach is to re-solve the models described in the previous
section when refined forecasts for these parameters become available. However,
it is not always practical to fully revise a project portfolio whenever better
forecasts become available.

In order to prioritize the project selection with risk forecast for these
parameters, the two-stage stochastic optimization model~\cite{PrioritizingProjectSelection} is employed to provide
priority lists to decision-makers to support better risk-informed decisions.
Its inputs include those described in above sections for different variant of
the capital budgeting problem, except that a probabilistic description of the
uncertain parameters is integrated in the optimization process. The two-stage
stochastic optimization model forms a priority list as its first- stage decision
and then forms a corresponding project portfolio for each scenario as its
second-stage decision. When forming the optimal second-stage project portfolio
under a specific scenario, the stochastic optimization model ensures that the
portfolio is consistent with the first-stage prioritization; i.e., a project can
be selected only if all high-priority projects are also selected. Thus, the
portfolios of projects corresponding to different scenarios are nested.

The notation and formulation of the risk-informed models are as follows:

\[
\begin{array}{ll}
%%%%%%%%%%%%%% INDICES AND SET %%%%%%%%%%%%%%%%
\multicolumn{2}{l}{\mbox{\em Indexes and sets:} } \\
t \in T  & \mbox{time periods (years)} \\
i,i^{'} \in I  & \mbox{candidate projects} \\
j \in J_{i}	& \mbox{options for selecting project $i$} \\
%i^{'},j^{'} \in IJ_{ij} & \mbox{piggybacking situations} \\
k \in K	& \mbox{types of resources} \\
m \in M & \mbox{units of NPP} \\
\omega \in \Omega & \mbox{scenarios}\\
\\
%%%%%%%%%%%%%% DATA %%%%%%%%%%%%%%%%
\multicolumn{2}{l}{\mbox{\em Data:}} \\
a_{i}^{\omega} & \mbox{reward of selecting project $i$ under scenario $\omega$}  \\
a_{ij}^{\omega} & \mbox{reward of selecting project $i$ via option $j$ under scenario $\omega$}  \\
b^{\omega} & \mbox{available budget under scenario $\omega$}\\
b_{k}^{\omega} & \mbox{available budget for a resource of type $k$ under scenario $\omega$}\\
b_{t}^{\omega} & \mbox{available budget in year $t$ under scenario $\omega$}\\
b_{m}^{\omega} & \mbox{available budget for unit $m$ under scenario $\omega$}\\
b_{kt}^{\omega} & \mbox{available budget for a resource of type $k$ in year $t$ under scenario $\omega$}\\
c_{i}^{\omega} & \mbox{cost of investment $i$ under scenario $\omega$} \\
c_{ik}^{\omega} & \mbox{consumption of resource of type $k$} \\
& \mbox{if project $i$ is selected under scenario $\omega$}\\
c_{ijt}^{\omega} & \mbox{consumption of resource in year $t$ } \\
& \mbox{if project $i$ is performed via option $j$ under scenario $\omega$}\\
c_{ijkt}^{\omega} & \mbox{consumption of resource of type $k$ in year $t$ } \\
& \mbox{if project $i$ is performed via option $j$ under scenario $\omega$}\\
q^{\omega} & \mbox{probability of scenario $\omega$}\\
\\
%%%%%%%%%%%%%% DECISION VARS %%%%%%%%%%%%%%%%
\multicolumn{2}{l}{\mbox{\em Decision variables:}}  \\
x_{i}^{\omega} & \mbox{1 if project $i$ is selected under scenario $\omega$; 0 otherwise} \hspace*{4.0in}\\
x_{im}^{\omega} & \mbox{1 if project $i$ is selected for unit $m$ under scenario $\omega$; 0 otherwise} \hspace*{4.0in}\\
x_{ij}^{\omega} & \mbox{1 if project $i$ is selected via option $j$ under scenario $\omega$; 0 otherwise} \hspace*{4.0in}\\
y_{ii^{'}} & \mbox{1 if project $i$ has no lower priority than project $i^{'}$; 0 otherwise} \hspace*{4.0in}\\
\end{array}
\]

\subsection{Risk-Informed Single Knapsack Problem Optimization}
\label{subsec:RIskp}

\subsubsection{Risk-Informed Simple Knapsack Problem}
\vst \noi {\em Formulation:}
\begin{subequations}\label{RISimpleKP}
\begin{eqnarray}
&\dst \max_{x} &  \dst \sum_{\omega\in\Omega} q^\omega \dst \sum_{i \in I} a_{i}^\omega x_{i}^\omega \\
& s. t. & \sum_{i \in I} c_{i}^\omega x_{i}^\omega \leq b^\omega \label{stoc_cona}\\
& & y_{ii'} + y_{i'i} \geq 1, i<i' \label{stoc_conb} \\
& & x_{i}^\omega \geq x_{i'}^\omega + y_{ii'}-1, i\neq i' \label{stoc_conc}
\end{eqnarray}
\end{subequations}
For simplicity, in what follows the variable $y_{ii'}=1$ means that project $i$
is higher priority than $i'$ even though the variable definition allows for ties,
i.e., the projects being the same priority.
Constraint~(\ref{stoc_cona}) requires that we be within budget under each scenarios.
Constraint~(\ref{stoc_conb}) indicates that either project $i$ is higher priority
than project $i'$ or vice versa, and further allows both (i.e., a tie).
Constraint~(\ref{stoc_conc}) indicates that if project $i$ is higher priority than
project $i'$ ($y_{ii'}=1$), then if we select the lower priority project then we
must also select the higher priority project; if $y_{ii'}=0$ or if $x_{i'}^\omega$
then the constraint is vacuous.
In order to handle the risk in the capital budgeting problems, the entity
\xmlNode{Uncertainties} (see section~\ref{subsec:Uncertainties}) is used to specify
different scenarios of input parameters.

Example LOGOS input XML:
\begin{lstlisting}[style=XML]
<Logos>
  <Sets>
    <investments>
      1,2,3,4,5,6,7,8,9,10
    </investments>
  </Sets>

  <Parameters>
    <net_present_values index="investments">
      18,20,17,19,25,21,27,23,25,24
    </net_present_values>
    <costs index="investments">
      1,3,7,4,8,9,6,10,2,5
    </costs>
    <available_capitals>
      15
    </available_capitals>
  </Parameters>

  <Uncertainties>
    <available_capitals>
      <totalScenarios>10</totalScenarios>
      <probabilities>
        0.012, 0.019, 0.032, 0.052, 0.086, 0.142, 0.235, 0.188, 0.141, 0.093
      </probabilities>
      <scenarios>
        11, 12, 13, 14, 15, 16, 17, 18, 19, 20
      </scenarios>
    </available_capitals>
    <net_present_values>
      <totalScenarios>2</totalScenarios>
      <probabilities>
        0.3, 0.7
      </probabilities>
      <scenarios>
        18,20,17,19,25,21,27,23,25,24,
        18,20,17,19,25,21,27,23,25,24
      </scenarios>
    </net_present_values>
  </Uncertainties>
  ...
</Logos>
\end{lstlisting}

When run this case, LOGOS would generate a CSV (comma separated values) file that
contains solutions for the optimization problem, i.e. values of decision variables
and maximum profit (MaxNPV is used to describe the maximum profit). The header of
this CSV file contains the indices listed under \xmlNode{investments} which are
used as indices for decision variables, the objective variable \textbf{MaxNPV},
the scenario name and the associated probability weight.
The data provides the values for both decision variables and objective variable.

Example LOGOS output CSV:
\begin{lstlisting}[basicstyle=\small,language=python]
1,10,2,3,4,5,6,7,8,9,ScenarioName,ProbabilityWeight,MaxNPV
1.0,0.0,0.0,0.0,0.0,0.0,0.0,1.0,0.0,1.0,scenario_1,0.0036,70.0
1.0,0.0,0.0,0.0,0.0,0.0,0.0,1.0,0.0,1.0,scenario_6,0.0224,70.0
1.0,0.0,0.0,0.0,0.0,0.0,0.0,1.0,0.0,1.0,scenario_5,0.0096,70.0
1.0,0.0,0.0,0.0,0.0,0.0,0.0,1.0,0.0,1.0,scenario_4,0.0133,70.0
1.0,0.0,0.0,0.0,0.0,0.0,0.0,1.0,0.0,1.0,scenario_3,0.0057,70.0
1.0,0.0,0.0,0.0,0.0,0.0,0.0,1.0,0.0,1.0,scenario_2,0.0084,70.0
1.0,1.0,0.0,0.0,0.0,0.0,0.0,1.0,0.0,1.0,scenario_7,0.0156,94.0
1.0,1.0,0.0,0.0,0.0,0.0,0.0,1.0,0.0,1.0,scenario_8,0.0364,94.0
1.0,1.0,0.0,0.0,0.0,0.0,0.0,1.0,0.0,1.0,scenario_9,0.0258,94.0
1.0,1.0,0.0,0.0,0.0,0.0,0.0,1.0,0.0,1.0,scenario_12,0.0994,94.0
1.0,1.0,0.0,0.0,0.0,0.0,0.0,1.0,0.0,1.0,scenario_11,0.0426,94.0
1.0,1.0,0.0,0.0,0.0,0.0,0.0,1.0,0.0,1.0,scenario_10,0.0602,94.0
1.0,1.0,1.0,0.0,0.0,0.0,0.0,1.0,0.0,1.0,scenario_16,0.1316,114.0
1.0,1.0,1.0,0.0,0.0,0.0,0.0,1.0,0.0,1.0,scenario_19,0.0279,114.0
1.0,1.0,1.0,0.0,0.0,0.0,0.0,1.0,0.0,1.0,scenario_15,0.0564,114.0
1.0,1.0,1.0,0.0,0.0,0.0,0.0,1.0,0.0,1.0,scenario_20,0.0651,114.0
1.0,1.0,1.0,0.0,0.0,0.0,0.0,1.0,0.0,1.0,scenario_14,0.1645,114.0
1.0,1.0,1.0,0.0,0.0,0.0,0.0,1.0,0.0,1.0,scenario_13,0.0705,114.0
1.0,1.0,1.0,0.0,0.0,0.0,0.0,1.0,0.0,1.0,scenario_17,0.0423,114.0
1.0,1.0,1.0,0.0,0.0,0.0,0.0,1.0,0.0,1.0,scenario_18,0.0987,114.0
\end{lstlisting}

\subsubsection{Risk-Informed Multi-Dimensional Knapsack Problem}

\vst \noi {\em Formulation:}
\begin{subequations}\label{RISimpleKP}
\begin{eqnarray}
&\dst \max_{x} &  \dst \sum_{\omega\in\Omega} q^\omega \dst \sum_{i \in I} a_{i}^\omega x_{i}^\omega \\
& s. t. & \sum_{i \in I} c_{it}^\omega x_{i}^\omega \leq b_{t}^\omega, t\in T \\
& & y_{ii'} + y_{i'i} \geq 1, i<i'\\
& & x_{i}^\omega \geq x_{i'}^\omega + y_{ii'}-1, i\neq i'
\end{eqnarray}
\end{subequations}

Example LOGOS input XML:
\begin{lstlisting}[style=XML]
<Logos>
  <Sets>
    <investments>
      1,2,3,4,5,6,7,8,9,10,11,12,13,14,15,16
    </investments>
    <time_periods>
      1,2,3,4,5
    </time_periods>
  </Sets>

  <Parameters>
    <net_present_values index="investments">
      2.315,0.824,22.459,60.589,0.667,5.173,4.003,0.582,0.122,
      -2.870,-0.102,-0.278,-0.322,-3.996,-0.246,-20.155
    </net_present_values>
    <costs index="investments, time_periods">
      0.219,0.257,0.085,0.0,0.0,
      0.0,0.0,0.122,0.103,0.013,
      5.044,1.839,0.0,0.0,0.0,
      6.74,6.134,10.442,0.0,0.0,
      0.425,0.0,0.0,0.0,0.0,
      2.125,2.122,0.0,0.0,0.0,
      2.387,0.19,0.012,2.383,0.192,
      0.0,0.95,0.0,0.0,0.0,
      0.03,0.03,0.688,0.0,0.0,
      0,0.2,0.763,0.739,2.539,
      0.081,0.032,0,0,0,
      0.3,0,0,0,0,
      0.347,0,0,0,0,
      4.025,0.297,0,0,0,
      0.095,0.095,0.095,0,0,
      5.487,5.664,0.5,6.803,6.778
    </costs>
    <available_capitals index="time_periods">
      18,18,18,18,18
    </available_capitals>
  </Parameters>

  <Uncertainties>
    <available_capitals>
      <totalScenarios>10</totalScenarios>
      <probabilities>
        0.012, 0.019, 0.032, 0.052, 0.086, 0.142, 0.235, 0.188, 0.141, 0.093
      </probabilities>
      <!--
        scenarios is ordered by numberScenarios * parametersIndex, the number of
        scenarios is determined by the number of elements in <probabilities>,
        for this case total element in scenarios:
        numberScenarios * time_periods = 10 * 5
      -->
      <scenarios>
        11, 11, 11, 11, 11,
        12, 12, 12, 12, 12,
        13, 13, 13, 13, 13,
        14, 14, 14, 14, 14,
        15, 15, 15, 15, 15,
        16, 16, 16, 16, 16,
        17, 17, 17, 17, 17,
        18, 18, 18, 18, 18,
        19, 19, 19, 19, 19,
        20, 20, 20, 20, 20
      </scenarios>
    </available_capitals>
  </Uncertainties>
  <Settings>
    <regulatoryMandated>10,11,12,13,14,15,16</regulatoryMandated>
    <solver>cbc</solver>
    <sense>maximize</sense>
  </Settings>
</Logos>
\end{lstlisting}

Example LOGOS output CSV:
\begin{lstlisting}[basicstyle=\tiny,language=python]
1,10,11,12,13,14,15,16,2,3,4,5,6,7,8,9,ScenarioName,ProbabilityWeight,MaxNPV
1.0,1.0,1.0,1.0,1.0,1.0,1.0,1.0,1.0,0.0,0.0,1.0,0.0,0.0,1.0,0.0,scenario_1,0.012,-23.581
1.0,1.0,1.0,1.0,1.0,1.0,1.0,1.0,1.0,0.0,0.0,1.0,0.0,0.0,1.0,1.0,scenario_2,0.019,-23.459
1.0,1.0,1.0,1.0,1.0,1.0,1.0,1.0,1.0,0.0,0.0,1.0,0.0,0.0,1.0,1.0,scenario_3,0.032,-23.459
1.0,1.0,1.0,1.0,1.0,1.0,1.0,1.0,1.0,0.0,0.0,1.0,0.0,0.0,1.0,1.0,scenario_4,0.052,-23.459
1.0,1.0,1.0,1.0,1.0,1.0,1.0,1.0,1.0,0.0,0.0,1.0,0.0,0.0,1.0,1.0,scenario_5,0.086,-23.459
1.0,1.0,1.0,1.0,1.0,1.0,1.0,1.0,1.0,0.0,0.0,1.0,0.0,0.0,1.0,1.0,scenario_6,0.142,-23.459
1.0,1.0,1.0,1.0,1.0,1.0,1.0,1.0,1.0,0.0,0.0,1.0,0.0,0.0,1.0,1.0,scenario_7,0.235,-23.459
1.0,1.0,1.0,1.0,1.0,1.0,1.0,1.0,1.0,0.0,1.0,1.0,0.0,0.0,1.0,1.0,scenario_8,0.188,37.130
1.0,1.0,1.0,1.0,1.0,1.0,1.0,1.0,1.0,0.0,1.0,1.0,0.0,0.0,1.0,1.0,scenario_9,0.141,37.1230
1.0,1.0,1.0,1.0,1.0,1.0,1.0,1.0,1.0,0.0,1.0,1.0,1.0,0.0,1.0,1.0,scenario_10,0.093,42.303
\end{lstlisting}

\subsection{Risk-Informed Multiple Knapsack Problem Optimization}
\label{subsec:RImkp}

Model formulation:\\

\begin{equation}\label{rimkp_obja}
\mathop{\max}_{x,y} \sum _{ \omega  \in  \Omega }q^{ \omega } \sum _{i \in I} \sum _{m \in M}a_{i}^{ \omega }x_{im}^{ \omega }
\end{equation}

\begin{equation}\label{rimkp_objb}
~~~~~~~~~~~~s.t.~~~~~y_{ii^{'}}+y_{i^{'}i} \geq 1,~ i<i^{'}\text{, i, }i^{'} \in I
\end{equation}

\begin{equation}\label{rimkp_objc}
~~~~~~~~\sum_{m=1}^{M} x_{im}^\omega \geq \sum_{m=1}^{M} x_{i'm}^\omega + y_{ii'} -1,~ i \neq i^{'}\text{, i, }i^{'} \in I,  \omega  \in  \Omega
\end{equation}

Constraint~(\ref{rimkp_objb}) indicates that either project $i$  is higher priority
than project  $i^{'}$  or vice versa, and further allows both (i.e., a tie).
Constraint~(\ref{rimkp_objc}) indicates that if project  $i$  is higher priority than
project  $i^{'}$ ($y_{ii^{'}}=1$), then if we select the lower priority project
\textit{for some unit} then we must also select the higher priority project;
if  $y_{ii^{'}}=0$  or if  $\sum_{m=1}^{M} x_{i'm}^\omega$  then the constraint is vacuous.\par

\begin{equation}\label{rimkp_objd}
 \sum _{i \in I}^{} c_{i}^{ \omega }x_{im}^{ \omega }~  \leq  b_{m}^{ \omega },~ m \in M,  \omega  \in  \Omega
\end{equation}

Constraint~(\ref{rimkp_objd}) requires that we be within budget
for each unit, and under each scenario.

\begin{equation}\label{rimkp_obje}
\sum_{m\in M} x_{im}^{ \omega } \leq 1,~ i \in I, \omega  \in  \Omega
\end{equation}

Constraint~(\ref{rimkp_obje}) simultaneously ensures that we select project $i$
only for one unit.

Example LOGOS input XML:
\begin{lstlisting}[style=XML]
  <Sets>
    <investments>
      1,2,3,4,5,6,7,8,9,10
    </investments>
    <capitals>
      unit_1, unit_2
    </capitals>
  </Sets>

  <Parameters>
    <net_present_values index="investments">
      78, 35, 89, 36, 94, 75, 74, 79, 80, 16
    </net_present_values>
    <costs index="investments">
      18, 9, 23, 20, 59, 61, 70, 75, 76, 30
    </costs>
    <available_capitals index="capitals">
      103, 156
    </available_capitals>
  </Parameters>

  <Uncertainties>
    <available_capitals>
      <totalScenarios>2</totalScenarios>
      <probabilities>
        0.3 0.7
      </probabilities>
      <scenarios>
        103, 156,
        103, 156
      </scenarios>
    </available_capitals>
    <net_present_values>
      <totalScenarios>2</totalScenarios>
      <probabilities>
        0.3, 0.7
      </probabilities>
      <scenarios>
        78, 35, 89, 36, 94, 75, 74, 79, 80, 16,
        78, 35, 89, 36, 94, 75, 74, 79, 80, 16
      </scenarios>
    </net_present_values>
  </Uncertainties>

  <Settings>
    <solver>glpk</solver>
    <sense>maximize</sense>
    <problem_type>MultipleKnapsack</problem_type>
  </Settings>
</Logos>
\end{lstlisting}

Example LOGOS output CSV:
\begin{lstlisting}[basicstyle=\tiny,language=python]
"('1', 'unit_1')","('1', 'unit_2')","('10', 'unit_1')","('10', 'unit_2')","('2', 'unit_1')","('2', 'unit_2')","('3', 'unit_1')",
"('3', 'unit_2')","('4', 'unit_1')","('4', 'unit_2')","('5', 'unit_1')","('5', 'unit_2')","('6', 'unit_1')","('6', 'unit_2')",
"('7', 'unit_1')","('7', 'unit_2')","('8', 'unit_1')","('8', 'unit_2')","('9', 'unit_1')","('9', 'unit_2')",
ScenarioName,ProbabilityWeight,MaxNPV
1.0,0.0,0.0,0.0,0.0,0.0,1.0,0.0,0.0,1.0,0.0,1.0,1.0,0.0,0.0,0.0,0.0,0.0,0.0,1.0,scenario_1,0.09,452.0
1.0,0.0,0.0,0.0,0.0,0.0,1.0,0.0,0.0,1.0,0.0,1.0,1.0,0.0,0.0,0.0,0.0,0.0,0.0,1.0,scenario_2,0.21,452.0
1.0,0.0,0.0,0.0,0.0,0.0,1.0,0.0,0.0,1.0,0.0,1.0,1.0,0.0,0.0,0.0,0.0,0.0,0.0,1.0,scenario_3,0.21,452.0
0.0,1.0,0.0,0.0,0.0,0.0,1.0,0.0,1.0,0.0,1.0,0.0,0.0,1.0,0.0,0.0,0.0,0.0,0.0,1.0,scenario_4,0.49,452.0
\end{lstlisting}

\subsection{Risk-Informed Multiple-Choice Multi-Dimensional Knapsack Problem Optimization}
\label{subsec:RImckp}

Model formulation:\\

\begin{equation}\label{stoc_obja}
\mathop{\max}_{x,y} \sum _{ \omega  \in  \Omega }^{}q^{ \omega } \sum _{i \in I}^{} \sum _{j \in J_{i}}^{}a_{ij}^{ \omega }x_{ij}^{ \omega }
\end{equation}

\begin{equation}\label{stoc_objb}
~~~~~~~~~~~~s.t.~~~~~y_{ii^{'}}+y_{i^{'}i} \geq 1,~ i<i^{'}\text{, i, }i^{'} \in I
\end{equation}

\begin{equation}\label{stoc_objc}
~~~~~~~~\sum_{j=1}^{J_i} x_{ij}^\omega \geq \sum_{j=1}^{J_i} x_{i'j}^\omega + y_{ii'} -1,~ i \neq i^{'}\text{, i, }i^{'} \in I,  \omega  \in  \Omega
\end{equation}

Constraint~(\ref{stoc_objb}) indicates that either project $i$  is higher priority
than project  $i^{'}$  or vice versa, and further allows both (i.e., a tie).
Constraint~(\ref{stoc_objc}) indicates that if project  $i$  is higher priority than
project  $i^{'}$  $y_{ii^{'}}=1$  then if we select the lower priority project
\textit{under some option} then we must also select the higher priority project;
if  $y_{ii^{'}}=0$  or if  $\sum_{j=1}^{J_i} x_{i'j}^\omega$  then the constraint is vacuous.\par

\begin{equation}\label{stoc_objd}
 \sum _{i \in I}^{} \sum _{j \in J_{i}}^{}\text{~ c}_{ijkt}^{ \omega }x_{ij}^{ \omega }~  \leq  b_{kt}^{ \omega },~ k \in K, t \in T,  \omega  \in  \Omega
\end{equation}

Constraint~(\ref{stoc_objd}) requires that we be within budget in each time period,
for each resource type, and under each scenario.

\begin{equation}\label{stoc_obje}
\sum_{j\in J_i} x_{ij}^{ \omega } \leq 1,~ i \in I, \omega  \in  \Omega
\end{equation}

Constraint~(\ref{stoc_obje}) simultaneously ensures that we select project  $i$
via at most one option. We note that this illustrates the situation in which we
must include the {\it Do Nothing}  option among the alternatives for optional projects.\par

Example LOGOS input XML:
\begin{lstlisting}[style=XML]
<Logos>
  <Sets>
    <investments>
      1, 2, 3, 4, 5, 6, 7, 8, 9, 10, 11, 12, 13, 14, 15, 16, 17
    </investments>
    <options index='investments'>
      1;
      1;
      1;
      1,2,3;
      1,2,3,4;
      1,2,3,4,5,6,7;
      1;
      1;
      1;
      1;
      1;
      1;
      1;
      1;
      1;
      1;
      1
    </options>
  </Sets>

  <Parameters>
    <net_present_values index='options'>
      2.046
      2.679
      2.489
      2.61
      2.313
      1.02
      3.013
      2.55
      3.351
      3.423
      3.781
      2.525
      2.169
      2.267
      2.747
      4.309
      6.452
      2.849
      7.945
      2.538
      1.761
      3.002
      3.449
      2.865
      3.999
      2.283
      0.9
      8.608
    </net_present_values>
    <costs index='options'>
      36538462
      83849038
      4615385
      2788461538
      2692307692
      5480769231
      1634615385
      2981730768
      7211538462
      9038461538
      649038462
      650000000
      216346154
      212500000
      3076923077
      3942307692
      1144230769
      675721154
      1442307692
      99711538
      4807692
      123076923
      138461538
      86538462
      108653846
      75092404
      6413462
      147932692
    </costs>
    <available_capitals>
      15E9
    </available_capitals>
  </Parameters>

  <Uncertainties>
    <available_capitals>
      <totalScenarios>3</totalScenarios>
      <probabilities>
        0.2,0.6,0.2
      </probabilities>
      <scenarios>
        5E9,10E9,15E9
      </scenarios>
    </available_capitals>
  </Uncertainties>

  <Settings>
    <solver>cbc</solver>
    <solverOptions>
      <threads>1</threads>
      <StochSolver>EF</StochSolver>
    </solverOptions>
    <sense>maximize</sense>
    <problem_type>mckp</problem_type>
  </Settings>
</Logos>
\end{lstlisting}

Example LOGOS output CSV:
\begin{lstlisting}[basicstyle=\tiny,language=python]
"('1', '1')","('10', '1')","('11', '1')","('12', '1')","('13', '1')","('14', '1')","('15', '1')","('16', '1')",
"('17', '1')","('2', '1')","('3', '1')","('4', '1')","('4', '2')","('4', '3')","('5', '1')","('5', '2')","('5', '3')",
"('5', '4')","('6', '1')","('6', '2')","('6', '3')","('6', '4')","('6', '5')","('6', '6')","('6', '7')","('7', '1')",
"('8', '1')","('9', '1')",ScenarioName,ProbabilityWeight,MaxNPV
1.0,1.0,1.0,1.0,1.0,1.0,1.0,1.0,1.0,1.0,1.0,0.0,0.0,0.0,0.0,0.0,0.0,0.0,0.0,0.0,0.0,0.0,0.0,0.0,1.0,1.0,1.0,1.0,scenario_1,0.2,53.865
1.0,1.0,1.0,1.0,1.0,1.0,1.0,1.0,1.0,1.0,1.0,1.0,0.0,0.0,1.0,0.0,0.0,0.0,0.0,0.0,0.0,0.0,0.0,0.0,1.0,1.0,1.0,1.0,scenario_2,0.6,59.488
1.0,1.0,1.0,1.0,1.0,1.0,1.0,1.0,1.0,1.0,1.0,1.0,0.0,0.0,0.0,0.0,1.0,0.0,0.0,0.0,0.0,0.0,0.0,0.0,1.0,1.0,1.0,1.0,scenario_3,0.2,59.826
\end{lstlisting}
