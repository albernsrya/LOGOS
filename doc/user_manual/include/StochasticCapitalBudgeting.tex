\section{Prioritizing Project Selection to Hedge Against Uncertainty}
\label{sec:StochasticCapitalBudgeting}

One limitation of traditional optimization models for capital budgeting is that
they do not account for risk/uncertainty in profit and cost streams associated
with individual projects, they do not account for risk in resource availability
in future years. Projects can incur cost over-runs, especially when projects are
large, performed infrequently, and when there is risk regarding technical viability,
external contractors, and/or suppliers of requisite parts and materials.
Occasionally, projects are performed ahead of schedule and with cost savings.
Planned budgets for capital improvements can be cut and key personnel may be
lost. Or, there may be surprise windfalls in budgets for maintenance activities
due to decreased costs for “unplanned” maintenance. In these cases, how should
we resolve capital budgeting when we have risk forecasts for costs, profits and
budgets? One approach is to re-solve the models described in the previous
section when refined forecasts for these parameters become available. However,
it is not always practical to fully revise a project portfolio whenever better
forecasts become available.

In order to prioritize the project selection with risk forecast for these
parameters, the two-stage stochastic optimization model~\cite{PrioritizingProjectSelection} is employed to provide
priority lists to decision-makers to support better risk-informed decisions.
Its inputs include those described in above sections for different variant of
the capital budgeting problem, except that a probabilistic description of the
uncertain parameters is integrated in the optimization process. The two-stage
stochastic optimization model forms a priority list as its first- stage decision
and then forms a corresponding project portfolio for each scenario as its
second-stage decision. When forming the optimal second-stage project portfolio
under a specific scenario, the stochastic optimization model ensures that the
portfolio is consistent with the first-stage prioritization; i.e., a project can
be selected only if all high-priority projects are also selected. Thus, the
portfolios of projects corresponding to different scenarios are nested.

The notation and formulation of the risk-informed models are as follows:

\[
\begin{array}{ll}
%%%%%%%%%%%%%% INDICES AND SET %%%%%%%%%%%%%%%%
\multicolumn{2}{l}{\mbox{\em Indexes and sets:} } \\
t \in T  & \mbox{time periods (years)} \\
i,i^{'} \in I  & \mbox{candidate projects} \\
j \in J_{i}	& \mbox{options for selecting project $i$} \\
i^{'},j^{'} \in IJ_{ij} & \mbox{piggybacking situations} \\
k \in K	& \mbox{types of resources} \\
\omega \in \Omega & \mbox{scenarios}\\
\\
%%%%%%%%%%%%%% DATA %%%%%%%%%%%%%%%%
\multicolumn{2}{l}{\mbox{\em Data:}} \\
a_{i}^{\omega} & \mbox{reward of selecting project $i$ under scenario $\omega$}  \\
a_{ij}^{\omega} & \mbox{reward of selecting project $i$ via option $j$ under scenario $\omega$}  \\
b^{\omega} & \mbox{available budget under scenario $\omega$}\\
b_{k}^{\omega} & \mbox{available budget for a resource of type $k$ under scenario $\omega$}\\
b_{t}^{\omega} & \mbox{available budget in year $t$ under scenario $\omega$}\\
b_{m}^{\omega} & \mbox{available budget for unit $m$ under scenario $\omega$}\\
b_{kt}^{\omega} & \mbox{available budget for a resource of type $k$ in year $t$ under scenario $\omega$}\\
c_{i}^{\omega} & \mbox{cost of investment $i$ under scenario $\omega$} \\
c_{ik}^{\omega} & \mbox{consumption of resource of type $k$} \\
& \mbox{if project $i$ is selected under scenario $\omega$}\\
c_{ijt}^{\omega} & \mbox{consumption of resource in year $t$ } \\
& \mbox{if project $i$ is performed via option $j$ under scenario $\omega$}\\
c_{ijkt}^{\omega} & \mbox{consumption of resource of type $k$ in year $t$ } \\
& \mbox{if project $i$ is performed via option $j$ under scenario $\omega$}\\
q^{\omega} & \mbox{probability of scenario $\omega$}\\
\\
%%%%%%%%%%%%%% DECISION VARS %%%%%%%%%%%%%%%%
\multicolumn{2}{l}{\mbox{\em Decision variables:}}  \\
x_{i}^{\omega} & \mbox{1 if project $i$ is selected under scenario $\omega$; 0 otherwise} \hspace*{4.0in}\\
x_{im}^{\omega} & \mbox{1 if project $i$ is selected for unit $m$ under scenario $\omega$; 0 otherwise} \hspace*{4.0in}\\
x_{ij}^{\omega} & \mbox{1 if project $i$ is selected via option $j$ under scenario $\omega$; 0 otherwise} \hspace*{4.0in}\\
y_{ii^{'}} & \mbox{1 if project $i$ has no lower priority than project $i^{'}$; 0 otherwise} \hspace*{4.0in}\\
\end{array}
\]

\subsection{Risk-Informed Single Knapsack Problem Optimization}
\label{subsec:RIskp}

\subsubsection{Risk-Informed Simple Knapsack Problem}
\vst \noi {\em Formulation:}
\begin{subequations}\label{RISimpleKP}
\begin{eqnarray}
&\dst \max_{x} &  \dst \sum_{\omega\in\Omega} q^\omega \dst \sum_{i \in I} a_{i}^\omega x_{i}^\omega \\
& s. t. & \sum_{i \in I} c_{i}^\omega x_{i}^\omega \leq b^\omega \label{stoc_cona}\\
& & y_{ii'} + y_{i'i} \geq 1, i<i' \label{stoc_conb} \\
& & x_{i}^\omega \geq x_{i'}^\omega + y_{ii'}-1, i\neq i' \label{stoc_conc}
\end{eqnarray}
\end{subequations}
For simplicity, in what follows the variable $y_{ii'}=1$ means that project $i$
is higher priority than $i'$ even though the variable definition allows for ties,
i.e., the projects being the same priority.
Constraint~(\ref{stoc_cona}) requires that we be within budget under each scenarios.
Constraint~(\ref{stoc_conb}) indicates that either project $i$ is higher priority
than project $i'$ or vice versa, and further allows both (i.e., a tie).
Constraint~(\ref{stoc_conc}) indicates that if project $i$ is higher priority than
project $i'$ ($y_{ii'}=1$), then if we select the lower priority project then we
must also select the higher priority project; if $y_{ii'}=0$ or if $x_{i'}^\omega$
then the constraint is vacuous.
In order to handle the risk in the capital budgeting problems, the entity
\xmlNode{Uncertainties} (see section~\ref{subsec:Uncertainties}) is used to specify
different scenarios of input parameters.

Example LOGOS input XML:
\begin{lstlisting}[style=XML]
<Logos>
  <Sets>
    <investments>
      1,2,3,4,5,6,7,8,9,10
    </investments>
  </Sets>

  <Parameters>
    <net_present_values index="investments">
      18,20,17,19,25,21,27,23,25,24
    </net_present_values>
    <costs index="investments">
      1,3,7,4,8,9,6,10,2,5
    </costs>
    <available_capitals>
      15
    </available_capitals>
  </Parameters>

  <Uncertainties>
    <available_capitals>
      <totalScenarios>10</totalScenarios>
      <probabilities>
        0.012, 0.019, 0.032, 0.052, 0.086, 0.142, 0.235, 0.188, 0.141, 0.093
      </probabilities>
      <scenarios>
        11, 12, 13, 14, 15, 16, 17, 18, 19, 20
      </scenarios>
    </available_capitals>
    <net_present_values>
      <totalScenarios>2</totalScenarios>
      <probabilities>
        0.3, 0.7
      </probabilities>
      <scenarios>
        18,20,17,19,25,21,27,23,25,24,
        18,20,17,19,25,21,27,23,25,24
      </scenarios>
    </net_present_values>
  </Uncertainties>
  ...
</Logos>
\end{lstlisting}

When run this case, LOGOS would generate a CSV (comma separated values) file that
contains solutions for the optimization problem, i.e. values of decision variables
and maximum profit (MaxNPV is used to describe the maximum profit). The header of
this CSV file contains the indices listed under \xmlNode{investments} which are
used as indices for decision variables, and the objective variable \textbf{MaxNPV}.
The data provides the values for both decision variables and objective variable.

Example LOGOS output CSV:
\begin{lstlisting}[language=python]
1,2,3,4,5,6,7,8,9,10,MaxNPV
0.0,1.0,0.0,0.0,0.0,0.0,1.0,0.0,1.0,1.0,96.0
\end{lstlisting}

In this case, the projects \textbf{2, 7, 9, 10} are selected under all scenarios
with maximum profit \textbf{96.0}.


\subsubsection{Risk-Informed Multi-Dimensional Knapsack Problem}



\subsection{Risk-Informed Multiple Knapsack Problem Optimization}
\label{subsec:RImkp}


\subsection{Risk-Informed Multiple-Choice Multi-Dimensional Knapsack Problem Optimization}
\label{subsec:RImckp}














The deterministic capital budgeting model previously developed allows for multiple options in how
we select a project. For example, we might select a project via Plan A, Plan B, Plan C, or not
select the project at all. In addition, the deterministic model allows for multiple types of
resources (e.g., capital budgets and O\&M budgets), and further allows for piggybacking constraints.
Our stochastic capital budgeting model illustrates the ideas of prioritization without the
additional features of multiple types of resources, piggybacking, and multiple options for selecting
each project.  The former two features integrate with the prioritization scheme in a straightforward
way, as we will describe below. The latter-most feature proves to have subtle interactions with
the notion of prioritization, and we discuss that in some detail in this section. The model
sketched here is new and, to our knowledge, has not appeared in the literature. Even though the
notation has been sketched above, we develop the full model here so that this section is
self-contained, given that it specifies our “full” mathematical model for stochastic capital
budgeting.



Model formulation:\\

\begin{equation}\label{stoc_obja}
\mathop{\max}_{s,x,y,z} \sum _{ \omega  \in  \Omega }^{}q^{ \omega } \sum _{i \in I}^{} \sum _{j \in J_{i}}^{}a_{ij}^{ \omega }x_{ij}^{ \omega }
\end{equation}

\begin{equation}\label{stoc_objb}
~~~~~~~~~~~~s.t.~~~~~s_{ii^{'}}+s_{i^{'}i} \geq 1,~ i<i^{'}\text{, i, }i^{'} \in I
\end{equation}

\begin{equation}\label{stoc_objc}
~~~~~~~~y_{i}^{ \omega } \geq y_{i^{'}}^{ \omega }+s_{ii^{'}}-1,~ i \neq i^{'}\text{, i, }i^{'} \in I,  \omega  \in  \Omega
\end{equation}

For simplicity, in what follows we will say that variable $s_{ii^{'}}=1$  means that project
$i$  is higher priority than $i^{'}$  even though the variable definition allows for ties,
i.e., the projects being the same priority. Constraint~(\ref{stoc_objb}) indicates that either
project  $i$  is higher priority than project  $i^{'}$  or vice versa, and further allows both
(i.e., a tie). Constraint~(\ref{stoc_objc}) indicates that if project  $i$  is higher priority
than project  $i^{'}$  $s_{ii^{'}}=1$  then if we select the lower priority project
\textit{under some option} then we must also select the higher priority project; if  $s_{ii^{'}}=0$
or if  $y_{i^{'}}^{\omega}=0$  then the constraint is vacuous.\par

\begin{equation}\label{stoc_objd}
 \sum _{i \in I}^{} \sum _{j \in J_{i}}^{}\text{~ c}_{ijkt}^{ \omega }x_{ij}^{ \omega }~  \leq  b_{kt}^{ \omega },~ k \in K, t \in T,  \omega  \in  \Omega
\end{equation}

Constraint~(\ref{stoc_objd}) requires that we be within budget in each time period, for each resource type, and under each scenario.

\begin{equation}\label{stoc_obje}
 \sum _{j \in J_{i}}^{}~~x_{ij}^{ \omega }= y_{i}^{ \omega },~~~~~~~~~~i \in I,~ \omega  \in  \Omega
\end{equation}

\begin{equation}\label{stoc_objf}
y_{i}^{ \omega }=1,~ i \in I,  \omega  \in  \Omega
\end{equation}

Constraint~(\ref{stoc_obje}) defines binary variable  $y_{i}^{ \omega }$  and simultaneously ensures that we select project  $i$  via at most one option. Constraint~(\ref{stoc_objf}) ensures that we select all must-do projects. We note that this illustrates the alternative to the situation in which we must include the {\it Do Nothing}  option among the alternatives for optional projects.\par

\begin{equation}\label{stoc_objg}
x_{i^{'}j^{'}}^{ \omega }  \leq \text{~ x}_{ij}^{ \omega },~~ \left( i^{'},j^{'} \right)  \in IJ_{ij} , j \in J_{i},i \in I
\end{equation}

Constraint~(\ref{stoc_objg}) captures piggybacking conditions.\par

\begin{equation}\label{stoc_objh}
s_{ii^{'}}+s_{i^{'}i} \leq 1,~ i<i^{'}\text{, i, }i^{'} \in I
\end{equation}

\begin{equation}\label{stoc_obji}
s_{ii^{'}~}+s_{i^{'}i^{''}}+s_{i^{''}i} \leq 2,~ i \neq i^{'},i^{'} \neq  i^{''},i^{''} \neq i,~i,~i^{'},i^{''} \in I
\end{equation}

Constraints~(\ref{stoc_objh})-(\ref{stoc_obji}) require that we produce a total ordering of the
projects rather than allowing for ties. If we remove constraints~(\ref{stoc_objh})-(\ref{stoc_obji})
then it will not change the optimal NPV that we obtain, but including the constraints can facilitate
easier parsing of the solutions.\par

\begin{equation}\label{stoc_objj}
x_{i^{'}j}^{ \omega }+s_{ii^{'}~}-1 \leq  \sum _{\begin{array}{c}
	j^{'} \in J_{i}\\
	j^{'} \leq j~\\
	\end{array}}^{}x_{ij}^{ \omega }~,~~i \neq i^{'}~i,~i^{'} \in I,~j \in J_{i^{'}},~ \omega  \in  \Omega
\end{equation}


Constraint~(\ref{stoc_objj}) is a type of consistency constraint with respect to the notion of
options; the constraint matters only when project  $i$  is higher priority than project
$i^{'}$ $s_{ii^{'}}=1$ . In this case, if we select Plan A for the lower priority project
then we must select plan A for the higher priority project. If we select Plan B for the lower
priority project, then we can select Plan A or Plan B for the higher priority project. And,
if we select Plan C for the lower priority project then we can select Plan A, B, or C for the
higher priority project. Inclusion of constraint~(\ref{stoc_objj}) is optional and reflects
how the decision maker prefers to interpret the notion of priorities.

\begin{equation}\label{stoc_objk}
 \sum _{ \omega  \in  \Omega }^{}\text{~ x}_{ij}^{ \omega } \leq   \vert  \Omega  \vert ~z_{ij} ,i \in I, j \in J_{i}
\end{equation}

\begin{equation}\label{stoc_objl}
 \sum _{j \in J_{i}}^{}z_{ij} \leq ~1,~  i \in I
\end{equation}

\begin{equation}\label{stoc_objm}
s_{ii^{'}},x_{ij}^{ \omega },y_{i}^{ \omega },z_{ij} \in { 0,1 } , i \neq i^{'},i,i^{'} \in I, j \in J_{i} ,  \omega  \in  \Omega
\end{equation}
%\begin{comment}

Constraints~(\ref{stoc_objk}) and (\ref{stoc_objl}) taken together indicate that, for each
project separately, we cannot mix use of Plans A, B, and C across different scenarios.
