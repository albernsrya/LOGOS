\section{Deterministic Capital Budgeting}
\label{sec:DeterministicCapitalBudgeting}

We consider a capital budgeting problem for a nuclear generation station, with possible extension to
a larger fleet of plants.
Due to limited resources, we can only select a subset from a list of
several candidate capital projects.
Our goal is to maximize overall NPV associated with the
selected subset.
In doing so, we must respect resource limits and capture key structural and
stochastic dependencies of the system, although in this section we start with the simpler
deterministic case, ignoring randomness.
Example projects include upgrading a steam turbine,
refurbishing or replacing a set of reactor coolant pumps, and replacing a set of feed-water heaters.

\[
\begin{array}{ll}
%%%%%%%%%%%%%% INDICES AND SET %%%%%%%%%%%%%%%%
\multicolumn{2}{l}{\mbox{\em Indexes and sets:} } \\
t \in T  & \mbox{time periods (years)} \\
i \in I  & \mbox{investment candidate projects} \\
j \in J_{i}	& \mbox{options for selecting project $i$} \\%, e.g., initiate project $i$ in year $t$ or $t+2$ and in a standard (three year) or in an expedited (two year) manner} \\
% i^{'},j^{'} \in IJ_{ij} & \mbox{piggybacking situations} \\%, i.e., option $j^{'}$ for project $i^{'}$ can be selected only if option $j$ is selected for project $i$} \\
k \in K	& \mbox{types of resources} \\%, e.g., capital funds, O\&M funds, labor-hours, time during outage} \\
\\
%%%%%%%%%%%%%% DATA %%%%%%%%%%%%%%%%
\multicolumn{2}{l}{\mbox{\em Data:}} \\
a_{ij} & \mbox{reward (revenue minus financial cost) of selecting project $i$ via option $j$}  \\
b_{kt} & \mbox{available budget for a resource of type $k$ in year $t$}\\
c_{ijkt}  & \mbox{consumption of resource of type $k$ in year $t$ if project $i$ is performed via option $j$} \\
\\
%%%%%%%%%%%%%% DECISION VARS %%%%%%%%%%%%%%%%
\multicolumn{2}{l}{\mbox{\em Decision variables:}}  \\
x_{ij} & \mbox{1 if project $i$ is selected via option $j$; 0 otherwise} \hspace*{4.0in}\\
\end{array}
\]

\vst \noi {\em Formulation:}
\begin{subequations}\label{model-deter}
\begin{eqnarray}
&\dst \max_{x} &  \dst \sum_{i \in I, j \in J_{i}} a_{ij} x_{ij} \label{obj_deter} \\
& s. t.  & \sum_{j \in J_{i}} x_{ij} = 1,   i \in I \\
& & \sum_{i \in I, j \in J_{i}} c_{ijkt} x_{ij} \leq b_{kt}, k \in K, t \in T \\
& & x_{ij} \in \{0,1\}, j \in J_{i}, i \in I.
\end{eqnarray}
\end{subequations}

The decision variables, $x_{ij}$, indicate whether we choose to do project i by means j. Restated,
if $x_{ij}=1$, then we recommend doing project $i$ via option $j$; taken together, these decision
variables produce both a portfolio of selected projects and a schedule for performing those projects
over time.  The set of available options, $j \in J_i$, can explicitly include the “do-nothing” option,
and the first constraint ensures that we choose exactly one option from the available set for each
project, including the possibility of selecting the do-nothing option. Even if we select the
do-nothing option for a project, it induces an NPV, $a_{ij}$, which may be negative, representing
growing O\&M costs, losses in plant efficiency, etc. The second structural constraint ensures that
the budget of each resource $k$ is respected in each year $t$. The objective function
includes the NPV for each project-option pair, $a_{ij}$, and the correct NPV is selected by
the $0-1$ decision variable, $x_{ij}$.

% The third structural constraint
% captures piggybacking situations in which option $j^{'}$ for project $i^{'}$ (which may have cheaper
% costs) may be selected only if project-option pair $(i,j)$ is also selected.

We note that certain projects must be done (e.g., for safety and or regulatory
reasons). This can be handled within the mathematical formulation just given, without introducing
additional constructs. The set $J_i$ typically includes a do-nothing option for each project,
but when project $i$ must be done, we simply do not include the do-nothing option. Mathematically,
one alternative is to disclude an explicit do-nothing option, replace the first structural
constraint with an inequality, and add an additional set of must-do projects with an equality
constraint. Both options are mathematically equivalent and simply represent a choice to be made by
the analyst. In LOGOS, we use \xmlNode{mandatory} and \xmlNode{nonSelection} to
handle these conditions. \xmlNode{mandatory} is used to specify the must-do projects,
while \xmlNode{nonSelection} is used to activate the do-nothing option.

\nb If a project is listed under \xmlNode{mandatory}, the do-nothing option is not allowed
for this project. In addition, handling the do-nothing option implicitly leads to the NPV being
calculated relative to that of the do-nothing option.

The objective of capital budgeting is to find the right combination of binary decisions for
every investment so that overall profit is maximized. The output is a
collection of projects to be carried out, and we refer to this selected collection of projects
as a ``project portfolio''. However, as is frequently the case for capital budgeting with
NPP applications, in practice, several optional constraints, such as resources/liabilities,
dependencies/synergies, options, time windows for every investment, etc., have to be
fulfilled. This leads to a various variations of the knapsack problem.
In the following subsection, we will present several different variants of the above
capital budgeting problem (i.e., variants of the knapsack problem).

\subsection{Single Knapsack Problem Optimization}
\label{subsec:skp}

\subsubsection{Simple Knapsack Problem}
The simple knapsack problem (KP) for capital budgeting can be defined as follows:
we are given an instance of the capital budgeting problem with investment set $I$,
consisting of $I$ investments $i$ with profit $a_i$ (e.g. NPV), cost $c_i$,
and available budget $b$. The objective is to select a subset of $I$ such
that the total profit of the selected investments is maximized and the total cost does
not exceed $b$. Alternatively, KP can be formulated as a solution of the following
linear integer programming formulation:

\begin{subequations}\label{simpleKP}
\begin{eqnarray}
&\dst \max_{x} &  \dst \sum_{i \in I} a_{i} x_{i} \\
& s. t. & \sum_{i \in I} c_{i} x_{i} \leq b\\
& & x_{i} \in \{0,1\}, i \in I.
\end{eqnarray}
\end{subequations}

Example LOGOS input XML:
\begin{lstlisting}[style=XML]
<Logos>
  <Sets>
    <investments>
      1,2,3,4,5,6,7,8,9,10
    </investments>
  </Sets>

  <Parameters>
    <net_present_values index="investments">
      18,20,17,19,25,21,27,23,25,24
    </net_present_values>
    <costs index="investments">
      1,3,7,4,8,9,6,10,2,5
    </costs>
    <available_capitals>
      15
    </available_capitals>
  </Parameters>

  <Settings>
    <solver>cbc</solver>
    <sense>maximize</sense>
  </Settings>
</Logos>
\end{lstlisting}

When running this case, LOGOS would generate a CSV (comma separated values) file
containing solutions for the optimization problem (i.e. values of decision variables
and maximum profit [MaxNPV is used to describe the maximum profit]). The header of
this CSV file contains the indices listed under \xmlNode{investments}
used as indices for decision variables and the objective variable \textbf{MaxNPV}.
The data provides the values for both decision variables and the objective variable.

Example LOGOS output CSV:
\begin{lstlisting}[language=python]
1,2,3,4,5,6,7,8,9,10,MaxNPV
1.0,1.0,0.0,1.0,0.0,0.0,0.0,0.0,1.0,1.0,106.0
\end{lstlisting}

In this case, projects \textbf{1, 2, 4, 9, and 10} are selected with a maximum
profit of 106.0.

\subsubsection{Bounded Knapsack Problem}
In the capital budgeting problem described above, it may be the case that not all
investments/projects are different from each other. For example, in practice,
there may be given a number ($n_i$) of identical pumps/valves to be replaced. In this
case, the number of decision variables is equal to the number of different
investments, rather than the total number of investments. The constraint for the
decision variables becomes:
\begin{equation}
0\leq x_i \leq n_i, i\in N
\end{equation}
The resulting problem is called the bounded knapsack problem (BKP) and is formally defined as:

\begin{subequations}\label{boundedKP}
\begin{eqnarray}
&\dst \max_{x} &  \dst \sum_{i \in I} a_{i} x_{i} \\
& s. t. & \sum_{i \in I} c_{i} x_{i} \leq b\\
& & x_{i} \in \{0,n_i\}, i \in I.
\end{eqnarray}
\end{subequations}

Example LOGOS input XML:
\begin{lstlisting}[style=XML]
<Logos>
  <Sets>
    <investments>
      1, 2, 3, 4, 5, 6, 7, 8, 9, 10, 11, 12, 13,
      14, 15, 16, 17, 18, 19, 20, 21, 22
    </investments>
  </Sets>

  <Parameters>
    <net_present_values index="investments">
      150,35,200,60,60,45,60,40,30,10,70,
      30,15,10,40,70,75,80,20,12,50,10
    </net_present_values>
    <costs index="investments">
      9,13,153,50,15,68,27,39,23,52,11,32,
      24,48,73,42,43,22,7,18,4,30
    </costs>
    <available_capitals>
      400
    </available_capitals>
  </Parameters>

  <Settings>
    <lowerBounds>
      0, 0, 0, 0, 0, 0, 0, 0, 0, 0, 0,
      0, 0, 0, 0, 0, 0, 0, 0, 0, 0, 0
    </lowerBounds>
    <upperBounds>
      1,1,2,2,2,3,3,3,1,3,1,
      1,2,2,1,1,1,1,1,2,1,2
    </upperBounds>
    <solver>glpk</solver>
    <sense>maximize</sense>
  </Settings>
</Logos>
\end{lstlisting}

Example LOGOS output CSV:
\begin{lstlisting}[language=python]
1,2,...,21,22,MaxNPV
1.0,1.0,...,1.0,0.0,1010.0
\end{lstlisting}

\subsubsection{Multi-Dimensional Knapsack Problem: DKP}
Moving in a different direction, we now take into account not only the cost constraint, but also
the limited commitment of critical resources, including: (i) capital cost, (ii) O\&M
costs, (iii) time and labor-hours during a planned outage, and (iv) personnel,
installation and maintenance equipment, workspaces, etc.. Denoting the cost of every
investment by $c_{ik}$ for each resource $k$ and introducing the corresponding limited resource
$b_k$, we can formulate the capital budgeting problem as a multi-dimensional
or D-dimensional knapsack problem formally defined by:

\begin{subequations}\label{boundedKP}
\begin{eqnarray}
&\dst \max_{x} &  \dst \sum_{i \in I} a_{i} x_{i} \\
& s. t. & \sum_{i \in I} c_{ik} x_{i} \leq b_k, k\in K\\
& & x_{i} \in \{0,1\}, i \in I.
\end{eqnarray}
\end{subequations}

Where the limited resources set is denoted by $K$, consisting of $k$ “colors” of money
within capital costs, O\&M costs, personnel availability, etc.
Another example is provided if the plant has multi-year investments. Consider a DKP problem in
which the costs of each investment and the available capitals vary according to time
period $t$. By defining $c_{it}$ as the cost of investment $i$ at time period $i$,
and $b_t$ as the available capital at time period $t$, we get:

\begin{subequations}\label{DKP}
\begin{eqnarray}
&\dst \max_{x} &  \dst \sum_{i \in I} a_{i} x_{i} \\
& s. t. & \sum_{i \in I} c_{it} x_{i} \leq b_t, t\in T\\
& & x_{i} \in \{0,1\}, i \in I.
\end{eqnarray}
\end{subequations}

Example LOGOS input XML:
\begin{lstlisting}[style=XML]
<Logos>
  <Sets>
    <investments>
      1,2,3,4,5,6,7,8,9
    </investments>
    <time_periods>
      1,2,3,4,5
    </time_periods>
  </Sets>

  <Parameters>
    <net_present_values index="investments">
      2.315,0.824,22.459,60.589,0.667,5.173,4.003,0.582,0.122
    </net_present_values>
    <costs index="investments, time_periods">
      0.219,0.257,0.085,0.0,0.0,
      0.0,0.0,0.122,0.103,0.013,
      5.044,1.839,0.0,0.0,0.0,
      6.74,6.134,10.442,0.0,0.0,
      0.425,0.0,0.0,0.0,0.0,
      2.125,2.122,0.0,0.0,0.0,
      2.387,0.19,0.012,2.383,0.192,
      0.0,0.95,0.0,0.0,0.0,
      0.03,0.03,0.688,0.0,0.0
    </costs>
    <available_capitals index="time_periods">
      0.665,4.712,9.642,3.458,1.683
    </available_capitals>
  </Parameters>

  <Settings>
    <solver>glpk</solver>
    <sense>maximize</sense>
  </Settings>
</Logos>
\end{lstlisting}

Example LOGOS output CSV:
\begin{lstlisting}[language=python]
1,2,3,4,5,6,7,8,9,MaxNPV
1.0,1.0,0.0,0.0,1.0,0.0,0.0,1.0,0.0,4.388
\end{lstlisting}


\subsection{Multiple Knapsack Problem Optimization}
\label{subsec:mkp}
Another interesting variant of the capital budgeting problem arises if we consider
maintenance for multiple units in a NPP in parallel, i.e. it has to be decided whether
to accept a particular replacement and, in the positive case, in which unit to conduct
the corresponding replacement. This can be formulated by
introducing a binary decision variable for every maintenance-unit combination.
If there are $I$ investments (investment set $I$) on the list of maintenance requests and $m$
units (unit set $M$) available, we use binary variables:

\begin{equation}
x_{im} \in \{0,1\}, i\in I, m\in M
\end{equation}
The resulting problem is called the multiple knapsack problem (MKP),
and the mathematical formulation is given by:

\begin{subequations}\label{MKP}
\begin{eqnarray}
&\dst \max_{x} &  \dst \sum_{m \in M} \sum_{i \in I} a_{i} x_{im} \\
& s. t. & \sum_{i \in I} c_{i} x_{im} \leq b_m, m \in M\\
& & \sum_{m \in M} x_{im} \leq 1 \\
& & x_{im} \in \{0,1\}, i \in I, m \in M.
\end{eqnarray}
\end{subequations}

Example LOGOS input XML:
\begin{lstlisting}[style=XML]
<Logos>
  <Sets>
    <investments>
      1,2,3,4,5,6,7,8,9,10
    </investments>
    <capitals>
      unit_1, unit_2
    </capitals>
  </Sets>

  <Parameters>
    <net_present_values index="investments">
      78, 35, 89, 36, 94, 75, 74, 79, 80, 16
    </net_present_values>
    <costs index="investments">
      18, 9, 23, 20, 59, 61, 70, 75, 76, 30
    </costs>
    <available_capitals index="capitals">
      103, 156
    </available_capitals>
  </Parameters>

  <Settings>
    <solver>cbc</solver>
    <sense>maximize</sense>
    <problem_type>MultipleKnapsack</problem_type>
  </Settings>
</Logos>
\end{lstlisting}

Example LOGOS output CSV:
\begin{lstlisting}[language=python]
1,2,3,4,5,6,7,8,9,10,capitals,MaxNPV
0.0,0.0,1.0,1.0,1.0,0.0,0.0,0.0,0.0,0.0,unit_1,452.0
1.0,0.0,0.0,0.0,0.0,1.0,0.0,0.0,1.0,0.0,unit_2,452.0
\end{lstlisting}


\subsection{Multiple-Choice Multi-Dimensional Knapsack Problem Optimization}
\label{subsec:mckp}
Another quite different variant of the capital budgeting problem appears if there may
be multiple ways to carry out each investment/project. Each investment $i$, however, exists
in a number of options in which the j-th option has cost $c_{ij}$ and profit
value $a_{ij}$. This problem can be expressed as the multiple-choice knapsack problem
(MCKP). Assume $J_i$ is the set of different options for investment $i$. Using the
decision variables $x_{ij}$ to denote whether option $j$ was chosen from the set $J_i$,
the mathematical formulation of MCKP is given by:

\begin{subequations}\label{MKP}
\begin{eqnarray}
&\dst \max_{x} &  \dst \sum_{i \in I, j \in J_{i}} a_{ij} x_{ij} \\
& s. t.  & \sum_{j \in J_{i}} x_{ij} = 1,   i \in I \\
& & \sum_{i \in I, j \in J_{i}} c_{ij} x_{ij} \leq b \\
& & x_{ij} \in \{0,1\}, j \in J_{i}, i \in I.
\end{eqnarray}
\end{subequations}

Considering the limited resources and multi-year investments, the MCKP may be extended
to a D-dimensional MCKP problem (D-MCKP). For example, a project may be performed over
a three-year period--say, years `$t$, $t+1$, $t+2$'--or the start of the project could
instead be two years hence, changing the equation to `$t+2$, $t+3$, $t+4$'.
Alternatively, at increased cost and benefit, it may be possible to complete
the project in two years: `$t$, $t+1$', or `$t+2$, $t+3$'. When selecting a project to
uprate plant capacity, we may have the options of increasing it by 3\% or 6\%.
In these cases, the problem can be expressed as the D-MCKP.
This problem is formally defined as follows:

\begin{subequations}\label{model-deter}
\begin{eqnarray}
&\dst \max_{x} &  \dst \sum_{i \in I, j \in J_{i}} a_{ij} x_{ij} \label{obj_deter} \\
& s. t.  & \sum_{j \in J_{i}} x_{ij} = 1,   i \in I \\
& & \sum_{i \in I, j \in J_{i}} c_{ijt} x_{ij} \leq b_{t}, t \in T \\
& & x_{ij} \in \{0,1\}, j \in J_{i}, i \in I.
\end{eqnarray}
\end{subequations}

or in the same as the problem described at the beginning of this chapter
(see Section~\ref{sec:DeterministicCapitalBudgeting}):

\begin{subequations}\label{model-deter}
\begin{eqnarray}
&\dst \max_{x} &  \dst \sum_{i \in I, j \in J_{i}} a_{ij} x_{ij} \label{obj_deter} \\
& s. t.  & \sum_{j \in J_{i}} x_{ij} = 1,   i \in I \\
& & \sum_{i \in I, j \in J_{i}} c_{ijkt} x_{ij} \leq b_{kt}, k \in K, t \in T \\
& & x_{ij} \in \{0,1\}, j \in J_{i}, i \in I.
\end{eqnarray}
\end{subequations}

Example LOGOS input XML:
\begin{lstlisting}[style=XML]
<Logos>
  <Sets>
    <investments>
      1, 2, 3, 4, 5, 6, 7, 8, 9, 10, 11, 12, 13, 14, 15, 16, 17
    </investments>
    <options index='investments'>
      1;
      1;
      1;
      1,2,3;
      1,2,3,4;
      1,2,3,4,5,6,7;
      1;
      1;
      1;
      1;
      1;
      1;
      1;
      1;
      1;
      1;
      1
    </options>
  </Sets>

  <Parameters>
    <net_present_values index='options'>
      2.046 2.679 2.489 2.61 2.313 1.02 3.013 2.55 3.351 3.423 3.781 2.525
      2.169 2.267 2.747 4.309 6.452 2.849 7.945 2.538 1.761 3.002 3.449
      2.865 3.999 2.283 0.9 8.608
    </net_present_values>
    <costs index='options'>
      36538462
      83849038
      4615385
      2788461538
      2692307692
      5480769231
      1634615385
      2981730768
      7211538462
      9038461538
      649038462
      650000000
      216346154
      212500000
      3076923077
      3942307692
      1144230769
      675721154
      1442307692
      99711538
      4807692
      123076923
      138461538
      86538462
      108653846
      75092404
      6413462
      147932692
    </costs>
    <available_capitals>
      15E9
    </available_capitals>
  </Parameters>

  <Settings>
    <solver>cbc</solver>
    <sense>maximize</sense>
    <problem_type>mckp</problem_type>
  </Settings>
</Logos>
\end{lstlisting}

Example LOGOS output CSV:
\begin{lstlisting}[language=python]
1__1,2__1,3__1,4__1,4__2,4__3,...,17__1,MaxNPV
1.0,1.0,1.0,1.0,0.0,0.0,...,1.0,59.82600000000001
\end{lstlisting}

In the output file, the names \textbf{``investmentsIndex"\_\_``optionsIndex"} are used to
specify the decision variable. For example, \textbf{1\_\_1} indicates that investment \textbf{1}
with option \textbf{1} is selected.
