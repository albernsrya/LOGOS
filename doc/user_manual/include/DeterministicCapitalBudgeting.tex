\section{Deterministic Capital Budgeting}
\label{sec:DeterministicCapitalBudgeting}

We consider a capital budgeting problem for a nuclear generation station, with possible extension to
a larger fleet of plants. Due to limited resources, we can only select a subset from a list of
several candidate capital projects. Our goal is to maximize overall NPV associated with the
selected subset. In doing so, we must respect resource limits and capture key structural and
stochastic dependencies of the system, although in this section we start with the simpler
deterministic case, ignoring randomness.  Example projects include upgrading a steam turbine,
refurbishing or replacing a set of reactor coolant pumps, and replacing a set of feed-water heaters.

\[
\begin{array}{ll}
%%%%%%%%%%%%%% INDICES AND SET %%%%%%%%%%%%%%%%
\multicolumn{2}{l}{\mbox{\em Indexes and sets:} } \\
t \in T  & \mbox{time periods (years)} \\
i \in I  & \mbox{candidate projects} \\
j \in J_{i}	& \mbox{options for selecting project $i$} \\%, e.g., initiate project $i$ in year $t$ or $t+2$ and in a standard (three year) or in an expedited (two year) manner} \\
i^{'},j^{'} \in IJ_{ij} & \mbox{piggybacking situations} \\%, i.e., option $j^{'}$ for project $i^{'}$ can be selected only if option $j$ is selected for project $i$} \\
k \in K	& \mbox{types of resources} \\%, e.g., capital funds, O\&M funds, labor-hours, time during outage} \\
\\
%%%%%%%%%%%%%% DATA %%%%%%%%%%%%%%%%
\multicolumn{2}{l}{\mbox{\em Data:}} \\
a_{ij} & \mbox{reward (revenue less financial cost) of selecting project $i$ via option $j$}  \\
b_{kt} & \mbox{available budget for a resource of type $k$ in year $t$}\\
c_{ijkt}  & \mbox{consumption of resource of type $k$ in year $t$ if project $i$ is performed via option $j$} \\
\\
%%%%%%%%%%%%%% DECISION VARS %%%%%%%%%%%%%%%%
\multicolumn{2}{l}{\mbox{\em Decision variables:}}  \\
x_{ij} & \mbox{1 if project $i$ is selected via option $j$; 0 otherwise} \hspace*{4.0in}\\
\end{array}
\]

\vst \noi {\em Formulation:}
\begin{subequations}\label{model-deter}
\begin{eqnarray}
&\dst \max_{x} &  \dst \sum_{i \in I, j \in J_{i}} a_{ij} x_{ij} \label{obj_deter} \\
& s. t.  & \sum_{j \in J_{i}} x_{ij} = 1,   i \in I \\
& & \sum_{i \in I, j \in J_{i}} c_{ijkt} x_{ij} \leq b_{kt}, k \in K, t \in T \\
& & x_{ij} \in \{0,1\}, j \in J_{i}, i \in I.
\end{eqnarray}
\end{subequations}

The decision variables, $x_{ij}$, indicate whether we choose to do project i by means j. Restated,
if $x_{ij}=1$, then we recommend doing project $i$ via option $j$, and taken together these decision
variables produce both a portfolio of selected projects and a schedule for performing those projects
over time.  The set of available options, $j \in J_i$, can explicitly include the “do-nothing” option,
and the first constraint ensures that we choose exactly one option from the available set for each
project, including the possibility of selecting the do-nothing option. Even if we select the
do-nothing option for a project, it induces an NPV, $a_{ij}$, which may be negative, representing
growing O\&M costs, losses in plant efficiency, etc. The second structural constraint ensures that
the budget of each resource $k$ is respected in each year $t$. The third structural constraint
captures piggybacking situations in which option $j^{'}$ for project $i^{'}$ (which may have cheaper
costs) may be selected only if project-option pair $(i,j)$ is also selected. The objective function
includes the NPV for each project-option pair, $a_{ij}$, and the correct NPV is selected by
the $0-1$ decision variable, $x_{ij}$.

We note that sometimes there are projects that must be done, e.g., for safety and or regulatory
reasons. This can be handled within the mathematical formulation just given, without introducing
additional constructs. The set $J_i$ typically includes a do-nothing option for each project,
but when project $i$ must be done, we simply do not include the do-nothing option. Mathematically,
an alternative is to not include an explicit do-nothing option, to replace the first structural
constraint with an inequality, and to add an additional set of must-do projects with an equality
constraint. Both options are mathematically equivalent and simply represent a choice to be made by
the analyst. The mathematical model is simpler using our approach in this section, but the set of
input data is a bit more complicated. Handling the do-nothing option implicitly leads to the
opposite situation. To illustrate ideas, we include an explicit do-nothing option in this
section, and later we illustrate the implicit alternative. We do note that the choice that we
make affects the NPV calculations. The scheme described in this section requires NPV calculations
that are absolute and includes an NPV value for the do-nothing case. In contrast, if the do-nothing
option is implicit then the NPV of plan A, plan B, etc., should be calculated relative to that of
the do-nothing option.

The deterministic model can be repeatedly solved by changing the input values,
$a_{ij}$, $b_{kt}$, $c_{ijkt}$. This will allow for what-if sensitivity analysis to identify the
crucial drivers behind the optimal project selection decision. Monte Carlo simulation permits a
powerful variant of this approach in which we model $a_{ij}$, $b_{kt}$, $c_{ijkt}$ as random
variables, sample from their distributions, and perform a form of uncertainty quantification
in terms of the resulting distributions governing the binary decisions selected, $x_{ij}$, and
the overall NPV of the selected portfolio.
