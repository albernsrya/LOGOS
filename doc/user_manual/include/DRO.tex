\section{Distributionally Robust Optimization}
\label{sec:DROCapitalBudgeting}

We consider another risk-averse decision making approach
using distributionally robust optimization (DRO). To this end, we begin with
a nominal stochastic optimization problem:

\begin{equation}
\max_{s\in S} \sum_{\sigma\in \Sigma} q^\sigma f(s,\xi^\sigma)
\end{equation}

In our context, the nominal model is a stochastic capital budgeting problem
in which we maximize expected NPV, and we could have
$f(s,\xi^\sigma)=NPV(s,\xi^\sigma)$. In this context ``$s\in S$'' simply
indicates the constraints that a prioritized solution must satisfy, wherein
we prioritize project selection subject to uncertainty in costs and the NPV
of each project as well as uncertainty in resource availability. The goal is
to prioritize so as to maximize expected NPV, assuming that nominal distribution,
specified by the probability mass function $q^\sigma, \sigma \in \Sigma$.

We suppose that $\xi$ is a discrete random variable with finite sample space $\Omega$,
so that $\xi^\omega, \omega \in \Omega$, enumerate all possible realizations.
We further suppose that we only have observations of
$\xi^\sigma, \sigma \in \Sigma \subset \Omega$,
i.e., possibly a strict subset, which may arise in a data-driven setting. In such a
data-driven setting we could have, for example, probability mass $q^\sigma = 1/|\Sigma|$
for all $\sigma \in \Sigma$.
A DRO variant of this nominal stochastic optimization model is then given by:

\begin{equation}
\max_{s\in S} \min_{p\in P} \sum_{\omega\in \Omega} p^\omega f(s,\xi^\omega)
\end{equation}

Here, we may view this DRO model as playing a ``game'' against nature. First, we select
$s\in S$, and then knowing $s$, nature selects a worst-case probability distribution,
$p\in P$, to minimize the expected NPV, which we seek to maximize. We will make precise
what we mean by the Distributional Uncertainty Set (DUS), denoted by $P$, below, but
for the moment it is enough to think of the set as representing a neighborhood of
probability distributions centered on the given probability mass function, $q$,
with the radius of the neighborhood specified by parameter $\varepsilon$. If
$\varepsilon = 0$ then the DRO model reduces to the nominal model. If $\varepsilon$ is very
large then nature will select the single worst-case scenario, e.g., the scenario
with lowest budgets, highest costs, and lowest NPVs. This is too conservative to be useful
(i.e. if we are living in this world, it is very likely that the plant would be
uneconomical no matter what decisions are made). However, with moderate values of
$\varepsilon$ we obtain solutions that hedge against deviations from $q$ without being
excessively conservative.

Importantly, we do not view nature as malevolent, despite occasional evidence to the contrary.
Rather, we use ``$min_{p\in P}$'' to combat over-adapting our solution to a specific assumption
about the probability distribution. In this sense, DRO plays the role of a ``regularizer'' to
combat over-fitting that is analogous to regularizers used in high-dimensional statistics
and statistical machine learning. Our approach to DRO requires further mathematical and
intuitive development before we can analyze solutions to the DRO problem.

\subsection{Defining a Distributional Uncertainty Set via the Wasserstein Distance}
\label{subsec:WassersteinDistance}
By the constraints denoted by $p\in P$ we require that nature select a probability
mass function, $p$, that is ``close'' to the nominal or empirical data-driven distribution
$q$. We define the DUS as follows:

\begin{equation}
P=\{p: D(p,q)\le \varepsilon, \sum_{\omega\in \Omega} p^\omega = 1, p^\omega \ge 0, \omega \in \Omega \}
\end{equation}

where $D(p, q)$ is the distance between nature’s choice, $p$, and the nominal data-driven distribution,
$q$. As indicated above, the radius parameter $\varepsilon$ governs the latitude we give nature,
which in turn governs the degree of conservatism that we face when selecting decision $s\in S$.

There are multiple ways to measure the ``distance'', $D(p, q)$, between two probability distributions,
which include the Kolmogorov-Smirnov distance, Kullback-Leibler divergence, chi-squared distances,
total variation, and more general $\psi$-divergences. The Wasserstein distance, which is based on
the idea of optimal transport, is a particularly useful way to measure such a distance in the
context of distributionally robust optimization. For a distribution with known finite support,
the Wasserstein distance, $D(p, q)$, between a given distribution, $q^\sigma, \sigma \in \Sigma$,
and a given candidate robust distribution, $p^\omega, \omega \in \Omega$, is provided by the
optimal value of the transportation problem:

\begin{subequations}\label{WassersteinEq}
\begin{eqnarray}
& & D(q, p) = \min_{z} \sum_{\sigma \in \Sigma, \omega \in \Omega} d_{\sigma, \omega} z_{\sigma, \omega} \\
& & s.t. \sum_{\omega \in \Omega} z_{\sigma, \omega} = q^\sigma, \sigma \in \Sigma \\
& & \sum_{\sigma \in \Sigma} z_{\sigma, \omega} = p^\omega, \omega \in \Omega \\
& & z_{\sigma, \omega} \ge 0, \sigma \in \Sigma, \omega \in \Omega
\end{eqnarray}
\end{subequations}

The intuition behind this measure concerns the magnitude of probability mass, $q^\sigma$,
that must be transported distance $d_{\sigma, \omega}$ from vector $\xi^\sigma$ to vector $\xi^\omega$
via variable $z_{\sigma, \omega}$. At one extreme case, if the two sample spaces and probability
mass functions coincide, i.e., $\Omega = \Sigma$ and $p^\omega = q^\omega$, and $d_{\omega, \omega} = 0$
for all $\omega \in \Omega$ then $D(p, q) = 0$.

To fully specify $D(p, q)$ we must define $d_{\sigma, \omega}= dist(\xi^\sigma, \xi^\omega)$.
To do so we can select $dist(\cdot, \cdot)$, for example, to be the two-norm distance, or a more
general $\eta$-norm distance, between the vectors, $\xi^\sigma$ and $\xi^\omega$,
i.e., $dist(\cdot, \cdot) = ||\xi^\sigma - \xi^\omega||_\eta$.

With the Wasserstein distance, if we are given distribution, $q$, we can then define:

\begin{equation}
P=\{p: D(p,q)\le \varepsilon, \sum_{\omega\in \Omega} p^\omega = 1, p^\omega \ge 0, \omega \in \Omega \}
\end{equation}

for a given radius $\varepsilon$. Here, we think of $P$ as a ball, or neighborhood, of
probability distributions centered on $q$, where the neighborhood has radius $\varepsilon$.
With $\Sigma \subset \Omega$, if $\varepsilon = 0$ then $P$ is the singleton ${q}$, and
larger values of $\varepsilon$ lead to increasingly large neighborhoods. In the context
of robust optimization, if $\varepsilon = 0$ then we will simply be solving the nominal
stochastic optimization model, and as $\varepsilon$ grows large we will consider
increasingly conservative models.

We can now represent the set $P$ via the following so-called extended-variable set of constraints:

\begin{subequations}\label{WassersteinConstraints}
\begin{eqnarray}
& & \sum_{\sigma \in \Sigma, \omega \in \Omega} d_{\sigma, \omega} z_{\sigma, \omega} \le \varepsilon \\
& & \sum_{\omega \in \Omega} z_{\sigma, \omega} = q^\sigma, \sigma \in \Sigma \\
& & \sum_{\sigma \in \Sigma} z_{\sigma, \omega} = p^\omega, \omega \in \Omega \\
& & z_{\sigma, \omega} \ge 0, \sigma \in \Sigma, \omega \in \Omega
\end{eqnarray}
\end{subequations}

\subsection{Towards a Computationally Tractable Reformulation}
\label{subsec:TractableReformulation}
Due to the max min construct in the DRO model, the model is not amenable to direct solution
via optimization software. So, we reformulate the model to facilitate computation.
For the moment let $s\in S$ be fixed so that $f(s, \xi^\omega)$ is just a known numerical
value for each $\omega \in \Omega$. Then, nature’s problem may be written:

\begin{subequations}
\begin{eqnarray}
& & \min_{p, z} \sum_{\omega \in \Omega} p^\omega f(s, \xi^\omega) \\
& & s.t. \sum_{\sigma \in \Sigma, \omega \in \Omega} d_{\sigma, \omega} z_{\sigma, \omega} \le \varepsilon : [-\gamma] \\
& & \sum_{\omega \in \Omega} z_{\sigma, \omega} = q^\sigma, \sigma \in \Sigma : [\nu^\sigma]\\
& & \sum_{\sigma \in \Sigma} z_{\sigma, \omega} = p^\omega, \omega \in \Omega : [\beta^\omega]\\
& & z_{\sigma, \omega} \ge 0, \sigma \in \Sigma, \omega \in \Omega
\end{eqnarray}
\end{subequations}

Here, $\gamma, \nu^\sigma, \beta^\omega$ denote dual variables.
In this model, nature optimizes over $z$ and over $p$ to select a worst-case distribution
within radius $\varepsilon$ of $q$.

Taking the dual of the linear DRO program, and substituting out the dual variable
$\beta^\omega= -f(s, \xi^\omega)$, we obtain the following:

\begin{subequations}\label{dualProlem}
\begin{eqnarray}
& & \max_{\gamma, \nu} -\gamma \varepsilon + \sum_{\sigma\in \Sigma} \nu^\sigma q^\sigma \\
& & s.t. -\gamma d_{\sigma, \omega} + \nu^\sigma \le f(s, \xi^\omega), \sigma \in \Sigma, \omega \in \Omega \\
& & \gamma \ge 0
\end{eqnarray}
\end{subequations}

To gain intuition regarding model, consider two extreme cases, $\varepsilon = 0$ and $\varepsilon = \infty$.
If $\varepsilon = 0$ then there is no penalty in the objective function for allowing $\gamma$
to grow large. As $gamma$ grows large, the constraint becomes vacuous for all $\sigma \ne \omega$;
however, for $\sigma = \omega$ we have $d_{\sigma, \sigma} = ||\xi^\sigma - \xi^\omega|| = 0$,
and hence the constraint reduces to $\nu^\sigma \le f(s, \xi^\omega)$, and coupled with the objective
function the optimal value reduces to $\sum_\sigma q^\sigma f(s, \xi^\sigma) = \sum_\omega p^\omega f(s, \xi^\omega)$,
i.e., it reduces to the objective function value of the nominal stochastic optimization model,
as it must with $\varepsilon = 0$.

In the other extreme, as $\varepsilon$ grows sufficiently large we must have $\gamma = 0$
to avoid a huge penalty in the objective function. Thus, for each $\sigma \in \Sigma$,
the constraint reduces to $\nu^\sigma \le f(s, \xi^\omega)$, i.e.,
the objective function reduces to:

\begin{equation}
\sum_\sigma \min_\omega f(s, \xi^\omega) q^\sigma = \min_\omega f(s, \xi^\omega) \sum_\sigma q^\sigma = \min_\omega f(s, \xi^\omega)
\end{equation}

Again, this matches what it must: if $\varepsilon = \infty$ then nature has enough latitude
to place probability one on the single worst-case scenario.

Specializing $s\in S$ to be the constraints for prioritization, and specializing
$f(s,\xi^\omega)$ to define the NPV under scenario $\omega\in \Omega$, the DRO variant of the
stochastic capital budgeting problem is as follows:

\begin{subequations}\label{fullDRO}
\begin{eqnarray}
& & \max_{x, y, \gamma, \nu} -\gamma \varepsilon + \sum_{\sigma\in \Sigma} \nu^\sigma q^\sigma \\
& & s.t. -\gamma d_{\sigma, \omega} + \nu^\sigma \le \sum _{i \in I}^{} \sum _{j \in J_{i}}^{}a_{ij}^{ \omega }x_{ij}^{ \omega }, \sigma \in \Sigma, \omega \in \Omega \\
& & y_{ii^{'}}+y_{i^{'}i} \geq 1,~ i<i^{'}\text{, i, }i^{'} \in I \\
& & \sum_{j=1}^{J_i} x_{ij}^\omega \geq \sum_{j=1}^{J_i} x_{i'j}^\omega + y_{ii'} -1,~ i \neq i^{'}\text{, i, }i^{'} \in I,  \omega  \in  \Omega \\
& & \sum _{i \in I}^{} \sum _{j \in J_{i}}^{}\text{~ c}_{ijkt}^{ \omega }x_{ij}^{ \omega }~  \leq  b_{kt}^{ \omega },~ k \in K, t \in T,  \omega  \in  \Omega \\
& & \sum_{j\in J_i} x_{ij}^{ \omega } \leq 1,~ i \in I, \omega  \in  \Omega \\
& & \gamma \ge 0
\end{eqnarray}
\end{subequations}

\subsection{LOGOS Settings for DRO Problems}
\label{subsec:DROSettings}
DRO approach is an extension for stochastic optimization approach discussed in
Section~\ref{sec:StochasticCapitalBudgeting}. Both of them share the same input
structures except the \xmlNode{Settings} block. In both cases, the user need to
specify a collection of scenarios via \xmlNode{Uncertainties} block. The
\xmlNode{problem\_type} within \xmlNode{Settings} block is used to select the
type of DRO problems. The currently available DRO problem types are:
\xmlString{droskp}, \xmlString{dromkp}, and \xmlString{dromckp}. The user can
use \xmlNode{radius\_ambiguity} to control the Wasserstein distance for DRO problem.

Example LOGOS input XML for DRO:
\begin{lstlisting}[style=XML]
<Settings>
<Logos>
  <solver>cbc</solver>
  <solverOptions>
    <StochSolver>EF</StochSolver>
    <!-- epsilon radius -->
    <radius_ambiguity>0.1</radius_ambiguity>
  </solverOptions>
  <sense>maximize</sense>
  <problem_type>droskp</problem_type>
</Settings>
</Logos>
\end{lstlisting}

\subsection{DRO for Single Knapsack Problem}
\label{subsec:DRO_SKP}

\vst \noi {\em Model Formulation:}
\begin{subequations}\label{DROSimpleKP}
\begin{eqnarray}
& & \max_{x, y} \max_{\gamma, \nu} (-\gamma \varepsilon + \sum_{\sigma \in \Sigma} \nu^\sigma q^\sigma)  \\
& &  -\gamma d_{\sigma, \omega} + \nu^\sigma \le \sum _{i \in I} a_{i}^{ \omega }x_{i}^{ \omega }, \sigma \in \Sigma, \omega \in \Omega \\
& & \sum_{i \in I} c_{i}^\omega x_{i}^\omega \leq b^\omega \\
& & y_{ii'} + y_{i'i} \geq 1, i<i'  \\
& & x_{i}^\omega \geq x_{i'}^\omega + y_{ii'}-1, i\neq i' \\
& & x_{i}^\omega, y_{ii'}^\omega \in {0, 1} \\
& & \gamma \ge 0
\end{eqnarray}
\end{subequations}

See next section~\ref{subsec:DRO_DKP} for the example, since multi-dimensional Knapsack problem
is just a simple extension of single Knapsack problem, and both of them belong to the same
\xmlNode{problem\_type}: \xmlString{droskp}.


\subsection{DRO for Multi-Dimensional Knapsack Problem}
\label{subsec:DRO_DKP}

\vst \noi {\em Model Formulation:}
\begin{subequations}\label{DROMultiDKP}
\begin{eqnarray}
& & \max_{x, y} \max_{\gamma, \nu} (-\gamma \varepsilon + \sum_{\sigma \in \Sigma} \nu^\sigma q^\sigma)  \\
& &  -\gamma d_{\sigma, \omega} + \nu^\sigma \le \sum _{i \in I} a_{i}^{ \omega }x_{i}^{ \omega }, \sigma \in \Sigma, \omega \in \Omega \\
& & \sum_{i \in I} c_{it}^\omega x_{i}^\omega \leq b_{t}^\omega, t\in T \\
& & y_{ii'} + y_{i'i} \geq 1, i<i'  \\
& & x_{i}^\omega \geq x_{i'}^\omega + y_{ii'}-1, i\neq i' \\
& & x_{i}^\omega, y_{ii'}^\omega \in {0, 1} \\
& & \gamma \ge 0
\end{eqnarray}
\end{subequations}

Example LOGOS input XML:
\begin{lstlisting}[style=XML]
<Logos>
  <Sets>
    <investments>
      1,2,3,4,5,6,7,8,9,10,11,12,13,14,15,16
    </investments>
    <time_periods>
      1,2,3,4,5
    </time_periods>
  </Sets>

  <Parameters>
    <net_present_values index="investments">
      2.315,0.824,22.459,60.589,0.667,5.173,4.003,0.582,0.122,
      -2.870,-0.102,-0.278,-0.322,-3.996,-0.246,-20.155
    </net_present_values>
    <costs index="investments, time_periods">
      0.219,0.257,0.085,0.0,0.0,
      0.0,0.0,0.122,0.103,0.013,
      5.044,1.839,0.0,0.0,0.0,
      6.74,6.134,10.442,0.0,0.0,
      0.425,0.0,0.0,0.0,0.0,
      2.125,2.122,0.0,0.0,0.0,
      2.387,0.19,0.012,2.383,0.192,
      0.0,0.95,0.0,0.0,0.0,
      0.03,0.03,0.688,0.0,0.0,
      0,0.2,0.763,0.739,2.539,
      0.081,0.032,0,0,0,
      0.3,0,0,0,0,
      0.347,0,0,0,0,
      4.025,0.297,0,0,0,
      0.095,0.095,0.095,0,0,
      5.487,5.664,0.5,6.803,6.778
    </costs>
    <available_capitals index="time_periods">
      18,18,18,18,18
    </available_capitals>
  </Parameters>

  <Uncertainties>
    <available_capitals>
      <totalScenarios>10</totalScenarios>
      <probabilities>
        0.012, 0.019, 0.032, 0.052, 0.086, 0.142, 0.235, 0.188, 0.141, 0.093
      </probabilities>
      <!--
        scenarios is ordered by numberScenarios * parametersIndex,
        the number of scenarios is determined by
        the number of elements in <probabilities>, for this case:
        numberScenarios * time_periods = 10 * 5
      -->
      <scenarios>
        11, 11, 11, 11, 11,
        12, 12, 12, 12, 12,
        13, 13, 13, 13, 13,
        14, 14, 14, 14, 14,
        15, 15, 15, 15, 15,
        16, 16, 16, 16, 16,
        17, 17, 17, 17, 17,
        18, 18, 18, 18, 18,
        19, 19, 19, 19, 19,
        20, 20, 20, 20, 20
      </scenarios>
    </available_capitals>
  </Uncertainties>

  <Settings>
    <mandatory>10,11,12,13,14,15,16</mandatory>
    <solver>cbc</solver>
    <solverOptions>
      <StochSolver>EF</StochSolver>
      <!-- epsilon radius -->
      <radius_ambiguity>0.1</radius_ambiguity>
    </solverOptions>
    <sense>maximize</sense>
    <problem_type>droskp</problem_type>
  </Settings>
</Logos>
\end{lstlisting}


\subsection{DRO for Multiple Knapsack Problem}
\label{subsec:DRO_MKP}

\vst \noi {\em Model Formulation:}
\begin{subequations}\label{DROMKP}
\begin{eqnarray}
& & \max_{x, y} \max_{\gamma, \nu} (-\gamma \varepsilon + \sum_{\sigma \in \Sigma} \nu^\sigma q^\sigma)  \\
& &  -\gamma d_{\sigma, \omega} + \nu^\sigma \le \sum _{i \in I} \sum _{m \in M}a_{i}^{ \omega }x_{im}^{ \omega }, \sigma \in \Sigma, \omega \in \Omega \\
& & \sum _{i \in I}^{} c_{i}^{ \omega }x_{im}^{ \omega }~  \leq  b_{m}^{ \omega },~ m \in M,  \omega  \in  \Omega \\
& & y_{ii'} + y_{i'i} \geq 1, i<i'  \\
& & \sum_{m=1}^{M} x_{im}^\omega \geq \sum_{m=1}^{M} x_{i'm}^\omega + y_{ii'} -1,~ i \neq i^{'}\text{, i, }i^{'} \in I,  \omega  \in  \Omega \\
& & x_{i,m}^\omega, y_{ii'}^\omega \in {0, 1} \\
& & \gamma \ge 0
\end{eqnarray}
\end{subequations}

Example LOGOS input XML:
\begin{lstlisting}[style=XML]
<Logos>
  <Sets>
    <investments>
      1,2,3,4,5,6,7,8,9,10
    </investments>
    <capitals>
      unit_1, unit_2
    </capitals>
  </Sets>

  <Parameters>
    <net_present_values index="investments">
      78, 35, 89, 36, 94, 75, 74, 79, 80, 16
    </net_present_values>
    <costs index="investments">
      18, 9, 23, 20, 59, 61, 70, 75, 76, 30
    </costs>
    <available_capitals index="capitals">
      103, 156
    </available_capitals>
  </Parameters>

  <Uncertainties>
    <available_capitals>
      <totalScenarios>10</totalScenarios>
      <probabilities>
        0.1 0.1 0.1 0.1 0.1 0.1 0.1 0.1 0.1 0.1
      </probabilities>
      <scenarios>
        101, 154,
        102, 155,
        103, 156,
        104, 157,
        105, 158,
        106, 159,
        107, 160,
        108, 161,
        109, 162,
        110, 163
      </scenarios>
    </available_capitals>
  </Uncertainties>

  <Settings>
    <solver>cbc</solver>
    <solverOptions>
      <StochSolver>EF</StochSolver>
      <!-- epsilon radius -->
      <radius_ambiguity>0.1</radius_ambiguity>
    </solverOptions>
    <sense>maximize</sense>
    <problem_type>dromkp</problem_type>
  </Settings>
</Logos>
\end{lstlisting}


\subsection{DRO for Multiple-Choice Knapsack Problem}
\label{subsec:DRO_MCKP}

\vst \noi {\em Model Formulation:}
\begin{subequations}\label{DROMCKP}
\begin{eqnarray}
& & \max_{x, y} \max_{\gamma, \nu} (-\gamma \varepsilon + \sum_{\sigma \in \Sigma} \nu^\sigma q^\sigma)  \\
& &  -\gamma d_{\sigma, \omega} + \nu^\sigma \le \sum _{i \in I}^{} \sum _{j \in J_{i}}^{}a_{ij}^{ \omega }x_{ij}^{ \omega }, \sigma \in \Sigma, \omega \in \Omega \\
& & \sum _{i \in I}^{} \sum _{j \in J_{i}}^{}\text{~ c}_{ijkt}^{ \omega }x_{ij}^{ \omega }~  \leq  b_{kt}^{ \omega },~ k \in K, t \in T,  \omega  \in  \Omega \\
& & y_{ii'} + y_{i'i} \geq 1, i<i'  \\
& & \sum_{j=1}^{J_i} x_{ij}^\omega \geq \sum_{j=1}^{J_i} x_{i'j}^\omega + y_{ii'} -1,~ i \neq i^{'}\text{, i, }i^{'} \in I,  \omega  \in  \Omega \\
& & x_{i,j}^\omega, y_{ii'}^\omega \in {0, 1} \\
& & \gamma \ge 0
\end{eqnarray}
\end{subequations}

Example LOGOS input XML:
\begin{lstlisting}[style=XML]
<Logos>
  <Sets>
    <investments>
      1, 2, 3, 4, 5, 6, 7, 8, 9, 10, 11, 12, 13, 14, 15, 16, 17
    </investments>
    <options index='investments'>
      1;
      1;
      1;
      1,2,3;
      1,2,3,4;
      1,2,3,4,5,6,7;
      1;
      1;
      1;
      1;
      1;
      1;
      1;
      1;
      1;
      1;
      1
    </options>
  </Sets>

  <Parameters>
    <net_present_values index='options'>
      2.046
      2.679
      2.489
      2.61
      2.313
      1.02
      3.013
      2.55
      3.351
      3.423
      3.781
      2.525
      2.169
      2.267
      2.747
      4.309
      6.452
      2.849
      7.945
      2.538
      1.761
      3.002
      3.449
      2.865
      3.999
      2.283
      0.9
      8.608
    </net_present_values>
    <costs index='options'>
      36538462
      83849038
      4615385
      2788461538
      2692307692
      5480769231
      1634615385
      2981730768
      7211538462
      9038461538
      649038462
      650000000
      216346154
      212500000
      3076923077
      3942307692
      1144230769
      675721154
      1442307692
      99711538
      4807692
      123076923
      138461538
      86538462
      108653846
      75092404
      6413462
      147932692
    </costs>
    <available_capitals>
      15E9
    </available_capitals>
  </Parameters>

  <Uncertainties>
    <available_capitals>
      <totalScenarios>3</totalScenarios>
      <probabilities>
        0.2,0.6,0.2
      </probabilities>
      <scenarios>
        5E9,10E9,15E9
      </scenarios>
    </available_capitals>
  </Uncertainties>

  <Settings>
    <solver>glpk</solver>
    <solverOptions>
      <StochSolver>EF</StochSolver>
      <!-- epsilon radius -->
      <radius_ambiguity>0.1</radius_ambiguity>
    </solverOptions>
    <sense>maximize</sense>
    <problem_type>dromckp</problem_type>
  </Settings>
</Logos>
\end{lstlisting}
