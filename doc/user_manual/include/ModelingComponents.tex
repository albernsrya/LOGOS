\section{Overview of Modeling Components}
\label{sec:ModelingComponents}

We consider a capital budgeting problem for a nuclear generation station, with possible
extension to a larger fleet of plants. Due to limited resources, we can only select a
subset from a number of candidate investment projects. Our goal is to maximize overall net
present value (NPV), or a variant of this objective when we incorporate uncertainty in
project cost–and project revenue–streams. In doing so, we must respect resource limits
and capture key structural and stochastic dependencies of the system. Example projects
include upgrading a steam turbine, refurbishing or replacing a set of reactor coolant pumps,
and replacing a set of feed-water heaters. Selecting an individual project has multiple
facets and implications.

\begin{itemize}
  \item \textbf{Rewards or Net Present Values}: Selecting a project can improve revenue, e.g.,
  upgrading a steam turbine may lead to an uprate in plant capacity resulting in larger
  revenue from selling power. Replacing a key system component can improve reliability,
  increasing revenue due to a reduction in forced outages and reducing operations and
  maintenance (O\&M) costs. Choosing to perform minimum maintenance versus refurbishing
  a component versus replacing and improving a system can produce “reward” streams over
  years which can be negative or positive depending on the selection. Parameter
  \xmlNode{net\_present\_values} is used to specify the rewards (See Section~\ref{subsec:Parameters}).

  \item \textbf{Resources and Liabilities}: Critical resources, including: (i) capital costs,
  (ii) O\&M costs, (iii) time and labor-hours during a planned outage, (iv) personnel,
  installation \& maintenance equipment, space, and more. Within these categories, resources
  can be placed in further subcategories, each with its own budget, due to a plant’s organizational
  structure so that there are multiple “colors” of money within capital costs, within O\&M costs,
  within personnel availability, etc. Set \xmlNode{resources} is used to specify the resources and
  liabilities (See Section~\ref{subsec:Sets}).

  \item \textbf{Costs}: Selecting a project in year $t$ induces multiple
  cost streams in year $t$ and in subsequent years, where we interpret “cost” broadly to
  include commitment of critical resources. Parameter \xmlNode{costs} is used to specify
  the costs (See Section~\ref{subsec:Parameters}).

  \item \textbf{Time Periods}: Multiple capital projects can compete for time that would
  limit project selection. Set \xmlNode{time\_periods} is used to provide indices for
  \textbf{costs} and \textbf{available capitals} (See Section~\ref{subsec:Sets}).

  \item \textbf{Options}: The goal of selecting a project is typically to improve or maintain
  a particular function that the plant performs, and there may be multiple ways to carry out
  the task. A project may be performed over a three-year period, say, years $t$,$t+1$, $t+2$, or the
  start of the project could instead be two years hence with project implementation over
  years $t+2$, $t+3$, $t+4$. Alternatively, at increased cost and increased benefit, it may be
  possible to complete the project in two years, $t$, $t+1$ or $t+2$, $t+3$. When selecting a project
  to uprate plant capacity, we may have two options that increase capacity by 3% or 6%.
  In all these cases, we can perform the project in at most one way, from a collection of
  multiple options. We represent this by cloning a “project” into multiple project-option pairs,
  and adding a constraint saying that we can select at most one of from this set of options.
  Sets \xmlNode{options} is used to provided indices for these multiple project-option pairs
  (See Section~\ref{subsec:Sets}).

  \item \textbf{Capitals}: If we consider a maintenance for multi-units NPP in parallel,
  i.e. it has to be decided whether to accept a particular replacement and in the positive
  case in which unit to conduct the corresponding replacement. In this case, set \xmlNode{capitals}
  is used to provided indices for these units (See Section~\ref{subsec:Sets}).

  \item \textbf{Available Capitals}: available budgets for resources/units. Parameter
  \xmlNode{available\_capitals} is used to specify the available capitals for different
  resources/captials at different $t$ (See Section~\ref{subsec:Parameters}).

  \item \textbf{Non-Selection}: Not selecting a project also has implications, inducing growth
  in O\&M costs in future years, a decrease in plant production, an increase in forced outages,
  and even risking a premature end to plant life. Thus, not selecting a project can be seen as
  one more “option” as to how a larger project is executed, expanding the list just discussed.
  Selection is of the “do nothing” option is reflected in both liability streams and reward
  streams. This can be activated via setting \xmlNode{nonSelection} to \xmlString{True}
  (See Section~\ref{subsec:Settings}).

  \item \textbf{Uncertainty}: One limitation of traditional optimization models for capital
  budgeting is that they do not account for uncertainty in reward and cost streams associated
  with individual projects, they do not account for uncertainty in resource availability in
  future years. Projects can incur cost over-runs, especially when projects are large, performed
  infrequently, and when there is uncertainty regarding technical viability, external contractors,
  and/or suppliers of requisite parts and materials. Occasionally, projects are performed ahead
  of schedule and with cost savings. Planned budgets for capital improvements can be cut and key
  personnel may be lost. Or, there may be surprise windfalls in budgets for maintenance activities
  due to decreased costs for “unplanned” maintenance. XML node \xmlNode{Uncertainties} is used
  to specify the uncertainties (See Section~\ref{subsec:Uncertainties}).

  %\item \textbf{Synergies}: Selecting a project may require replacing a structure, system, or
  %component (SSC) during a planned outage of the plant. Depending on the physical location of
  %an SSC in the plant and its relationship to other components, selecting one project may
  %reduce the cost of selecting another project (e.g., time or know-how required to implement
  %the project) if they are selected at the same time or close in proximity. For example, if
  %a plant has two units, selecting a project for one unit in a spring outage (e.g., replacement
  %of a condensate cooler and a set of feed-water heaters) may be followed by the same activity
  %in the fall outage in the second unit, at reduced cost.

  %\item \textbf{Planned Outage}: Nuclear power plants have planned outages at regular
  %intervals (e.g., every 18 months) often in the fall and spring to be well-prepared for winter
  %and summer peaks in load. While refueling only takes a fraction of a two-month (say) period
  %without power production, maintenance projects may be deferred until an outage. Moreover,
  %an outage can provide the only possible time period in which to carry out certain types of
  %projects. Because of lost revenue, an operator seeks to limit downtime. As a result, this
  %provides a special type of resource constraint limiting project selection due to multiple
  %projects competing for time, space, personnel, and equipment during an outage.

\end{itemize}

LOGOS consists of a collection of modeling entities/components that define different
aspects of the model, including \xmlNode{Sets}, \xmlNode{Parameters},
\xmlNode{Uncertainties} and \xmlNode{ExternalConstraints}. In addition, \xmlNode{Setting}
block specifies how the overall computation should be run.

%
\subsection{Sets}
\label{subsec:Sets}

This sub-section contains the information regarding the XML nodes used to define the
\xmlNode{Sets} of the optimization model that is being performed through LOGOS.
\xmlNode{Sets} specifies a collection of data, possibly including
numeric data (e.g. real or integer values) as well as symbolic data (e.g. strings)
that is typically used to specify the valid indices for an indexed components.
\nb numeric data provided in the \xmlNode{Sets} would be treated as strings.
\xmlNode{Sets} accepts the following additional sub-nodes:
\begin{itemize}
  \item \xmlNode{investments}, \xmlDesc{comma/space-separated string, required}, specifies
  the valid indices for investment projects.
  \item \xmlNode{capitals}, \xmlDesc{comma/space-separated string, optional},
  specifies the indices for NPP units.
  \item \xmlNode{time\_periods}, \xmlDesc{comma/space-separated string, optional},
  specifies the indices for time.
  \item \xmlNode{resources}, \xmlDesc{comma/space-separated string, optional},
  specifies indices for the resources and liabilities.
  \item \xmlNode{options}, \xmlDesc{comma/space-separated string, optional},
  specifies the indices for multiple project-option pairs.
  This sub-node accepts the following attribute:
  \begin{itemize}
    \item \xmlAttr{index}, \xmlDesc{string, required}, specifies the index dependence.
    Valid index is \xmlString{investments}.
  \end{itemize}
\end{itemize}

Example XML:
\begin{lstlisting}[style=XML]
<Sets>
  <investments>
      HPFeedwaterHeaterUpgrade
      PresurizerReplacement
      ...
      ReplaceMoistureSeparatorReheater
  </investments>
  <time_periods>year1 year2 year3 year4 year5</time_periods>
  <resources>CapitalFunds OandMFunds</resources>
  <options index='investments'>
    PlanA PlanB DoNothing;
    PlanA PlanB PlanC;
    ...
    PlanA PlanB PlanC DoNothing
  </options>
</Sets>
\end{lstlisting}


%
\subsection{Parameters}
\label{subsec:Parameters}
This sub-section contains the information regarding the XML nodes used to define the
\xmlNode{Parameters} of the optimization model that is being performed through LOGOS:
\begin{itemize}
  \item \xmlNode{net\_present\_values}, \xmlDesc{comma/space-separated string, required},
  specifies the NPVs for capital projects or project-option pairs. This node accepts the
  following optional attribute:
  \begin{itemize}
    \item \xmlAttr{index}, \xmlDesc{comma-separated string, optional},
    specifies the indices of this parameter, keywords should be predefined in \xmlNode{Sets}.
    Valid keywords are \xmlString{investments} and \xmlString{options}.
    \default{investments}
  \end{itemize}
  \item \xmlNode{costs}, \xmlDesc{comma/space-separated string, required},
  specifies the costs for capital projects or project-option pairs. This node accepts the
  following optional attribute:
  \begin{itemize}
    \item \xmlAttr{index}, \xmlDesc{comma-separated string, optional},
    specifies the indices of this parameter, keywords should be predefined in \xmlNode{Sets}.
    Valid keywords are \xmlString{investments} and \xmlString{investments, time\_periods},
    \xmlString{options}, \xmlString{options, resources}, \xmlString{options, time\_periods},
    \xmlString{options, resources, time\_periods}.
    \default{'investments'}
  \end{itemize}
  \item \xmlNode{available\_capitals}, \xmlDesc{comma/space-separated string, required},
  specifies the available capitals for capital projects or project-option pairs.
  This node accepts the following optional attribute:
  \begin{itemize}
    \item \xmlAttr{index}, \xmlDesc{comma-separated string, optional},
    specifies the indices of this parameter, keywords should be predefined in \xmlNode{Sets}.
    Valid keywords are \xmlString{resources}, \xmlString{time\_periods}, \xmlString{capitals},
    \xmlString{resources, time\_periods}, \xmlString{capitals, time\_periods}
    \default{None}
  \end{itemize}
\end{itemize}

Example XML:
\begin{lstlisting}[style=XML]
<Parameters>
  <net_present_values index='options'>
    27.98 27.17 0.
    -10.07 -9.78 -9.22
    ...
    8.26 7.56 7.34 0.
  </net_present_values>
  <costs index='options, resources, time_periods'>
    12.99 1.3 0 0 0
    ...
    0.01 0 0 0 0
  </costs>
  <available_capitals index="resources,time_periods">
    22.6 36.7 20.6 23.6 22.7
    0.08 0.17 0.05 0.15 0.14
  </available_capitals>
</Parameters>
\end{lstlisting}


%
\subsection{Uncertainties}
\label{subsec:Uncertainties}
This sub-section contains the information regarding the XML nodes used to define the
\xmlNode{Uncertainties} of the optimization model that is being performed through LOGOS:
\begin{itemize}
  \item \xmlNode{available\_capitals}, \xmlDesc{optional}, specifies the scenarios
  associated with available capitals. This node accepts the attribute \xmlAttr{index} which
  should be consistent with \xmlNode{available\_capitals} defined in \xmlNode{Parameters}.
  This node accepts the following sub-nodes:
  \begin{itemize}
    \item \xmlNode{totalScenarios}, \xmlDesc{integer, required}, specifies the total
    number of scenarios for this parameter.
    \item \xmlNode{probabilities}, \xmlDesc{comma/space-separated float, required},
    specifies the probability for each scenario. The length should be equal total number of
    scenarios.
    \item \xmlNode{scenarios}, \xmlDesc{comma/space-separated float, required},
    specifies all scenarios for this parameter. The length should be equal total number
    of scenarios multiply the length of this parameter defined in \xmlNode{Parameters}.
  \end{itemize}

  \item \xmlNode{net\_present\_values}, \xmlDesc{optional}, specifies the scenarios
  associated with net\_present\_values. This node accepts the attribute \xmlAttr{index} which
  should be consistent with \xmlNode{net\_present\_values} defined in \xmlNode{Parameters}.
  \begin{itemize}
    \item \xmlNode{totalScenarios}, \xmlDesc{integer, required}, specifies the total
    number of scenarios for this parameter.
    \item \xmlNode{probabilities}, \xmlDesc{comma/space-separated float, required},
    specifies the probability for each scenario. The length should be equal total number of
    scenarios.
    \item \xmlNode{scenarios}, \xmlDesc{comma/space-separated float, required},
    specifies all scenarios for this parameter. The length should be equal total number
    of scenarios multiply the length of this parameter defined in \xmlNode{Parameters}.
  \end{itemize}
\end{itemize}

Example XML:
\begin{lstlisting}[style=XML]
<Uncertainties>
  <available_capitals index="resources,time_periods">
    <totalScenarios>10</totalScenarios>
    <probabilities>
      0.5, 0.5
    </probabilities>
    <scenarios>
      20.0 34.0 17.0 20.0 18.0 0.08 0.17 0.05 0.15 0.14
      23.0 38.0 22.0 25.0 24.0 0.08 0.17 0.05 0.15 0.14
    </scenarios>
  </available_capitals>
  <net_present_values index='options'>
    <totalScenarios>9</totalScenarios>
    <probabilities>
      0.3 0.8
    </probabilities>
    <scenarios>
      13.3129 12.0228 0.0 -10.07
      ...
    </scenarios>
  </net_present_values>
</Uncertainties>
\end{lstlisting}


%
\subsection{External Constraints}
\label{subsec:ExternalConstraints}

This sub-section contains the information regarding the XML nodes used to define the
\xmlNode{ExternalConstraints} of the optimization model that is being performed through LOGOS.
This node accepts the following sub-node(s):
\begin{itemize}
  \item \xmlNode{constraint}, \xmlDesc{string, required}, specifies the external python
  module file name. This external python module contains the user defined additional constraint
  on the optimization problem. This sub-node also requires the following attribute:
  \begin{itemize}
    \item \xmlAttr{name}, \xmlDesc{string, required}, specifies the name of constraint that will
    be added to optimization problem.
  \end{itemize}
\end{itemize}

Example XML:
\begin{lstlisting}[style=XML]
<ExternalConstraints>
  <constraint name="con_I">externalConst</constraint>
  <constraint name="con_II">externalConstII</constraint>
</ExternalConstraints>
\end{lstlisting}


%
\subsection{Settings: Options for Optimization}
\label{subsec:Settings}

This sub-section contains the information regarding the XML nodes used to define the
\xmlNode{Settings} of the optimization model that is being performed through LOGOS:
\begin{itemize}
  \item \xmlNode{problem\_type}, \xmlDesc{string, required parameter}, specifies the type of
  optimization problem. Available types including \xmlString{SingleKnapsack},
  \xmlString{MultipleKnapsack} and \xmlString{MCKP}.
  \item \xmlNode{solver}, \xmlDesc{string, optional parameter}, available solvers including:
  \xmlNode{cbc} from \url{https://github.com/coin-or/Cbc.git} and \xmlNode{glpk} from
  \url{https://www.gnu.org/software/glpk/}
  \item \xmlNode{sense}, \xmlDesc{string, optional parameter}, specifies the \xmlString{minimize}
  or \xmlString{maximize} for minimization or maximization, respectively.
  \default{minimize}
  \item \xmlNode{regulatoryMandated}, \xmlDesc{comma/space-separated string, optional parameter},
  specifies the regulatory mandated or must do projects.
  \item \xmlNode{nonSelection}, \xmlDesc{boolean, optional parameter}, indicates whether the
  options of investments including \textit{DoNothing} option or not.
  \default{False}
  \item \xmlNode{consistentConstraintI}, \xmlDesc{string, optional parameter}, indicates whether
  this constraint is enabled or not.
  \default{True}
  \item \xmlNode{consistentConstraintII}, \xmlDesc{string, optional parameter}, indicates whether
  this constraint is enabled or not.
  \default{False}
  \item \xmlNode{solverOptions}, \xmlDesc{optional parameter}. This node will accept
  different options for given solver provided in \xmlNode{solver}. A simple XML nodes only contains
  node tag and node text can be used to provide the options for the solver. For example:
  \begin{lstlisting}[style=XML]
    <solverOptions>
      <threads>1</threads>
      <StochSolver>EF</StochSolver>
    </solverOptions>
  \end{lstlisting}
\end{itemize}

Example XML:
\begin{lstlisting}[style=XML]
<Settings>
  <regulatoryMandated>
    PresurizerReplacement
    ...
    ReplaceInstrumentationAndControlCables
  </regulatoryMandated>
  <nonSelection>True</nonSelection>
  <consistentConstraintI>True</consistentConstraintI>
  <consistentConstraintII>True</consistentConstraintII>
  <solver>cbc</solver>
  <solverOptions>
    <threads>1</threads>
    <StochSolver>EF</StochSolver>
  </solverOptions>
  <sense>maximize</sense>
  <problem_type>mckp</problem_type>
</Settings>
\end{lstlisting}
