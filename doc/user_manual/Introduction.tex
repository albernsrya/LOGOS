\section{Introduction}
\label{sec:Introduction}

Industry Equipment Reliability (ER) and Asset Management (AM) Programs are essential elements that
help ensure the safe and economical operation of Nuclear Power Plants (NPPs). The effectiveness of
these programs is addressed in several industry developed and regulatory programs. However, these
programs have proven to be labor intensive and expensive. There is an opportunity to significantly
enhance the collection, analysis, and use of this information to provide more cost-effective plant
operation. LOGOS is providing computational capabilities to optimize plant resources such as
maintenance optimization (ER application) and optimal component replacement schedule (AM application)
by using state-of-the-art discrete optimization methods.

LOGOS is a software package and a RAVEN~\cite{RAVEN,RAVENtheoryMan} plugin which
contains a set of discrete optimization models that can be
employed for capital budgeting optimization problems, and LOGOS integrates economic and reliability
risk in a single analysis framework. More specifically,  Provided Systems, Structures and Components
(SSCs) health (e.g., failure rate or failure probability), OM costs, replacement costs, cost
associated to component failure and budget constraints, LOGOS provides the optimal set of projects
(e.g., SSC replacement) that maximizes profit and satisfies the provided requirements. Input data
listed above can be either deterministic or stochastic in nature, i.e., they can be point values
or probability distribution functions. In the latter case, several scenarios are generated by
sampling the provided distributions.

The developed models are based on different versions of the knapsack optimization algorithms.
Two main classes of optimization models have been initially developed: deterministic and stochastic.
Stochastic optimization models evolve deterministic models by explicitly considering data
uncertainties (associated to constraints or item cost and reward).

These models can be employed as stand-alone models or interfaced with the INL developed RAVEN code
to propagate data uncertainties and analyze the generated data (i.e., sensitivity analysis).

This document provides a detailed description of the LOGOS, and the features included in LOGOS are:
\begin{itemize}
	\item Deterministic Capital Budgeting (See Section~\ref{sec:DeterministicCapitalBudgeting})
	\item Risk-informed stochastic Capital Budgeting (See Section~\ref{sec:StochasticCapitalBudgeting})
	\item Multiple Knapsack problem optimization
	\item Multi-dimensional Knapsack problem optimization
	\item Multi-choice Knapsack problem optimization
	\item Multi-choice multi-dimensional Knapsack problem optimization
	\item SSC cashflow and NPV models
	\item Plugin for the RAVEN code.
\end{itemize}
