\section{Introduction}
\label{sec:Introduction}

Industry equipment reliability (ER) and asset management (AM) programs are essential elements that
help ensure the safe and economical operation of nuclear power plants (NPPs). The effectiveness of
these programs is addressed in several industry-developed and regulatory programs. However, these
programs have proven labor-intensive and expensive. There is an opportunity to significantly
enhance the collection, analysis, and use of this information in order to provide more cost-effective plant
operation. LOGOS is providing computational capabilities to optimize plant resources such as
maintenance optimization (ER application) and optimal component replacement schedule (AM application)
by using state-of-the-art discrete optimization methods.

LOGOS is a software package and RAVEN~\cite{RAVEN,RAVENtheoryMan} plugin that
contains a set of discrete optimization models employable for capital budgeting optimization
problems, and LOGOS integrates economic and reliability
risk into a single analysis framework. More specifically,  given system, structure and component
(SSC) health (e.g., failure rate or probability), O\&M costs, replacement costs, and cost
associated with component failure and budget constraints, LOGOS provides the optimal set of projects
(e.g., SSC replacement) to maximize profit and satisfy the provided requirements.
The aforementioned input data can be either deterministic or stochastic in nature (i.e., they can be point values
or probability distribution functions). In the latter case, several scenarios are generated by
sampling the provided distributions.

The developed models are based on different versions of the knapsack optimization problem.
Two main classes of optimization models were initially developed: deterministic and stochastic.
Stochastic optimization models evolve deterministic models by explicitly considering data
uncertainties associated with constraints or item cost and reward. In FY-20, we moved forward
by implementing two versions of schedule optimization methods. The first one reformulates the
capital budgeting problem in a distributionally robust form which allows the user to rely on
data directly rather than proposing a distribution from the data itself. The second version
reformulates the capital budgeting explicitly using risk measures as variable to maximize or
minimize.

These models can be employed as stand-alone models or interfaced with the INL-developed RAVEN code
to propagate data uncertainties and analyze the generated data (i.e., sensitivity analysis).

\subsection{Acquiring and Installing LOGOS}
LOGOS is supported on three separate computing platforms: Linux, OSX (Apple Macintosh), and Microsoft
Windows. Currently, LOGOS is downloadable from the LOGOS GitLab repository:
\url{https://hpcgitlab.hpc.inl.gov/RAVEN_PLUGINS/LOGOS.git}. New users should contact LOGOS developers to
get started with LOGOS. This typically involves the following steps:

\begin{itemize}
  \item \textit{Download LOGOS}
    \\ You can download the source code for LOGOS from \url{https://hpcgitlab.hpc.inl.gov/RAVEN_PLUGINS/LOGOS.git}.
  \item \textit{Install LOGOS dependencies}
	\begin{lstlisting}[language=bash]
	path/to/LOGOS/build.sh --install
	\end{lstlisting}
  \item \textit{Activate LOGOS Libraries}
  \begin{lstlisting}[language=bash]
  source activate LOGOS_libraries
  \end{lstlisting}
  \item \textit{Test LOGOS}
	\begin{lstlisting}[language=bash]
	./run_tests
	\end{lstlisting}
  	Alternatively, the \texttt{logos} script
    contained in the folder ``\texttt{LOGOS}'' can be directly used:
\begin{lstlisting}[language=bash]
path/to/LOGOS/logos -i <inputFile.xml> -o <outputFile.csv>
\end{lstlisting}
	\item \textit{For use as a RAVEN Plugin}, RAVEN must first be downloaded from
  \url{https://github.com/idaholab/raven.git}.
		\\ Detailed instructions are available from \url{https://github.com/idaholab/raven/wiki}.
    To register a plugin with RAVEN and make its components accessible, run the script:
    \begin{lstlisting}[language=bash]
  	raven/scripts/install_plugins.py -s /abs/path/to/LOGOS
  	\end{lstlisting}
    After the plugin registration, then following the installation instruction at
    \url{https://github.com/idaholab/raven/wiki/installationMain} to install the
    required dependencies.
\end{itemize}

\subsection{User Manual Formats}
In this manual, we employ the following formats to highlight certain parts with
particular meanings (i.e. input structure, examples, and terminal commands):

\begin{itemize}
\item \textbf{\textit{Python Coding:}}
\begin{lstlisting}[language=python]
class AClass():
  def aMethodImplementation(self):
    pass
\end{lstlisting}
\item \textbf{\textit{LOGOS XML input example:}}
\begin{lstlisting}[style=XML,morekeywords={anAttribute}]
<MainXMLBlock>
  ...
  <aXMLnode anAttribute='aValue'>
     <aSubNode>body</aSubNode>
  </aXMLnode>
  <!-- This is  commented block -->
  ...
</MainXMLBlock>
\end{lstlisting}
\item \textbf{\textit{Bash Commands:}}
\begin{lstlisting}[language=bash]
cd path/to/LOGOS/
./build.sh --install
cd ../../
\end{lstlisting}
\end{itemize}

\subsection{Components of LOGOS}
In LOGOS, eXtensible Markup Language (XML) format is used to create the input file. For more
information about XML, please click on the link:
\href{https://www.w3schools.com/xml/default.asp}{\textbf{XML tutorial}}.
%
\\The main input blocks are as follows:
\begin{itemize}
  \item \xmlNode{Logos}: The root node containing the
  entire input; all of
  the following blocks fit inside the \emph{LOGOS} block.
  %
  \item \xmlNode{Settings}: Specifies the calculation settings (i.e. options for
	optimization solvers, options for constraints, and working directory.)
  %
  \item \xmlNode{Sets}: Specifies a collection of data, possibly including
	numeric data (e.g. real or integer values) as well as symbolic data (e.g. strings)
	typically used to specify valid indices for indexed components.
	\nb numeric data provided in the \xmlNode{Sets} would be treated as strings.
  %
	\item \xmlNode{Parameters}: Specifies a collection of parameters, which are
  numerical values used to formulate constraints and objectives in a
	optimization model. A parameter can denote a single value, an array of values, or a multi-dimensional
	array of values.
	%
	\item \xmlNode{Uncertainties}: Specifies a collection of scenarios, which are
	numerical values used to simulate variations within parameters. A scenarios should follow
	the same format as the parameter.
	%
	%
	\item \xmlNode{ExternalConstraints}: Specifies a collection of external constraints, which are
  Python functions used to add additional constraints to the
	current optimization problem.
	%
\end{itemize}

Each of these components is explained in dedicated sections of the user manual.

\subsection{Capabilities of LOGOS}
This document provides a detailed description of LOGOS. The features included in LOGOS are:
\begin{itemize}
  \item Overview of modeling components (see Section~\ref{sec:ModelingComponents})
	\item Deterministic capital budgeting (see Section~\ref{sec:DeterministicCapitalBudgeting})
	\item Prioritizing project selection to hedge against uncertainty (see Section~\ref{sec:StochasticCapitalBudgeting})
  \item Distributionally robust optimization (see Section~\ref{sec:DROCapitalBudgeting})
  \item Risk-based stochastic capital budgeting using conditional Value-at-Risk (See Section)
  \item Plugin for the RAVEN code (see Section~\ref{sec:RavenPlugin})
	\item SSC cashflow and NPV models (see Section~\ref{sec:SSCNPV})
\end{itemize}
