\section{Software Design}
\subsection{Introduction}
Industry Equipment Reliability (ER) and Asset Management (AM) Programs are
essential elements that help ensure the safe and economical operation of
Nuclear Power Plants (NPPs). The effectiveness of these programs is addressed
in several industry developed and regulatory programs. However, these programs
have proven to be labor intensive and expensive. There is an opportunity to
significantly enhance the collection, analysis, and use of this information to
provide more cost-effective plant operation. LOGOS is providing computational
capabilities to optimize plant resources such as maintenance optimization
(ER application) and optimal component replacement schedule (AM application)
by using state-of-the-art discrete optimization methods.

\textit{LOGOS} has been coded using the language \textit{Python}.
The \textit{Python} code provides several different stochastic optimizations
algorithms for capital budgeting problems. It also provides ``external model''
in RAVEN(for installation and usage instructions, see~\cite{RAVENuserManual}).

The input of  \textit{LOGOS} is an XML file, which can be read and
executed by the  \textit{LOGOS}.

\subsection{System Structure (Code)}

\textit{LOGOS} is supported on three separate computing platforms:
Linux, OSX (Apple Macintosh), and Microsoft Windows. Currently, \textit{LOGOS}
is downloadable from the LOGOS GitHub repository:
\url{https://github.com/idaholab/LOGOS.git}. This typically involves the following steps:

\begin{itemize}
  \item \textit{Download LOGOS}
    \\ You can download the source code for LOGOS from \url{https://github.com/idaholab/LOGOS.git}.
  \item \textit{Install LOGOS dependencies}
	\begin{lstlisting}[language=bash]
	path/to/LOGOS/build.sh --install
	\end{lstlisting}
  \item \textit{Activate LOGOS Libraries}
  \begin{lstlisting}[language=bash]
  source activate LOGOS_libraries
  \end{lstlisting}
  \item \textit{Test LOGOS}
	\begin{lstlisting}[language=bash]
	./run_tests
	\end{lstlisting}
  	Alternatively, the \texttt{logos} script
    contained in the folder ``\texttt{LOGOS}'' can be directly used:
\begin{lstlisting}[language=bash]
path/to/LOGOS/logos -i <inputFile.xml> -o <outputFile.csv>
\end{lstlisting}
	\item \textit{For use as a RAVEN Plugin}, RAVEN must first be downloaded from
  \url{https://github.com/idaholab/raven.git}.
		\\ Detailed instructions are available from \url{https://github.com/idaholab/raven/wiki}.
    To register a plugin with RAVEN and make its components accessible, run the script:
    \begin{lstlisting}[language=bash]
  	raven/scripts/install_plugins.py -s /abs/path/to/LOGOS
  	\end{lstlisting}
    After the plugin registration, then following the installation instruction at
    \url{https://github.com/idaholab/raven/wiki/installationMain} to install the
    required dependencies.
\end{itemize}

The addition of \textit{LOGOS} plugin does not require modifying RAVEN
itself. Instead, \textit{LOGOS} module is going to be embedded in
RAVEN at run-time.

\subsection{LOGOS Structure}
LOGOS is developed to solve capital budgeting problems for a nuclear generation station,
with possible extension to a larger fleet of plants. Due to limited resources, we can
only select a subset from a number of candidate investment projects. Our goal is to
maximize overall net present value (NPV), or a variant of this objective, when we
incorporate uncertainty into project cost and revenue streams. In doing so, we must
respect resource limits and capture key structural and stochastic dependencies of
the system. Example projects include upgrading a steam turbine, refurbishing or replacing
a set of reactor coolant pumps, and replacing a set of feed-water heaters. Selecting an
individual project has multiple facets and implications.

\begin{itemize}
  \item \textbf{Rewards or Net Present Values}: Selecting a project can improve revenue (e.g.,
  upgrading a steam turbine may lead to an uprate in plant capacity resulting in larger
  revenue from selling power.) Replacing a key system component can improve reliability,
  increasing revenue due to a reduction in forced outages as well as operations and
  maintenance (O\&M) costs. Choosing to perform minimum maintenance versus refurbishing
  a component or replacing and improving a system can produce reward streams that
  can be negative or positive depending on the selection. Parameter
  \xmlNode{net\_present\_values} is used to specify the rewards (see Section~\ref{subsubsec:Parameters}).

  \item \textbf{Resources and Liabilities}: Critical resources, including (i) capital costs,
  (ii) O\&M costs, (iii) time and labor-hours during a planned outage, and (iv) personnel,
  installation and maintenance of equipment, workspaces, etc.. Within these categories, resources
  can further subcategorized, (witch each subcategory having its own budget), according to the plant’s organizational
  structure to provide multiple “colors” of money within capital costs, O\&M costs,
  personnel availability, etc.. The set \xmlNode{resources} and parameter
  \xmlNode{available\_capitals} are used to specify the resources and
  liabilities (see Section~\ref{subsubsec:Sets} and Section~\ref{subsubsec:Parameters}).

  \item \textbf{Costs}: Selecting a project in year $t$ induces multiple
  cost streams in year $t$ as well as in subsequent years; we interpret “cost” broadly to
  include commitment of critical resources. The parameter \xmlNode{costs} is used to specify
  the costs (see Section~\ref{subsubsec:Parameters}).

  \item \textbf{Time Periods}: Multiple capital projects can compete for same time period,
  limiting project selection. The set \xmlNode{time\_periods} is used to provide indices for
  \textbf{costs} and \textbf{available capitals} (see Section~\ref{subsubsec:Sets}).

  \item \textbf{Options}: The goal of selecting a project is typically to improve or maintain
  a particular function the plant performs, and there may be multiple ways to carry out
  the task. A project may be performed over a three-year period--say, years `$t$, $t+1$, $t+2$'--or the
  start of the project could instead be two years hence, changing the equation to
  `$t+2$, $t+3$, $t+4$'. Alternatively, at increased cost and benefit, it may be
  possible to complete the project in two years: `$t$, $t+1$' or `$t+2$, $t+3$'. When selecting a project
  to uprate plant capacity, we may have the options of increasing it by 3\% or 6\%.
  In all these cases, we can perform the project in, at most, one particular way, out of a collection of
  options. We represent this by cloning a “project” into multiple project-option pairs,
  and adding a constraint saying that we can select, at most, one from this set of options.
  The set \xmlNode{options} is used to provided indices for these multiple project-option pairs
  (see Section~\ref{subsubsec:Sets}).

  \item \textbf{Capitals}: If we consider maintenance for multiple units in an NPP in parallel,
  it has to be decided whether to accept a particular replacement and, in the positive
  case in which unit to conduct the corresponding replacement. In this case, the set \xmlNode{capitals}
  is used to provided indices for these units (see Section~\ref{subsubsec:Sets}).

  \item \textbf{Available Capitals}: This is available budgets for resources/units. The parameter
  \xmlNode{available\_capitals} is used to specify the available capitals for different
  resources/units at different $t$ (see Section~\ref{subsubsec:Parameters}).

  \item \textbf{Non-Selection}: Not selecting a project also has implications, inducing a growth
  in O\&M costs in future years, a decrease in plant production, an increase in forced outages,
  and even risking a premature end to plant life. Thus, not selecting a project can be seen as
  one more “option” for how a larger project is executed, expanding the list discussed earlier.
  Selection is of the “do nothing” option is reflected in both liability streams and reward
  streams. This can be activated by setting \xmlNode{nonSelection} to \xmlString{True}
  (see Section~\ref{subsubsec:Settings}).

  \item \textbf{Uncertainty}: One limitation of traditional optimization models for capital
  budgeting is that they do not account for uncertainty in reward and cost streams associated
  with individual projects, nor do they account for uncertainty in resource availability in
  future years. Projects can incur cost over-runs, especially when projects are large, performed
  infrequently, or when there is uncertainty regarding technical viability, external contractors,
  and/or suppliers of requisite parts and materials. Occasionally, projects are performed ahead
  of schedule and with savings in cost. Planned budgets for capital improvements can be cut, and key
  personnel may be lost. Or, there may be surprise budgetary windfalls for maintenance activities
  due to decreased costs for “unplanned” maintenance. The XML node \xmlNode{Uncertainties} is used
  to specify such uncertainties (see Section~\ref{subsubsec:Uncertainties}).
\end{itemize}

Based on the above description, LOGOS consists of a collection of modeling
entities/components that define different aspects of the model, including
\xmlNode{Sets}, \xmlNode{Parameters}, \xmlNode{Uncertainties}, and
\xmlNode{ExternalConstraints}. In addition, the \xmlNode{Setting}
block specifies how the overall computation should run.

\subsubsection{Sets}
\label{subsubsec:Sets}

This subsection contains information regarding the XML nodes used to define the
\xmlNode{Sets} of the optimization model being performed through LOGOS.
\xmlNode{Sets} specifies a collection of data, possibly including
numeric data (e.g. real or integer values) as well as symbolic data (e.g. strings)
typically used to specify the valid indices for indexed components.
\nb Numeric data provided in \xmlNode{Sets} would be treated as strings.
\xmlNode{Sets} accepts the following additional sub-nodes:
\begin{itemize}
  \item \xmlNode{investments}, \xmlDesc{comma/space-separated string, required}, specifies
  the valid indices for investment projects.
  \item \xmlNode{capitals}, \xmlDesc{comma/space-separated string, optional},
  specifies the indices for NPP units.
  \item \xmlNode{time\_periods}, \xmlDesc{comma/space-separated string, optional},
  specifies the indices for time.
  \item \xmlNode{resources}, \xmlDesc{comma/space-separated string, optional},
  specifies indices for the resources and liabilities.
  \item \xmlNode{options}, \xmlDesc{semi-colon separated list of strings, optional},
  specifies the indices for multiple project-option pairs.
  This sub-node accepts the following attribute:
  \begin{itemize}
    \item \xmlAttr{index}, \xmlDesc{string, required}, specifies the index dependence.
    Valid index is \xmlString{investments}.
  \end{itemize}
\end{itemize}

Example XML:
\begin{lstlisting}[style=XML]
<Sets>
  <investments>
      HPFeedwaterHeaterUpgrade
      PresurizerReplacement
      ...
      ReplaceMoistureSeparatorReheater
  </investments>
  <time_periods>year1 year2 year3 year4 year5</time_periods>
  <resources>CapitalFunds OandMFunds</resources>
  <options index='investments'>
    PlanA PlanB DoNothing;
    PlanA PlanB PlanC;
    ...
    PlanA PlanB PlanC DoNothing
  </options>
</Sets>
\end{lstlisting}

%
\subsubsection{Parameters}
\label{subsubsec:Parameters}
This subsection contains information regarding the XML nodes used to define the
\xmlNode{Parameters} of the optimization model being performed through LOGOS:
\begin{itemize}
  \item \xmlNode{net\_present\_values}, \xmlDesc{comma/space-separated string, required},
  specifies the NPVs for capital projects or project-option pairs. This node accepts the
  following optional attribute:
  \begin{itemize}
    \item \xmlAttr{index}, \xmlDesc{comma-separated string, optional},
    specifies the indices of this parameter; keywords should be predefined in \xmlNode{Sets}.
    Valid keywords are \xmlString{investments} and \xmlString{options}.
    \default{investments}
  \end{itemize}
  \item \xmlNode{costs}, \xmlDesc{comma/space-separated string, required},
  specifies the costs for capital projects or project-option pairs. This node accepts the
  following optional attribute:
  \begin{itemize}
    \item \xmlAttr{index}, \xmlDesc{comma-separated string, optional},
    specifies the indices of this parameter; keywords should be predefined in \xmlNode{Sets}.
    Valid keywords are \xmlString{investments}, \xmlString{investments, time\_periods},
    \xmlString{options}, \xmlString{options, resources}, \xmlString{options, time\_periods},
    and \xmlString{options, resources, time\_periods}.
    \default{`investments'}
  \end{itemize}
  \item \xmlNode{available\_capitals}, \xmlDesc{comma/space-separated string, required},
  specifies the available capitals for capital projects or project-option pairs.
  This node accepts the following optional attribute:
  \begin{itemize}
    \item \xmlAttr{index}, \xmlDesc{comma-separated string, optional},
    specifies the indices of this parameter; keywords should be predefined in \xmlNode{Sets}.
    Valid keywords are \xmlString{resources}, \xmlString{time\_periods}, \xmlString{capitals},
    \xmlString{resources, time\_periods}, and \xmlString{capitals, time\_periods}
    \default{None}
  \end{itemize}
\end{itemize}

Example XML:
\begin{lstlisting}[style=XML]
<Parameters>
  <net_present_values index='options'>
    27.98 27.17 0.
    -10.07 -9.78 -9.22
    ...
    8.26 7.56 7.34 0.
  </net_present_values>
  <costs index='options,resources,time_periods'>
    12.99 1.3 0 0 0
    ...
    0.01 0 0 0 0
  </costs>
  <available_capitals index="resources,time_periods">
    22.6 36.7 20.6 23.6 22.7
    0.08 0.17 0.05 0.15 0.14
  </available_capitals>
</Parameters>
\end{lstlisting}

%
\subsubsection{Uncertainties}
\label{subsubsec:Uncertainties}
This subsection contains information regarding the XML nodes used to define the
\xmlNode{Uncertainties} of the optimization model being performed through LOGOS:
\begin{itemize}
  \item \xmlNode{available\_capitals}, \xmlDesc{optional}, specifies the scenarios
  associated with available capitals. This node accepts the attribute \xmlAttr{index}, which
  should be consistent with the \xmlNode{available\_capitals} defined in \xmlNode{Parameters}.
  This node accepts the following sub-nodes:
  \begin{itemize}
    \item \xmlNode{totalScenarios}, \xmlDesc{integer, required}, specifies the total
    number of scenarios for this parameter.
    \item \xmlNode{probabilities}, \xmlDesc{comma/space-separated float, required},
    specifies the probability for each scenario. The length should be equal to the total number of
    scenarios.
    \item \xmlNode{scenarios}, \xmlDesc{comma/space-separated float, required},
    specifies all scenarios for this parameter. The length should be equal to the total number
    of scenarios multiplied by the length of this parameter, as defined in \xmlNode{Parameters}.
  \end{itemize}

  \item \xmlNode{net\_present\_values}, \xmlDesc{optional}, specifies the scenarios
  associated with net\_present\_values. This node accepts the attribute \xmlAttr{index}, which
  should be consistent with the \xmlNode{net\_present\_values} defined in \xmlNode{Parameters}.
  \begin{itemize}
    \item \xmlNode{totalScenarios}, \xmlDesc{integer, required}, specifies the total
    number of scenarios for this parameter.
    \item \xmlNode{probabilities}, \xmlDesc{comma/space-separated float, required},
    specifies the probability for each scenario. The length should be equal to the total number of
    scenarios.
    \item \xmlNode{scenarios}, \xmlDesc{comma/space-separated float, required},
    specifies all scenarios for this parameter. The length should be equal to the total number
    of scenarios multiplied by the length of this parameter, as defined in \xmlNode{Parameters}.
  \end{itemize}
\end{itemize}

The overall number of scenarios is the total number of scenarios for \xmlNode{available\_capitals}
multiplied by the total number of scenarios for \xmlNode{net\_present\_values}.

Example XML:
\begin{lstlisting}[style=XML]
<Uncertainties>
  <available_capitals index="resources,time_periods">
    <totalScenarios>10</totalScenarios>
    <probabilities>
      0.5, 0.5
    </probabilities>
    <scenarios>
      20.0 34.0 17.0 20.0 18.0 0.08 0.17 0.05 0.15 0.14
      23.0 38.0 22.0 25.0 24.0 0.08 0.17 0.05 0.15 0.14
    </scenarios>
  </available_capitals>
  <net_present_values index='options'>
    <totalScenarios>9</totalScenarios>
    <probabilities>
      0.3 0.8
    </probabilities>
    <scenarios>
      13.3129 12.0228 0.0 -10.07
      ...
    </scenarios>
  </net_present_values>
</Uncertainties>
\end{lstlisting}

%
\subsubsection{External Constraints}
\label{subsubsec:ExternalConstraints}

This subsection contains information regarding the XML nodes used to define the
\xmlNode{ExternalConstraints} of the optimization model being performed through LOGOS.
This node accepts the following sub-node(s):
\begin{itemize}
  \item \xmlNode{constraint}, \xmlDesc{string, required}, specifies the external Python
  module file name along with its absolute or relative path. This external Python
  module contains the user-defined additional constraint.
  \nb If a relative path is specified, the code first checks relative to the working directory,
  then it checks with respect to the location of the input file. The working directory can be
  specified in \xmlNode{Settings} (see Section~\ref{subsubsec:Settings}). In addition, the extension
  `.py' is optional for the module file name that was inputted in this node.
  This sub-node also requires the following attribute:
  \begin{itemize}
    \item \xmlAttr{name}, \xmlDesc{string, required}, specifies the name of the constraint that will
    be added to the optimization problem.
  \end{itemize}
\end{itemize}

Example XML:
\begin{lstlisting}[style=XML]
<ExternalConstraints>
  <constraint name="con_I">externalConst</constraint>
  <constraint name="con_II">externalConstII.py</constraint>
</ExternalConstraints>
\end{lstlisting}

These constraints are Python modules, with a format automatically interpretable by
LOGOS. For example, users can define their own constraint, and the code will be embedded
and use the constraint as though it were an active part of the code itself.
The following provides an example of a user-defined external constraint:

Example Python Function:
\begin{lstlisting}[language=python]
# External constraint function
import numpy as np
import pyomo.environ as pyomo

def initialize():
  """
    Optional Method
    Optimization model parameters values can be updated/modified
    without directly accessing the optimization model.
    Value(s) will be updated in-place.
    @ In, None
    @ Out, updateDict, dict, {paramName:paramInfoDict},  where
      paramInfoDict contains {Indices:Values}
      Indices are parameter indices (either strings or tuples of
      strings, depending on whether there is one or
      more than one dimension). Values are the new values being
      assigned to the parameter at the given indices.
  """
  updateDict = {'available_capitals':{'None':16},
                'costs':{'1':1,'2':3,'3':7,'4':4,'5':8,
                         '6':9,'7':6,'8':10,'9':2,'10':5}
               }
  return updateDict

def constraint(var, sets, params):
  """
    Required Method
    External constraint provided by users that will be added to
    optimization problem
    @ In, sets, dict, all "Sets" provided in the Logos input
      file will be stored and available in this dictionary,
      i.e. {setName: setObject}
    @ In, params, dict, all "Parameters" provided in the
      Logos input file will be stored and
      available in this dictionary, i.e. {paramName: paramObject}
    @ In, var, object, the internally used decision variable,
      the dimensions/indices of this variable depend the type of
      optimization problems (i.e. "<problem_type>" from Logos
      input file). Currently, we will accept the following
      problem types:

      1. "singleknapsack": in this case, "var" will be var[:],
         where the index will be the element from
         xml node of "investment" in Logos input file.

      2. "multipleknapsack": in this case, "var" will be var[:,:],
         where the indices are the combinations element from set
         "investment" and element from set "capitals" in Logos
         input file

      3. "mckp": in this case, "var" will be var[:,:], where the
         indices are the combinations element from set
         "investment" and element from set "options" in Logos
         input file

      (Note that any element that is used as index will be
      converted to a string even if a number is provided in
      the Logos input file).

    @ Out, constraint, tuple, either (constraintRule,)
      or (constraintRule, indices)

    (Note that any modifications in provided sets and params
    will only have impact on this local module,
    i.e. the external constraint. In other words, the Sets
    and Params used in the internal constraints and
    objective will be kept unchanged!)
  """
  # All sets and parameters can be retrieved from dictionary
  # "sets" and "params" investments = sets['investments']

  def constraintRule(self, i):
    """
      Expression for user provided external constraint
      @ In, self, object, required to present, but not used
      @ In, i, str, element for the index set
      @ Out, constraintRule, function expression, expression
        to define user provided constraint

      Note that: Constraints can be indexed by lists or sets.
      When the return of function "constraint" contains
      lists or sets except the "constraintRule", the elements
      are iteratively passed to the rule function. If there
      is more than one, then the cross product is sent.
      For example, this constraint could be interpreted as
      placing limit on "ith" decision variable "var".
      A list of constraints for all "ith" decision variable
      "var" will be added to the optimization model
    """
    return var[i] <= 1

  # A tuple is required for the return, the first element
  # should be always the "constraintRule",
  # while the rest of elements are the lists or sets
  # if the user wants to construct the constraints
  # iteratively (See the docstring in "constraintRule"),
  # otherwise, keep it empty
  return (constraintRule, investments)
\end{lstlisting}

%
\subsubsection{Settings: Options for Optimization}
\label{subsubsec:Settings}

This subsection contains  information regarding the XML nodes used to define the
\xmlNode{Settings} of the optimization model being performed through LOGOS:
\begin{itemize}
  \item \xmlNode{problem\_type}, \xmlDesc{string, required parameter}, specifies the type of
  optimization problem. Available types include \xmlString{SingleKnapsack},
  \xmlString{MultipleKnapsack}, and \xmlString{MCKP} for risk-informed stochastic optimization.
  Available types include \xmlString{droskp}, \xmlString{dromkp}, and \xmlString{dromckp}
  for distributionally robust optimization. Available types include \xmlString{cvarskp},
  \xmlString{cvarmkp}, and \xmlString{cvarmckp}.
  \item \xmlNode{solver}, \xmlDesc{string, optional parameter}, represents available solvers including
  \xmlNode{cbc} from \url{https://github.com/coin-or/Cbc.git} and \xmlNode{glpk} from
  \url{https://www.gnu.org/software/glpk/}.
  \item \xmlNode{sense}, \xmlDesc{string, optional parameter}, specifies \xmlString{minimize}
  or \xmlString{maximize} for minimization or maximization, respectively.
  \default{minimize}
  \item \xmlNode{mandatory}, \xmlDesc{comma/space-separated string, optional parameter},
  specifies regulatorily mandated or must-do projects.
  \item \xmlNode{nonSelection}, \xmlDesc{boolean, optional parameter}, indicates whether the
  investments options includes \textit{DoNothing} option.
  \default{False}
  \item \xmlNode{lowerBounds}, \xmlDesc{comma/space-separated integers, optional parameter}, specifies the lower bounds
  for decision variables.
  \item \xmlNode {upperBounds}, \xmlDesc{comma/space-separated integers, optional parameter}, specifies the upper bounds
  for decision variables.
  \item \xmlNode{consistentConstraintI}, \xmlDesc{string, optional parameter}, indicates whether
  this constraint is enabled.
  \default{True}
  \item \xmlNode{consistentConstraintII}, \xmlDesc{string, optional parameter}, indicates whether
  this constraint is enabled or not.
  \default{False}
  \item \xmlNode{solverOptions}, \xmlDesc{optional parameter}, will accept
  different options for the given solver provided in \xmlNode{solver}. A simple XML node only containing
  node tags and node texts can be used to provide the options for the solver. For example:
  \begin{lstlisting}[style=XML]
    <solverOptions>
      <threads>1</threads>
      <StochSolver>EF</StochSolver>
    </solverOptions>
  \end{lstlisting}
  In addition, if the problem type is distributionally robust optimization, additional option
  \xmlNode{radius\_ambiguity} can be used to control the Wasserstein distance.
  If the problem type is conditional value at risk, additional options are available:
  \begin{itemize}
    \item \xmlNode{risk\_aversion}, \xmlDesc{float within $[0,1]$, optional parameter}, indicates the weight on
    maximizing expected NPV versus penalizing solutions that yield low-NPV scenarios.
    \item \xmlNode{confidence\_level}, \xmlDesc{float within $[0,1]$, optional parameter},  indicates
    the confidence level, i.e. the $\alpha$-percentile of the loss.
  \end{itemize}
\end{itemize}

Example XML:
\begin{lstlisting}[style=XML]
<Settings>
  <mandatory>
    PresurizerReplacement
    ...
    ReplaceInstrumentationAndControlCables
  </mandatory>
  <nonSelection>True</nonSelection>
  <consistentConstraintI>True</consistentConstraintI>
  <consistentConstraintII>True</consistentConstraintII>
  <solver>cbc</solver>
  <solverOptions>
    <threads>1</threads>
    <StochSolver>EF</StochSolver>
  </solverOptions>
  <sense>maximize</sense>
  <problem_type>mckp</problem_type>
</Settings>
\end{lstlisting}

\subsection{Components of LOGOS}
In LOGOS, eXtensible Markup Language (XML) format is used to create the input file. For more
information about XML, please click on the link:
\href{https://www.w3schools.com/xml/default.asp}{\textbf{XML tutorial}}.
%
\\The main input blocks are as follows:
\begin{itemize}
  \item \xmlNode{Logos}: The root node containing the
  entire input; all of
  the following blocks fit inside the \emph{LOGOS} block.
  %
  \item \xmlNode{Settings}: Specifies the calculation settings (i.e. options for
	optimization solvers, options for constraints, and working directory.)
  %
  \item \xmlNode{Sets}: Specifies a collection of data, possibly including
	numeric data (e.g. real or integer values) as well as symbolic data (e.g. strings)
	typically used to specify valid indices for indexed components.
	\nb numeric data provided in the \xmlNode{Sets} would be treated as strings.
  %
	\item \xmlNode{Parameters}: Specifies a collection of parameters, which are
  numerical values used to formulate constraints and objectives in a
	optimization model. A parameter can denote a single value, an array of values, or a multi-dimensional
	array of values.
	%
	\item \xmlNode{Uncertainties}: Specifies a collection of scenarios, which are
	numerical values used to simulate variations within parameters. A scenarios should follow
	the same format as the parameter.
	%
	%
	\item \xmlNode{ExternalConstraints}: Specifies a collection of external constraints, which are
  Python functions used to add additional constraints to the
	current optimization problem.
	%
\end{itemize}

\subsection{LOGOS Plug-in Structure}
The  \textit{LOGOS} plug-in contains the following methods (API from RAVEN):

\begin{lstlisting}[language=python]
from ExternalModelPluginBase import ExternalModelPluginBase

class CapitalInvestmentModel(ExternalModelPluginBase):
  def run (self, container, Inputs)
  def _readMoreXML(self, container, xmlNode)
  def initialize(self, container, runInfo, inputs)
  def createNewInput(self, container, inputs, samplerType, **Kwargs)
\end{lstlisting}
In the following sub-sections all the methods are explained.
\subsubsection{Method: \texttt{run}}
\label{subsubsec:runExternalModelPlugin}
\begin{lstlisting}[language=python]
def run (self, container, Inputs)
\end{lstlisting}

In this function, the LOGOS analysis is coded.
%
The only two attributes this method is going to receive are a Python list of inputs
(the inputs coming from the \texttt{createNewInput} method (see \cite{RAVENuserManual})
and a ``self-like'' object named ``container''.
%
All the outcomes of the LOGOS module will be stored in the above mentioned ``container''
in order to allow RAVEN to collect them.

\subsubsection{Method: \texttt{\_readMoreXML}}
\label{subsubsec:externalReadMoreXMLExternalModelPlugin}
\begin{lstlisting}[language=python]
def _readMoreXML(self, container, xmlNode)
\end{lstlisting}
In this method, the LOGOS input is read and made available to the plug-in and RAVEN.
%
The read information are stored in the ``self-like'' object ``container''
in order to be available to all the other methods, specifically the  \textbf{run}
and  \textbf{initialize} methods.
%
The method receives from RAVEN an attribute of type ``xml.etree.ElementTree'',
containing all the sub-nodes and attribute of the XML block \xmlNode{ExternalModel}.
%

Example XML:
\begin{lstlisting}[style=XML,morekeywords={subType,ModuleToLoad}]
  <Models>
    <ExternalModel name="mkp" subType="LOGOS.CapitalInvestmentModel">
      <variables>available_capitals,i1,i2,i3,i4,i5,i6,i7,i8,
        i9,i10,MaxNPV</variables>
      <ModelData>
        <Sets>
          <investments>
            i1,i2,i3,i4,i5,i6,i7,i8,i9,i10
          </investments>
          <capitals>
            unit_1, unit_2
          </capitals>
        </Sets>

        <Parameters>
          <net_present_values index="investments">
            78, 35, 89, 36, 94, 75, 74, 79, 80, 16
          </net_present_values>
          <costs index="investments">
            18, 9, 23, 20, 59, 61, 70, 75, 76, 30
          </costs>
          <available_capitals index="capitals">
            103, 156
          </available_capitals>
        </Parameters>

        <Settings>
          <solver>glpk</solver>
          <sense>maximize</sense>
          <problem_type>MultipleKnapsack</problem_type>
        </Settings>
      </ModelData>
    </ExternalModel>
  </Models>
\end{lstlisting}

\subsubsection{Method: \texttt{initialize}}
\label{subsubsec:externalInitializeExternalModelPlugin}
\begin{lstlisting}[language=python]
def initialize(self, container, runInfo, inputs)
\end{lstlisting}

The \textbf{initialize} method is implemented  to initialize the capital budgeting analysis
based on the current RAVEN status and LOGOS input file (The XML input file).
%
 \\Indeed, RAVEN is going to call this method at the initialization stage of
 each ``Step'' (see section \cite{RAVENuserManual}).
%
RAVEN will communicate, through a set of method attributes, all the information
that are needed to perform an initialization:
\begin{itemize}
  \item runInfo, a dictionary containing information regarding how the
  calculation is set up (e.g. number of processors, etc.).
  %
  It contains the following attributes:
  \begin{itemize}
    \item \texttt{DefaultInputFile} -- default input file to use
    \item \texttt{SimulationFiles} -- the xml input file
    \item \texttt{ScriptDir} -- the location of the pbs script interfaces
    \item \texttt{FrameworkDir} -- the directory where the framework is located
    \item \texttt{WorkingDir} -- the directory where the framework should be
    running
    \item \texttt{TempWorkingDir} -- the temporary directory where a simulation
    step is run
    \item \texttt{NumMPI} -- the number of mpi process by run
    \item \texttt{NumThreads} -- number of threads by run
    \item \texttt{numProcByRun} -- total number of core used by one run (number
    of threads by number of mpi)
    \item \texttt{batchSize} -- number of contemporaneous runs
    \item \texttt{ParallelCommand} -- the command that should be used to submit
    jobs in parallel (mpi)
    \item \texttt{numNode} -- number of nodes
    \item \texttt{procByNode} -- number of processors by node
    \item \texttt{totalNumCoresUsed} -- total number of cores used by driver
    \item \texttt{queueingSoftware} -- queueing software name
    \item \texttt{stepName} -- the name of the step currently running
    \item \texttt{precommand} -- added to the front of the command that is run
    \item \texttt{postcommand} -- added after the command that is run
    \item \texttt{delSucLogFiles} -- if a simulation (code run) has not failed,
    delete the relative log file (if True)
    \item \texttt{deleteOutExtension} -- if a simulation (code run) has not
    failed, delete the relative output files with the listed extension (comma
    separated list, for example: `e,r,txt')
    \item \texttt{mode} -- running mode, curently the only mode supported is
      mpi (but custom modes can be created)
    \item \textit{expectedTime} -- how long the complete input is expected to
    run
    \item \textit{logfileBuffer} -- logfile buffer size in bytes
  \end{itemize}
  \item inputs, a list of all the inputs that have been specified in the
  ``Step'' using this model.
  %
\end{itemize}

\subsubsection{Method: \texttt{createNewInput}}
\label{subsubsec:externalCreateNewInputExternalModelPlugin}
\begin{lstlisting}[language=python]
  def createNewInput(self, container, inputs, samplerType, **Kwargs)
\end{lstlisting}
The \textbf{createNewInput} method is implemented to create a new LOGOS input
with information coming from the RAVEN framework. In this function, the user can
retrieve the information coming from RAVEN framework, during the employment of
a calculation flow, and use them to construct a new input that is going to be
transferred to the ``run'' method.
This method expects that the new input is returned in a Python dictionary.
RAVEN communicates, through a set of method attributes, all the information
that are generally needed to create a new input:

\begin{itemize}
  \item \texttt{container}, ``self-like'' object, all the outcomes of the LOGOS
    module will be stored in the ``container'' that can be collected by RAVEN.
  \item \texttt{inputs}, python list, a list of all the inputs that have been
    defined in the ``Step'' using this model.
  \item \texttt{samplerType}, string, the type of Sampler, if a sampling strategy
    is employed; will be None otherwise.
  \item \texttt{Kwargs}, dictionary, a dictionary containing several pieces of
    information (that can change based on the ``Step'' type). If a sampling strategy
    is employed, this dictionary contains another dictionary identified by the
    keyword ``SampledVars'', in which the variables perturbed by the sampler are reported.
\end{itemize}
